\documentclass[10pt,preprint]{aastex}

\usepackage{amsfonts}
\usepackage{amsmath}
\usepackage{amssymb}
\usepackage{amsthm}
\usepackage{booktabs}
\usepackage{mathrsfs}
\usepackage{cite}
\usepackage{times}
\usepackage{url}
\usepackage{hyperref}
\usepackage{lineno}
\usepackage{yhmath}
\usepackage{natbib}
\usepackage{../definitions}
\hypersetup{
  bookmarksnumbered = true,
  bookmarksopen=false,
  pdfborder=0 0 0,         % make all links invisible, so the pdf looks good when printed
  pdffitwindow=true,      % window fit to page when opened
  pdfnewwindow=true, % links in new window
  colorlinks=true,           % false: boxed links; true: colored links
  linkcolor=blue,            % color of internal links
  citecolor=magenta,    % color of links to bibliography
  filecolor=magenta,     % color of file links
  urlcolor=cyan              % color of external links
}

\usepackage{graphicx}
\newtheorem*{remark}{Remark}
\graphicspath{{Figures/}}

\begin{document}

\title{Nodal Discontinuous Galerkin Method for the Euler Equations in GR}
\author{Samuel J Dunham\altaffilmark{1}, Eirik Endeve\altaffilmark{2}, et al.}
\altaffiltext{2}{Department of Astronomy, Vanderbilt University, 6301 Stevenson Science Center, Nashville TN, 37212, USA; samuel.j.dunham@vanderbilt.edu}
\altaffiltext{2}{Computational and Applied Mathematics Group, Oak Ridge National Laboratory, Oak Ridge, TN 37831-6354, USA; endevee@ornl.gov}

\tableofcontents

\section{Discontinuous Galerkin Scheme}
\label{sec:dgMethod}

We assume a spacetime metric
\begin{equation}
  ds^{2}=-\alpha^{2}\,dt^{2}+\gamma_{ij}\,dx^{i}\,dx^{j},
\end{equation}
and consider the system of conservation laws with sources
\begin{equation}
  \pd{}{t}\big(\sqrt{\gamma}\,\bU\big)+\sum_{i=1}^{d}\pd{}{i}\big(\alpha\,\sqrt{\gamma}\,\bF^{i}(\bU)\big)=\alpha\,\sqrt{\gamma}\,\bG(\bU),
  \label{eq:conservationLaws}
\end{equation}
where
\begin{align}
  \bU
  &=\big(D,\,S_{j},\,\tau\big)^{\mbox{\tiny T}}
  =\big(\rho\,W,\,\rho\,h\,W^{2}\,v_{j},\,\rho\,h\,W^{2}-p-D\big)^{\mbox{\tiny T}}, \\
  \bF^{i}(\bU)
  &=\big(D\,v^{i},\,\big)^{\mbox{\tiny T}}
\end{align}


\section{Bound-Preserving Methods Using First-Order DG Scheme}

\subsection{Cartesian Coordinates}
This section closely follows \citet{Qin2016}.

\subsubsection{Set of Admissible States}
We consider a one-dimensional system of conservation laws:
\begin{equation*}
    \pd{\vect{U}}{t} + \pd{\vect{F}\left(\vect{U}\right)}{x} = \vect{0},
\end{equation*}
where $\vect{U}$ is a vector of conserved variables, defined as:
\begin{equation}\label{Eq:ConservedVariables}
    \vect{U}\longrightarrow\begin{pmatrix} D\\ S\\ \tau\end{pmatrix}=\begin{pmatrix} \rho\,W\\ \rho\,h\,W^{2}\,v\\ \rho\,W\left(h\,W-1\right)-p\end{pmatrix},
\end{equation}
and $\vect{F}\left(\vect{U}\right)$ are the fluxes of those conserved quantities:
\begin{equation}\label{Eq:FluxVector}
    \vect{F}\left(\vect{U}\right)\longrightarrow\begin{pmatrix}\rho\,W\,v\\ \rho\,h\,W^{2}\,v^{2}+p\\\rho\,h\,W^{2}\,v-D\,v\end{pmatrix}.
\end{equation}


The physics leads us to define a set of admissible states, $\mc{G}_{p}$ (the subscript $p$ stands for primitive), as:
\begin{equation*}
    \mc{G}_{p}\equiv\left\{\vect{U}\Big|\rho>0,\,p>0,\,v^{2}<1\right\}.
\end{equation*}

It is shown in \citet{Mignone2005} that $\mc{G}$ is a convex set\footnote{Convex in the sense that if $\vect{U}_{1}\in\mc{G}$ and $\vect{U}_{2}\in\mc{G}$, then $\alpha_{1}\,\vect{U}_{1}+\alpha_{2}\,\vect{U}_{2}\in\mc{G}$, where $\alpha_{1},\,\alpha_{2}\in\left[0,1\right]$ and $\alpha_{1}+\alpha_{2}=1$.} and can equivalently be written as:
\begin{equation}\label{Eq:SetOfAdmissibleStates}
    \mc{G}\equiv\left\{\vect{U}\Big|D>0,\,\tau+D>\sqrt{D^{2}+S^{2}}\right\}.
\end{equation}

\subsubsection{Time-Step Derivation/CFL Condition}
For the first-order DG method using forward-Euler time-stepping, we evolve the vector of conserved variables as:
\begin{equation}\label{Eq:1stOrderDG}
    \vect{\ol{U}}^{n+1}_{i}=\vect{\ol{U}}^{n}_{i}-\eta\left[\hat{\vect{F}}\left(\vect{\ol{U}}^{n}_{i},\vect{\ol{U}}^{n}_{i+1}\right)-\hat{\vect{F}}\left(\vect{\ol{U}}^{n}_{i-1},\vect{\ol{U}}^{n}_{i}\right)\right],
\end{equation}
where
\begin{equation*}
    \vect{\ol{U}}\equiv\f{1}{\Delta x}\int_{k}\vect{U}\,dx,
\end{equation*}
$\eta\equiv\Delta t/\Delta x$, and $\hat{\vect{F}}$ is the numerical flux. In this document we use the local Lax-Friedrichs flux, defined as:
\begin{equation}\label{Eq:LLF}
    \hat{\vect{F}}\left(a,b\right)=\f{1}{2}\left[\vect{F}\left(a\right)+\vect{F}\left(b\right) - \alpha_{ab}\left(b-a\right)\right],
\end{equation}
where $a$ and $b$ represent the state of the fluid in two different elements, $\alpha_{ab}$ is an estimate for the wave-speed:
\begin{equation*}
    \alpha_{ab}=\text{max}\left[\alpha\left(a\right),\alpha\left(b\right)\right],
\end{equation*}
and $\alpha$ is the largest (in absolute value) eigenvalue of the flux-Jacobian:
\begin{equation*}
    \alpha=\left|\left|\f{\p\vect{F}}{\p\vect{U}}\right|\right|.
\end{equation*}
Using this we define the following variables:
\begin{equation}\label{Eq:EigVals}
    \alpha_{i+\f{1}{2}}=\text{max}\left[\alpha\left(\vect{\ol{U}}_{i}\right),\alpha\left(\vect{\ol{U}}_{i+1}\right)\right],\hspace{3em} \alpha_{i-\f{1}{2}}=\text{max}\left[\alpha\left(\vect{\ol{U}}_{i-1}\right),\alpha\left(\vect{\ol{U}}_{i}\right)\right].
\end{equation}

Substituting \eqref{Eq:LLF} with \eqref{Eq:EigVals} into \eqref{Eq:1stOrderDG}:
\begin{align}
    \vect{\ol{U}}^{n+1}_{i}&=\vect{\ol{U}}^{n}_{i}-\f{\eta}{2}\left[\vect{F}\left(\vect{\ol{U}}^{n}_{i}\right)+\vect{F}\left(\vect{\ol{U}}^{n}_{i+1}\right)-\alpha_{i+\f{1}{2}}\left(\vect{\ol{U}}^{n}_{i+1}-\vect{\ol{U}}^{n}_{i}\right)\right.\nonumber\\
    &\left.\hspace{7em}-\vect{F}\left(\vect{\ol{U}}^{n}_{i}\right)-\vect{F}\left(\vect{\ol{U}}^{n}_{i-1}\right)+\alpha_{i-\f{1}{2}}\left(\vect{\ol{U}}^{n}_{i}-\vect{\ol{U}}^{n}_{i-1}\right)\right]\nonumber\\
    &=\left[1-\f{\eta}{2}\left(\alpha_{i+\f{1}{2}}+\alpha_{i-\f{1}{2}}\right)\right]\vect{\ol{U}}^{n}_{i}+\f{\eta}{2}\,\alpha_{i+\f{1}{2}}\left[\vect{\ol{U}}^{n}_{i+1}-\f{1}{\alpha_{i+\f{1}{2}}}\vect{F}\left(\vect{\ol{U}}^{n}_{i+1}\right)\right]\nonumber\\
    &\hspace{15em}+\f{\eta}{2}\,\alpha_{i-\f{1}{2}}\left[\vect{\ol{U}}^{n}_{i-1}+\f{1}{\alpha_{i-\f{1}{2}}}\vect{F}\left(\vect{\ol{U}}^{n}_{i-1}\right)\right]\nonumber\\
    &=\left[1-\f{\eta}{2}\left(\alpha_{i+\f{1}{2}}+\alpha_{i-\f{1}{2}}\right)\right]\vect{\ol{U}}^{n}_{i}+\f{\eta}{2}\,\alpha_{i+\f{1}{2}}\,\vect{H}^{-}\left(\vect{\ol{U}}^{n}_{i+1},\alpha_{i+\f{1}{2}}\right)+\f{\eta}{2}\,\alpha_{i+\f{1}{2}}\,\vect{H}^{+}\left(\vect{\ol{U}}^{n}_{i-1},\alpha_{i-\f{1}{2}}\right),\label{Eq:ConvComb}
\end{align}
where
\begin{equation}\label{Eq:Hpm}
    \vect{H}^{\pm}\left(\vect{\ol{U}},\alpha\right)\equiv\vect{\ol{U}}\pm\f{1}{\alpha}\,\vect{F}\left(\vect{\ol{U}}\right).
\end{equation}

The proof that $\vect{H}^{\pm}\in\mc{G}$ is given in \citet{Qin2016}. Therefore, we see that with a restrictions on $\alpha_{i\pm\f{1}{2}}$ that \eqref{Eq:ConvComb} is a convex combination. The restriction is (recalling that $\eta=\Delta t/\Delta x$):
\begin{equation*}
    1-\f{\eta}{2}\left(\alpha_{i+\f{1}{2}}+\alpha_{i-\f{1}{2}}\right)>0\implies\f{\eta}{2}\left(\alpha_{i+\f{1}{2}}+\alpha_{i-\f{1}{2}}\right)<1\implies \Delta t<\f{2\,\Delta x}{\alpha_{i+\f{1}{2}}+\alpha_{i-\f{1}{2}}}.
\end{equation*}

We want a time-step that is the same for all elements at a given time, so we tighten the restriction slightly to:
\begin{equation*}
    \Delta t<\f{\Delta x}{\text{max}_{i}\left(\alpha_{i-\f{1}{2}}\right)}.
\end{equation*}

\newpage


\subsection{Curvilinear Coordinates}
NOTE: We assume a conformally-flat, time-independent spatial three-metric.

\subsubsection{Set of Admissible States}
We again consider a one-dimensional system of conservation laws, but this time with a curvilinear metric:
\begin{equation*}
    \pd{\left(\sqrtgm\,\vect{U}\right)}{t}+\pd{\left(\sqrtgm\,\vect{F}\right)}{1}=\sqrtgm\,\vect{S},
\end{equation*}
where $\vect{U}$ is given by:
\begin{equation*}
    \vect{U}\longrightarrow\begin{pmatrix} D\\ S_{1}\\ \tau\end{pmatrix}=\begin{pmatrix} \rho\,W\\ \rho\,h\,W^{2}\,v_{1}\\ \rho\,W\left(h\,W-1\right)-p\end{pmatrix}=\begin{pmatrix} \rho\,W\\ \rho\,h\,W^{2}\,\gamma_{1j}\,v^{j}\\ \rho\,W\left(h\,W-1\right)-p\end{pmatrix},
\end{equation*}
$\vect{F}\left(\vect{U}\right)$ are the fluxes of those conserved quantities:
\begin{equation*}
    \vect{F}\left(\vect{U}\right)\longrightarrow\begin{pmatrix}\rho\,W\,v^{1}\\ \rho\,h\,W^{2}\,v^{1}\,v_{1}+p\,\delta^{1}_{~1}\\\rho\,h\,W^{2}\,v^{1}-D\,v^{1}\end{pmatrix}=\begin{pmatrix}\rho\,W\,v^{1}\\ \rho\,h\,W^{2}\,\gamma_{1j}\,v^{1}\,v^{j}+p\\\rho\,h\,W^{2}\,v^{1}-D\,v^{1}\end{pmatrix},
\end{equation*}
and $\vect{S}$ is a source term:
\begin{align*}
    \vect{S}\longrightarrow\begin{pmatrix}0\\\f{1}{2}\,P^{1k}\,\p_{1}\gamma_{1k}\\0\end{pmatrix}=\begin{pmatrix}0\\\f{1}{2}\,P^{11}\,\p_{1}\gamma_{11}\\0\end{pmatrix}&=\begin{pmatrix}0\\\f{1}{2}\left[S^{1}\,v^{1}+p\,\gamma^{11}\right]\,\p_{1}\gamma_{11}\\0\end{pmatrix}=\begin{pmatrix}0\\\f{1}{2}\gamma^{11}\left[S_{1}\,v^{1}+p\right]\p_{1}\gamma_{11}\\0\end{pmatrix}\\
    &=\begin{pmatrix}0\\\f{1}{2}\left[S_{1}\,v^{1}+p\right]\p_{1}\ln\gamma_{11}\\0\end{pmatrix},
\end{align*}
where the last equality follows because for a conformally-flat spatial three-metric, $\gamma^{11}=1/\gamma_{11}$.

These definitions lead us to define the same set of admissible states as before, namely:
\begin{equation*}
    \mc{G}_{p}\equiv\left\{\vect{U}\Big|\rho>0,\,p>0,\,v^{2}<1\right\},
\end{equation*}
the difference being that $v^{2}$ now involves the metric:
\begin{equation*}
    v^{2}=v^{i}\,v_{i}=v^{1}\,v_{1}=\gamma_{1j}\,v^{1}\,v^{j}.
\end{equation*}

Before continuing, we show that the introduction of the metric doesn't affect the translation between $\mc{G}_{p}$ and $\mc{G}$...\sd{Need to do this}

\subsubsection{Time-Step Derivation/CFL Condition}
We start by integrating both sides over the element and dividing by the volume, $V$:
\begin{equation*}
    \f{1}{V}\int_{k}\pd{\left(\sqrtgm\,\vect{U}\right)}{t}dx+\f{1}{V}\int_{k}\pd{\left(\sqrtgm\,\vect{F}\left(\vect{U}\right)\right)}{x}dx=\f{1}{V}\int_{k}\sqrtgm\,\vect{S}\,dx,
\end{equation*}
where:
\begin{equation*}
    V=\int_{k}dV=\int_{k}\sqrtgm\,dx.
\end{equation*}

By defining the cell-average as:
\begin{equation*}
    \vect{\ol{W}}\equiv\f{1}{V}\int_{k}\vect{W}\,dV,
\end{equation*}
we have:
\begin{equation*}
    \f{d\,\vect{\ol{U}}}{dt}+\f{1}{V}\left.\left(\sqrtgm\,\hat{\vect{F}}\left(\vect{\ol{U}}\right)\right)\right|^{x_{\iph}}_{x_{\imh}}=\vect{\ol{S}},
\end{equation*}
or, using the common notation of the time step being represented as a superscript and the spatial element represented by a subscript:
\begin{equation*}
    \vect{\ol{U}}^{n+1}_{i}=\vect{\ol{U}}^{n}_{i}-\f{\Delta t}{V}\left[\sqrtgm_{\iph}\,\hat{\vect{F}}^{n}_{\iph}-\sqrtgm_{\imh}\,\hat{\vect{F}}^{n}_{\imh}\right]+\Delta t\,\vect{\ol{S}}^{n}_{i}.
\end{equation*}

Now we define a parameter a la \citet{ZS2011b}: $\ve\in\left(0,1\right)$, such that (NOTE: \citet{ZS2011b} set $\ve=1/2$):
\begin{equation*}
    \vect{\ol{U}}^{n}_{i}=\ve\,\vect{\ol{U}}^{n}_{i}+\left(1-\ve\right)\vect{\ol{U}}^{n}_{i}.
\end{equation*}
We can use the first term to balance out the term in the square brackets and the second term to balance out the source term.

So, we get:
\begin{align*}
    \vect{\ol{U}}^{n+1}_{i}&=\ve\left\{\vect{\ol{U}}^{n}_{i}-\f{\Delta t}{\ve\,V}\left[\sqrtgm_{\iph}\,\hat{\vect{F}}^{n}_{\iph}-\sqrtgm_{\imh}\,\hat{\vect{F}}^{n}_{\imh}\right]\right\}+\left(1-\ve\right)\vect{\ol{U}}^{n}_{i}+\Delta t\,\vect{\ol{S}}^{n}_{i}\\
    &=\ve\left\{\vect{\ol{U}}^{n}_{i}-\eta_{i}\left[\sqrtgm_{\iph}\,\hat{\vect{F}}\left(\vect{\ol{U}}^{n}_{i+1},\vect{\ol{U}}^{n}_{i}\right)-\sqrtgm_{\imh}\,\hat{\vect{F}}\left(\vect{\ol{U}}^{n}_{i},\vect{\ol{U}}^{n}_{i-1}\right)\right]\right\}+\left(1-\ve\right)\vect{\ol{U}}^{n}_{i}+\Delta t\,\vect{\ol{S}}^{n}_{i},\hspace{1em}\eta_{i}\equiv\f{\Delta t}{\ve\,V_{i}}.
\end{align*}

We proceed by focusing on each term individually.

\subsubsection{Numerical flux term}
We start with the term in the curly brackets and again we use the Local-Lax-Friedrichs flux, \eqref{Eq:LLF}, yielding:
\begin{align*}
    \vect{\ol{U}}^{n}_{i}-\f{\eta_{i}}{2}\Big\{&\sqrtgm_{\iph}\left[\vect{F}\left(\vect{\ol{U}}^{n}_{i+1}\right)+\vect{F}\left(\vect{\ol{U}}^{n}_{i}\right)-\alpha_{\iph}\left(\vect{\ol{U}}^{n}_{i+1}-\vect{\ol{U}}^{n}_{i}\right)\right]\\
    &-\sqrtgm_{\imh}\left[\vect{F}\left(\vect{\ol{U}}^{n}_{i}\right)+\vect{F}\left(\vect{\ol{U}}^{n}_{i-1}\right)-\alpha_{\imh}\left(\vect{\ol{U}}^{n}_{i}-\vect{\ol{U}}^{n}_{i-1}\right)\right]\Big\}\\
    &\hspace{-10em}=\left(1-\f{1}{2}\,\eta_{i}\,\sqrtgm_{\iph}\,\alpha_{\iph}-\f{1}{2}\,\eta_{i}\,\sqrtgm_{\imh}\,\alpha_{\imh}\right)\vect{\ol{U}}^{n}_{i}\\
    &\hspace{-8em}-\f{1}{2}\,\eta_{i}\,\sqrtgm_{\iph}\,\vect{F}\left(\vect{\ol{U}}^{n}_{i}\right)+\f{1}{2}\,\eta_{i}\,\sqrtgm_{\imh}\,\vect{F}\left(\vect{\ol{U}}^{n}_{i}\right)\\
    &\hspace{-8em}+\f{1}{2}\,\eta_{i}\,\sqrtgm_{\imh}\,\alpha_{\imh}\left[\vect{\ol{U}}^{n}_{i-1}+\f{1}{\alpha_{\imh}}\vect{F}\left(\vect{\ol{U}}^{n}_{i-1}\right)\right]+\f{1}{2}\,\eta_{i}\,\sqrtgm_{\iph}\,\alpha_{\iph}\left[\vect{\ol{U}}^{n}_{i+1}-\f{1}{\alpha_{\iph}}\vect{F}\left(\vect{\ol{U}}^{n}_{i+1}\right)\right].
\end{align*}
Now we add and subtract $\f{1}{2}\,\eta_{i}\,\sqrtgm_{\iph}\,\alpha_{\iph}\,\vect{\ol{U}}^{n}_{i}$ and $\f{1}{2}\,\eta_{i}\,\sqrtgm_{\imh}\,\alpha_{\imh}\,\vect{\ol{U}}^{n}_{i}$, yielding:
\begin{align*}
    &\left(1-\eta_{i}\,\sqrtgm_{\iph}\,\alpha_{\iph}-\eta_{i}\,\sqrtgm_{\imh}\,\alpha_{\imh}\right)\vect{\ol{U}}^{n}_{i}\\
    &+\f{1}{2}\,\eta_{i}\,\sqrtgm_{\iph}\,\alpha_{\iph}\left[\vect{\ol{U}}^{n}_{i}-\f{1}{\alpha_{\iph}}\vect{F}\left(\vect{\ol{U}}^{n}_{i}\right)\right]+\f{1}{2}\,\eta_{i}\,\sqrtgm_{\imh}\,\alpha_{\imh}\left[\vect{\ol{U}}^{n}_{i}+\f{1}{\alpha_{\imh}}\vect{F}\left(\vect{\ol{U}}^{n}_{i}\right)\right]\\
    &+\f{1}{2}\,\eta_{i}\,\sqrtgm_{\imh}\,\alpha_{\imh}\left[\vect{\ol{U}}^{n}_{i-1}+\f{1}{\alpha_{\imh}}\vect{F}\left(\vect{\ol{U}}^{n}_{i-1}\right)\right]+\f{1}{2}\,\eta_{i}\,\sqrtgm_{\iph}\,\alpha_{\iph}\left[\vect{\ol{U}}^{n}_{i+1}-\f{1}{\alpha_{\iph}}\vect{F}\left(\vect{\ol{U}}^{n}_{i+1}\right)\right].
\end{align*}
All of the terms in square brackets are similar to the $\vect{H}$ quantities in \citet{Qin2016}, and are therefore in $\mc{G}$. It can easily be seen that the sum of the coefficients is unity. The final condition is that the coefficient of $\vect{\ol{U}}^{n}_{i}>0$, or (recalling that $\eta=\Delta t/\Delta x$):
\begin{align*}
    1-\eta_{i}\,\sqrtgm_{\iph}\,\alpha_{\iph}&-\eta_{i}\,\sqrtgm_{\imh}\,\alpha_{\imh}>0\implies\eta_{i}\left(\sqrtgm_{\iph}\,\alpha_{\iph}+\sqrtgm_{\imh}\,\alpha_{\imh}\right)<1\\
    &\implies\Delta t<\f{\Delta x}{\sqrtgm_{\iph}\,\alpha_{\iph}+\sqrtgm_{\imh}\,\alpha_{\imh}}.
\end{align*}
Again we want a time-step that is the same for all elements at a given time, so:
\begin{equation*}
    \Delta t<\frac{\Delta x}{2\,\text{max}_{i}\left(\sqrtgm_{\iph}\,\alpha_{\iph}\right)}.
\end{equation*}
We see that this differs from the case of Cartesian coordinates by a factor of two, as well as a factor of the metric determinant at the boundaries of the element. \sd{Why doesn't this reduce to the Cartesian result when $\sqrtgm_{\iph}=1$?}

Next we handle the source term.

\subsubsection{Source term}
We have to show that $\left(1-\ve\right)\vect{\ol{U}}^{n}_{i}+\Delta t\,\vect{\ol{S}}^{n}_{i}\in\mc{G}$. Following \citet{ZS2011b}, we write out explicitly the cell-averages:
\begin{equation*}
    \left(1-\ve\right)\vect{\ol{U}}^{n}_{i}+\Delta t\,\vect{\ol{S}}^{n}_{i}=\f{1-\ve}{V}\int\left[\vect{U}^{n}_{i}+\f{\Delta t}{1-\ve}\vect{S}^{n}_{i}\right]dV.
\end{equation*}
Here we note that for a one-dimensional first-order DG scheme we have:
\begin{equation*}
    \int\vect{W}^{n}_{i}\left(x\right)\,dV=\Delta x\,w_{1}\,\vect{W}^{n}\left(\eta_{1}\right)\,\sqrtgm\left(\eta_{1}\right),
\end{equation*}
and
\begin{equation*}
    \int dV=\Delta x\,w_{1}\,\sqrtgm\left(\eta_{1}\right),
\end{equation*}
so,
\begin{equation*}
    \f{1}{V}\int\vect{W}^{n}_{i}\left(x\right)\,dV=\vect{W}^{n}\left(\eta_{1}\right).
\end{equation*}
So, we have:
\begin{equation*}
    \f{1-\ve}{V}\int\left[\vect{U}^{n}_{i}+\f{\Delta t}{1-\ve}\vect{S}^{n}_{i}\right]dV=\left(1-\ve\right)\left[\vect{U}^{n}\left(\eta_{1}\right)+\f{\Delta t}{1-\ve}\vect{S}^{n}\left(\eta_{1}\right)\right].
\end{equation*}
So we need to show that if $\vect{U}^{n}_{i}\in\mc{G}$, then (omitting the arguments $\eta_{1}$ and superscripts $n$):
\begin{equation*}
    \begin{pmatrix}D\\ S_{1}+\f{\Delta t}{2\left(1-\ve\right)}\left[S_{1}\,v^{1}+p\right]\p_{1}\ln\gamma_{11}\\\tau\end{pmatrix}\in\mc{G}.
\end{equation*}
The first condition of $\mc{G}$ is immediately obvious, i.e. that $D>0$. The second condition reduces to the requirement that the argument of the square root in the second condition of $\mc{G}$ be non-negative:
\begin{equation*}
    D^{2}+\left\{S_{1}+\f{\Delta t}{2\left(1-\ve\right)}\left[S_{1}\,v^{1}+p\right]\p_{1}\ln\gamma_{11}\right\}^{2}\geq0.
\end{equation*}
We define the (scalar) term in the square brackets as $K$, yielding (noting that for the second quantity in curly brackets we convert all the contravariant indices to covariant indices and vice-versa):
\begin{equation*}
    D^{2}+\left\{S_{1}+\f{\Delta t}{2\left(1-\ve\right)}K\,\p_{1}\ln\gamma_{11}\right\}\times\left\{S^{1}+\f{\Delta t}{2\left(1-\ve\right)}K\,\p^{1}\ln\gamma^{11}\right\}\geq0.
\end{equation*}
Here we note that:
\begin{equation*}
    \p^{1}\ln\gamma^{11}=\gamma^{11}\,\p_{1}\ln\gamma^{11}=\gamma^{11}\,\p_{1}\ln\f{1}{\gamma_{11}}=-\gamma^{11}\,\p_{1}\ln\gamma_{11}.
\end{equation*}
Using this, we have:
\begin{equation*}
    D^{2}+\left\{S_{1}+\f{\Delta t}{2\left(1-\ve\right)}K\,\p_{1}\ln\gamma_{11}\right\}\times\left\{S^{1}-\f{\Delta t}{2\left(1-\ve\right)}K\,\gamma^{11}\,\p_{1}\ln\gamma_{11}\right\}\geq0,
\end{equation*}
which gives:
\begin{equation*}
    D^{2}+S_{1}\,S^{1}-\left(\f{K\,\p_{1}\ln\gamma_{11}}{2\left(1-\ve\right)}\right)^{2}\gamma^{11}\,\left(\Delta t\right)^{2}-S_{1}\f{\Delta t}{2\left(1-\ve\right)}K\,\gamma^{11}\,\p_{1}\ln\gamma_{11}+S^{1}\f{\Delta t}{2\left(1-\ve\right)}K\,\p_{1}\ln\gamma_{11}\geq0.
\end{equation*}
Since $S^{1}=\gamma^{11}\,S_{1}$, the last two terms cancel, yielding (recalling that $\gamma^{11}=1/\gamma_{11}$):
\begin{align*}
    D^{2}&+S_{1}\,S^{1}-\left(\f{K\,\p_{1}\ln\gamma_{11}}{2\left(1-\ve\right)}\right)^{2}\gamma^{11}\left(\Delta t\right)^{2}\geq0\\
    &\implies D^{2}+S_{1}\,S^{1}\geq\left(\f{K\,\p_{1}\gamma_{11}}{2\left(1-\ve\right)}\right)^{2}\f{1}{\gamma_{11}^{3}}\left(\Delta t\right)^{2}\\
    &\implies\Delta t\leq\f{2\left(1-\ve\right)\sqrt{D^{2}+S_{1}\,S^{1}}}{\left|\left[S_{1}\,v^{1}+p\right]\p_{1}\gamma_{11}\right|}\gamma_{11}^{3/2}.
\end{align*}

\newpage






\bibliographystyle{apj}
\bibliography{../References/references.bib}

\end{document}