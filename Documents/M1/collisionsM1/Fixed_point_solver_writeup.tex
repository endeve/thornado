\documentclass[12pt]{article}

\usepackage{amsfonts}
\usepackage{amsmath}
\usepackage{amssymb}
\usepackage{amsthm}
\usepackage{booktabs}
\usepackage{mathrsfs}
\usepackage{graphicx}
\usepackage{cite}
\usepackage{times}
\usepackage{url}
\usepackage{hyperref}
\usepackage{lineno}
\usepackage{yhmath}
\usepackage{natbib}
\usepackage{../../definitions}
\hypersetup{
  bookmarksnumbered = true,
  bookmarksopen=false,
  pdfborder=0 0 0,         % make all links invisible, so the pdf looks good when printed
  pdffitwindow=true,      % window fit to page when opened
  pdfnewwindow=true, % links in new window
  colorlinks=true,           % false: boxed links; true: colored links
  linkcolor=blue,            % color of internal links
  citecolor=magenta,    % color of links to bibliography
  filecolor=magenta,     % color of file links
  urlcolor=cyan              % color of external links
}

\newcommand{\IN}{\mbox{\tiny In}}
\newcommand{\OUT}{\mbox{\tiny Out}}
\newcommand{\NNS}{\mbox{\tiny NNS}} % Neutrino-Nucleon/Nuclei Scattering
\newcommand{\NES}{\mbox{\tiny NES}} % Neutrino-Electron Scattering
\newcommand{\PROD}{\mbox{\tiny Pr}}
\newcommand{\ANN}{\mbox{\tiny An}}
\newcommand{\SC}{\mbox{\tiny Sc}}      % Scattering
\newcommand{\TP}{\mbox{\tiny TP}}      % Pair-Processes

\newtheorem{define}{Definition}
\newtheorem{lemma}{Lemma}
\newtheorem{prop}{Proposition}
\newtheorem{rem}{Remark}
\newtheorem{theorem}{Theorem}

\begin{document}

\title{A writeup on fixed point formulations of the collision solver}
\author{Paul Laiu}

\maketitle

\section{Background}
\subsection{Moment Equations}
The moment equation for the number density
\begin{equation}
  \deriv{\cJ_{s}}{t}
  =\widehat{\eta}_{s} - \widehat{\chi}_{s}\,\cJ_{s},
  \label{eq:momentNumber}
\end{equation}
where we have defined the total emissivity 
\begin{equation}
  \widehat{\eta}_{s}(\cJ_{s},\bar{\cJ}_{s})
  =\tilde{\chi}_{s}\,\cJ_{0,s} + \eta_{\SC,s}(\cJ_{s}) + \eta_{\TP,s}(\bar{\cJ}_{s})
\end{equation}
and total opacity
\begin{equation}
  \widehat{\chi}_{s}(\cJ_{s},\bar{\cJ}_{s})
  =\tilde{\chi}_{s} + \chi_{\SC,s}(\cJ_{s}) + \chi_{\TP,s}(\bar{\cJ}_{s}).  
\end{equation}
We have also defined the scattering emissivity
\begin{equation}
  \eta_{\SC,s}(\cJ_{s})
  =\int_{\bbR^{+}}\Phi_{0,s}^{\IN}(\varepsilon,\varepsilon')\,\cJ_{s}(\varepsilon')\,dV_{\varepsilon'},
\end{equation}
the scattering opacity
\begin{equation}
  \chi_{\SC,s}(\cJ_{s})
  =\int_{\bbR^{+}}\big[\,\Phi_{0.s}^{\IN}(\varepsilon,\varepsilon')\,\cJ(\varepsilon')+\Phi_{0,s}^{\Out}(\varepsilon,\varepsilon')\,\big(1-\cJ(\varepsilon')\big)\,\big]\,dV_{\varepsilon'},
\end{equation}
the emissivity due to thermal pair processes
\begin{equation}
  \eta_{\TP,s}(\bar{\cJ}_{s})
  =\int_{\bbR^{+}}\Phi_{0,s}^{\PROD}(\varepsilon,\varepsilon')\,\big(1-\bar{\cJ}_{s}(\varepsilon')\big)\,dV_{\varepsilon'},
\end{equation}
and the opacity due to thermal pair processes
\begin{equation}
  \chi_{\TP,s}(\bar{\cJ}_{s})
  =\int_{\bbR^{+}}\big[\,\Phi_{0,s}^{\PROD}(\varepsilon,\varepsilon')\,\big(1-\bar{\cJ}_{s}(\varepsilon')\big)+\Phi_{0,s}^{\ANN}(\varepsilon,\varepsilon')\,\bar{\cJ}_{s}(\varepsilon')\,\big]\,dV_{\varepsilon'}.  
\end{equation}
Note that the opacities due to scattering and pair processes depend on $\cJ_{s}$ and $\bar{\cJ}_{s}$, respectively, due to the Fermi blocking factors.  
Since $0\le\cJ_{s},\bar{\cJ}_{s}\le1$, we have $\eta_{\SC,s},\chi_{\SC,s},\eta_{\TP,s},\chi_{\TP,s}\ge0$.  
Also note that the emissivities and opacities depend on the neutrino energy $\varepsilon$ and local thermodynamic conditions (e.g., density $\rho$, temperature $T$, and electron fraction $Y_{e}$).  


\subsection{Matter Equations}

Neutrino-matter interactions mediate exchange of lepton number, momentum, and energy between matter and neutrinos.  
As a first approximation, we will assume that the fluid remains static and ignore momentum exchange.  
Only emission and absorption due to electron capture modify $Y_{e}$, while all processes modify the internal energy.  
The electron fraction $Y_{e}$ and internal energy $e=\rho\,\epsilon$ evolve according to
\begin{equation}
  \rho\deriv{Y_{e}}{t}
  =-m_{b}\int_{\bbR^{+}}
  \big[\,
    \tilde{\chi}_{\nu_{e}}\,\big(\cJ_{0,\nu_{e}}-\cJ_{\nu_{e}}\,\big)
    -\tilde{\chi}_{\bar{\nu}_{e}}\,\big(\,\cJ_{0,\bar{\nu}_{e}}-\cJ_{\bar{\nu}_{e}}\,\big)
  \,\big]\,dV_{\varepsilon},
\end{equation}
and
\begin{equation}
  \rho\deriv{\epsilon}{t}
  =-\sum_{s}\int_{\bbR^{+}}\big(\,\widehat{\eta}_{s} - \widehat{\chi}_{s}\,\cJ_{s}\,\big)\,\varepsilon\,dV_{\varepsilon},
\end{equation}
where $\rho$ is the mass density and $m_{b}$ is the average baryon mass.  

The matter sources are constructed to ensure conservation of lepton number and energy.  
The neutrino density $J_{s}(\vect{x},t)$ and energy $E_{s}(\vect{x},t)$ are defined as
\begin{equation}
  \big\{\,J_{s},\,E_{s}\,\big\}
  =\int_{\bbR^{+}}\cJ_{s}(\varepsilon)\big\{\,1,\,\varepsilon\,\big\}\,dV_{\varepsilon}.
\end{equation}
Lepton number conservation is due to
\begin{equation}\label{eq:LeptonConservation}
  \deriv{}{t}\big(\,J_{\nu_{e}}-J_{\bar{\nu}_{e}}\,\big)+\f{\rho}{m_{b}}\deriv{Y_{e}}{t} = 0,
\end{equation}
while energy conservation is due to
\begin{equation}\label{eq:EnergyConservation}
  \sum_{s}\deriv{E_{s}}{t}+\rho\deriv{\epsilon}{t} = 0.
\end{equation}


\section{Fixed-point formulation}


We are interested in solving the number density equation \eqref{eq:momentNumber} with the conservations \eqref{eq:LeptonConservation} and \eqref{eq:EnergyConservation} guaranteed.
For now, we only consider the electron type neutrinos, thus \eqref{eq:LeptonConservation} and \eqref{eq:EnergyConservation} respectively become
\begin{equation}\label{eq:LeptonConservationNue}
  \deriv{}{t}\int_{\bbR^{+}}\cJ_{\nu_{e}}(\varepsilon)\,dV_{\varepsilon}+\f{\rho}{m_{b}}\deriv{Y_{e}}{t} = 0\:,
\end{equation}
and 
\begin{equation}\label{eq:EnergyConservationNue}
  \deriv{}{t}\int_{\bbR^{+}}\varepsilon\cJ_{\nu_{e}}(\varepsilon)\,dV_{\varepsilon}+\rho\deriv{\epsilon}{t} = 0\:,
\end{equation}
where $\cJ_{\nu_{e}}(\varepsilon)$ is governed by (with the $\f{1}{c}$)
\begin{equation}\label{eq:momentNumberNue}
\f{1}{c}  \deriv{\cJ_{\nu_{e}}}{t}
  =\widehat{\eta}_{\nu_{e}} - \widehat{\chi}_{\nu_{e}}\,\cJ_{\nu_{e}}
\end{equation}
with
\begin{equation}
  \widehat{\eta}_{\nu_{e}}(\cJ_{\nu_{e}})
  =\tilde{\chi}_{\nu_{e}}\,\cJ_{0,\nu_{e}} + \eta_{\SC,\nu_{e}}(\cJ_{\nu_{e}})\:,\quad
  \eta_{\SC,\nu_{e}}(\cJ_{\nu_{e}})
  =\int_{\bbR^{+}}\Phi_{0,\nu_{e}}^{\IN}(\varepsilon,\varepsilon')\,\cJ_{\nu_{e}}(\varepsilon')\,dV_{\varepsilon'},
\end{equation}
and 
\begin{equation}
  \widehat{\chi}_{\nu_{e}}(\cJ_{\nu_{e}})
  =\tilde{\chi}_{\nu_{e}} + \chi_{\SC,\nu_{e}}(\cJ_{\nu_{e}})\:,
\end{equation}
\begin{equation}
  \chi_{\SC,\nu_{e}}(\cJ_{\nu_{e}})
  =\int_{\bbR^{+}}\big[\,\Phi_{0,\nu_{e}}^{\IN}(\varepsilon,\varepsilon')\,\cJ_{\nu_{e}}(\varepsilon')+\Phi_{0,\nu_{e}}^{\Out}(\varepsilon,\varepsilon')\,\big(1-\cJ_{\nu_{e}}(\varepsilon')\big)\,\big]\,dV_{\varepsilon'}\:.
\end{equation}

Discretizing \eqref{eq:LeptonConservationNue}--\eqref{eq:momentNumberNue} in time and using numerical integration gives
\begin{equation}\label{eq:LeptonConservationNueDis}
  \sum_{q} w^{(2)}_q \big(\cJ_{\nu_{e},q}^{n+1} - \cJ_{\nu_{e},q}^{n}\big) + \f{\rho}{m_{b}} \big(Y_{e}^{n+1} - Y_{e}^{n} \big) = 0\:,
\end{equation}
\begin{equation}\label{eq:EnergyConservationNueDis}
    \sum_{q} w^{(3)}_q \big(\cJ_{\nu_{e},q}^{n+1} - \cJ_{\nu_{e},q}^{n}\big)+\rho \big(\epsilon^{n+1} - \epsilon^{n} \big) = 0\:,
\end{equation}
and
\begin{equation}
  \cJ_{\nu_{e}}^{n+1} = \cJ_{\nu_{e}}^{n} + c\dt \big(\widehat{\eta}_{\nu_{e}}(\cJ_{\nu_{e}}^{n+1}) - \widehat{\chi}_{\nu_{e}}(\cJ_{\nu_{e}}^{n+1})\,\cJ_{\nu_{e}}^{n+1}\big)\:.
\end{equation}
We rewrite the above equation as
\begin{equation}\label{eq:momentNumberNueDis}
  \cJ_{\nu_{e}}^{n+1} = \f{\cJ_{\nu_{e}}^{n} + c\dt \widehat{\eta}_{\nu_{e}}(\cJ_{\nu_{e}}^{n+1})}{1 + c\dt\widehat{\chi}_{\nu_{e}}(\cJ_{\nu_{e}}^{n+1})}\:.
\end{equation}


We propose two fixed-point formulations for \eqref{eq:LeptonConservationNueDis}, \eqref{eq:EnergyConservationNueDis}, and \eqref{eq:momentNumberNueDis}.

The first one is a fixed-point problem on $(Y_{e}, \,\epsilon,\,\cJ_{\nu_{e}})$, for which Picard iteration takes the form
\begin{subequations}
\begin{equation}\label{eq:}
Y_{e}^{(k+1)} = Y_{e}^{n} + \f{m_{b}}{\rho} \sum_{q} w^{(2)}_q  \cJ_{\nu_{e},q}^{n}  - \f{m_{b}}{\rho} \sum_{q} w^{(2)}_q\cJ_{\nu_{e},q}^{(k)}\:,
\end{equation}
\begin{equation}\label{eq:}
\epsilon^{(k+1)}  = \epsilon^{n} + \f{1}{\rho} \sum_{q} w^{(3)}_q \cJ_{\nu_{e},q}^{n} - \f{1}{\rho} \sum_{q} w^{(3)}_q \cJ_{\nu_{e},q}^{(k)}\:,   
\end{equation}
and
\begin{equation}\label{eq:}
  \cJ_{\nu_{e}}^{(k+1)} = \f{\cJ_{\nu_{e}}^{n} + c\dt \widehat{\eta}_{\nu_{e}}^{(k)}(\cJ_{\nu_{e}}^{(k)})}{1 + c\dt\widehat{\chi}_{\nu_{e}}^{(k)}(\cJ_{\nu_{e}}^{(k)})}\:,
\end{equation}
\end{subequations}
where
\begin{equation}
\widehat{\eta}_{\nu_{e}}^{(k)}
  =\tilde{\chi}_{\nu_{e}}\,\cJ_{0,\nu_{e}} + \eta_{\SC,\nu_{e}}^{(k)} \:,\quad
  \widehat{\chi}_{\nu_{e}}^{(k)}
  =\tilde{\chi}_{\nu_{e}} + \chi_{\SC,\nu_{e}}^{(k)}
\end{equation}
with $\eta_{\SC,\nu_{e}}^{(k)}$ and $\chi_{\SC,\nu_{e}}^{(k)}$ evaluated using $(Y_{e}^{(k)}, \,\epsilon^{(k)})$.


The second formulation is a nested fixed-point problem with the outer one on $(Y_{e}, \,\epsilon)$:
\begin{subequations}
\begin{equation}\label{eq:}
Y_{e}^{(k+1)} = Y_{e}^{n} + \f{m_{b}}{\rho} \sum_{q} w^{(2)}_q  \cJ_{\nu_{e},q}^{n}  - \f{m_{b}}{\rho} \sum_{q} w^{(2)}_q\cJ_{\nu_{e},q}^{(k)}\:,
\end{equation}
\begin{equation}\label{eq:}
\epsilon^{(k+1)}  = \epsilon^{n} + \f{1}{\rho} \sum_{q} w^{(3)}_q \cJ_{\nu_{e},q}^{n} - \f{1}{\rho} \sum_{q} w^{(3)}_q \cJ_{\nu_{e},q}^{(k)}\:,   
\end{equation}
and $\cJ_{\nu_{e},q}^{(k)}$ is given by the solution of an inner fixed-point problem
\begin{equation}
  \cJ_{\nu_{e}}^{(k,\ell+1)} = \f{\cJ_{\nu_{e}}^{n} + c\dt \widehat{\eta}_{\nu_{e}}^{(k)}(\cJ_{\nu_{e}}^{(k,\ell)})}{1 + c\dt\widehat{\chi}_{\nu_{e}}^{(k)}(\cJ_{\nu_{e}}^{(k,\ell)})}\:.
\end{equation}


\end{subequations}


\bibliographystyle{plain}
\bibliography{../../References/references.bib}

\end{document}