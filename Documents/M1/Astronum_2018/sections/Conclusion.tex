\section{Conclusion}\label{se:Conclusion}

We have developed IMEX schemes for two-moment neutrino transport in \texttt{thornado} that respect Fermi-Dirac statistics.
The schemes employ algebraic closure based on Fermi-Dirac statistics, a first-order discontinuous Galerkin method, the simple Lax-Friedrichs flux, and convex-invariant time integration to maintain point-wise realizability of the moments.
Since the realizability-preserving property is obtained from the convexity of the realizable domain, it's possible to construct a scheme with a high-order DG method.
See \cite{chu_etal_2018} for some examples with third-order DG methods.

In the application motivating this work, the neutrino distribution function can vary from 0 to 1.
Hence, we have considered algebraic closure respecting Fermi-Dirac statistics for both low and high occupancy.
Among the seven algebraic closures we considered -- Kershaw\cite{kershaw_1976}, Wilson\cite{wilson_1975,leblancWilson_1970}, Levermore\cite{levermore_1984}, Minerbo \cite{minerbo_1978}, Janka 1\cite{janka_1991}, Janka 2\cite{janka_1992}, and Cernohorsky \& Bludman\cite{cernohorskyBludman_1994} -- only the Cernohorsky \& Bludman closure respected Fermi-Dirac statistics for all occupancies.
As a result, we employed the Cernohorsky \& Bludman closure in all of the numerical tests in Section~\ref{se:NumericalTests}.

Two PD-ARS schemes are proposed for the desired IMEX scheme.
The one with SSPRK2 is second-order accurate and strong-stability preserving in the streaming limit while the other with SSPRK3 is third-order accurate.
Their properties, both accuracy and convex-invariance, were demonstrated with numerical tests.

In this work, we adopted Cartesian coordinates, linear collision term, and a fixed non-relativistic material background.
More realistic problems of scientific interest, such as with the scattering with energy exchange and relativistic effects, are left for future research.