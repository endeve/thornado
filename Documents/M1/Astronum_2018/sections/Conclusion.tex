\section{Conclusion}\label{se:Conclusion}

We have developed IMEX schemes suitable for a two-moment model of neutrino transport in \texttt{thornado} that can obey Fermi-Dirac statistics.  \ee{(What is \texttt{thornado}?  It is mentioned here and in the abstract, and not anywhere else.  Need a paragraph in the intro?)}
The scheme employs algebraic closure based on Fermi-Dirac statistics, high-order discontinuous Galerkin method for spatial discretization, and convex-invariant time integration to maintain realizability of the moments.  
Since the realizability-preserving property is obtained from the convexity of the realizable domain, it is possible to construct a scheme with the high-order DG method \ee{(Not sure what this means.  Does it have to be DG?)}.  
See \cite{chu_etal_2018} for some examples with third-order DG methods.  

In the application motivating this work, the neutrino distribution function can vary from 0 to 1.  
Hence, we have considered algebraic closures based on Fermi-Dirac statistics for both low and high occupancy.  
Among the seven algebraic closures we considered -- Kershaw~\cite{kershaw_1976}, Wilson~\cite{wilson_1975,leblancWilson_1970}, Levermore~\cite{levermore_1984}, Minerbo~\cite{minerbo_1978}, Janka 1~\cite{janka_1991}, Janka 2~\cite{janka_1992}, and Cernohorsky \& Bludman~\cite{cernohorskyBludman_1994} -- only the Cernohorsky \& Bludman closure obeys Fermi-Dirac statistics for all occupancies.  
As a result, we employed the Cernohorsky \& Bludman closure in all of the numerical tests in Section~\ref{se:NumericalTests}.  \ee{(But you said you used Minerbo for the accuracy tests)}

Two PD-ARS schemes are proposed.  
The one with SSPRK2 is second-order accurate and strong-stability preserving in the streaming limit while the other with SSPRK3 is third-order accurate in the streaming limit.  
Their accuracy were demonstrated on problems with known smooth solutions in streaming, absorption, and scattering-dominated regimes.
The neutrino transport test with emission, absorption, and elastic scattering through a stationary background, was designed to test the convex-invariance of our PD-ARS schemes.
The neutrino stationary state test shows that it's necessary to combine an algebraic closure based on Fermi-Dirac statistics and convex-invariant time integration for a solid CCSN simulation.  \ee{(This statement is too strong.  Rewrite.)}

In this work, we adopted Cartesian coordinates, linear collision term, and a fixed non-relativistic material background.
More realistic problems of scientific interest, such as with the scattering with energy exchange and relativistic effects, are left for future research.