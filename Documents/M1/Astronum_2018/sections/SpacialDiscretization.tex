\section{Spatial Discretization}\label{se:SpacialDiscretization}
Here we briefly outline our spatial discretization using the first-order discontinuous Galerkin method.
Let the spatial dimension be $d$ and employ Cartesian coordinates.
We divide the spatial domain $D$ evenly into $N$ cells and denote the $i$th cell as $\bK_{i}$ with $i = 1,\ldots,N$, so that
\begin{equation*}
D = \cup_{i = 1}^{N} \bK_{i} \quad \text{with} \quad
  \bK_{i}=\{\,\vect{x} : x^{i} \in K^{i} := (\xL^{i},\xH^{i}),~|~i=1,\ldots,d\,\}.
\end{equation*}
Each of the cells has a volume, $|\bK|$, with
\begin{equation*}
  |\bK| = \int_{\bK}d\vect{x}, \quad\text{where}\quad d\vect{x} = \prod_{i=1}^{d}dx^{i}.  
\end{equation*}
Then, we have cell-average moments:
\begin{equation}
\bcM_{i} = \dfrac{1}{|\bK|} \int_{\bK_i}\bcM dx.
\end{equation}
Integrating Eq.~\eqref{eq:momentEquations} for each cell $\bK_{i}$ given
\begin{equation}
\dfrac{d \bcM_{i}}{d t} = - \dfrac{1}{|\bK|} \left( \widehat{\bcF}(\bcM_{i},\bcM_{i+1}) -  \widehat{\bcF}(\bcM_{i-1},\bcM_{i})\right) + \f{1}{\tau}\,\cC(\bcM_{i}),
\label{eq:SemiDiscretizatedMomentEquation}
\end{equation}
where $\widehat{\bcF}(\vect{\cM}_{a},\vect{\cM}_{b})$ is the numerical flux and $\f{1}{\tau}\,\cC(\bcM_{i})$ is the integrated collision term.
In this paper we use the simple Lax-Friedrichs flux, given by
\begin{equation}
  \widehat{\bcF}_{LF}(\vect{\cM}_{a},\vect{\cM}_{b})
  =\f{1}{2}\,\big(\,\bcF(\vect{\cM}_{a})+\bcF(\vect{\cM}_{b})-(\,\vect{\cM}_{b}-\vect{\cM}_{a}\,)\,\big).
  \label{eq:Lax-Friedrichs flux}
\end{equation}
Therefore, by treating the flux term explicitly and the collision term implicitly, we have
\begin{align}
\bcM_{i}^{n+1} = \widetilde{\bcM}^{n}_{i} + \f{\dt}{\tau}\,\cC(\bcM^{n+1}_{i})
\label{eq:MomentIMEX}
\end{align}
with
\begin{align}
\widetilde{\bcM}^{n}_{i} 
& = \bcM_{i}^{n} - \dfrac{\dt}{|\bK|} \left( \widehat{\bcF}(\bcM^{n}_{i},\bcM^{n}_{i+1}) -  \widehat{\bcF}(\bcM^{n}_{i-1},\bcM^{n}_{i})\right)\nonumber \\
& = (1-\beta)\bcM_{i}^{n} + \beta\left[ \f{1}{2}\left( \bcM^{n}_{i+1}-\bcF(\bcM^{n}_{i+1})\right)  + \f{1}{2}\left( \bcM^{n}_{i-1}+\bcF(\bcM^{n}_{i-1})\right)\right]
\label{eq:widetildeM}
\end{align}
where $\beta := \frac{\dt}{|\bK|}$.

Assume $\bcM_{i}^{n}$ is realizable for all $i$. 
For Eq.~\eqref{eq:MomentIMEX}, lemma 3 in \cite{chu_etal_2018} says that $\bcM^{n+1}_{i}$ remains realizable given $\f{\dt}{\tau} > 0$ and given that $\widetilde{\bcM}^{n}_{i}$ stay realizable.
Substituting the Lax-Friedrichs flux, Eq.~\eqref{eq:Lax-Friedrichs flux}, in Eq.~\eqref{eq:SemiDiscretizatedMomentEquation}, $\widetilde{\bcM}^{n}_{i}$ becomes a convex combination of $\bcM_{i}^{n}$ and the square bracket in Eq.~\eqref{eq:widetildeM} if $\beta \in [0,1]$, and it is realizable if $\beta \in [0,1]$ and the term in square brackets is realizable.
Given Lemma 2 in \cite{chu_etal_2018}, the term in square brackets is indeed realizable.
Therefore, $\bcM^{n+1}_{i}$ remains realizable given $\beta \in [0,1]$.
Note that $\beta \in [0,1]$ is also the CFL condition of the Forward-Euler method applied to a transport-only Boltzmann equation.
