\documentclass[a4paper]{jpconf}
\usepackage{graphicx}
\usepackage{./definitions}
\usepackage{lineno,hyperref}
\usepackage{amsmath}
\usepackage{amsfonts}
\usepackage{amssymb}
\usepackage{amsthm}
\usepackage{mathrsfs}
\usepackage{color}
\usepackage{url}
\usepackage[numbers]{natbib}

\newcommand{\ee}[1]{{\color{red} EE:~#1}}
\newcommand{\rc}[1]{{\color{blue} RC:~#1}}
\newcommand{\modified}[1]{{\color{red} #1}}

\bibliographystyle{iopart-num}

\begin{document}
\title{\texttt{thornado}-transport: IMEX schemes for two-moment neutrino transport respecting Fermi-Dirac statistics}

\author{Ran Chu}
\address{Department of Physics and Astronomy, University of Tennessee, Knoxville, TN 37996-1200}
\ead{rchu@vols.utk.edu}

\author{Eirik Endeve}
\address{Computational and Applied Mathematics Group, Oak Ridge National Laboratory, Oak Ridge, TN 37831 USA}
\address{Department of Physics and Astronomy, University of Tennessee, Knoxville, TN 37996-1200}
\address{Joint Institute for Computational Sciences, Oak Ridge National Laboratory, Oak Ridge, TN 37831 USA}
\ead{endevee@ornl.gov}

\author{Cory D. Hauck}
\address{Computational and Applied Mathematics Group, Oak Ridge National Laboratory, Oak Ridge, TN 37831 USA}
\address{Department of Mathematics, University of Tennessee, Knoxville, TN 37996-1320}
\ead{hauckc@ornl.gov}

\author{Anthony Mezzacappa}
\address{Department of Physics and Astronomy, University of Tennessee, Knoxville, TN 37996-1200}
\address{Joint Institute for Computational Sciences, Oak Ridge National Laboratory, Oak Ridge, TN 37831 USA}
\ead{mezz@utk.edu}

\author{Bronson Messer}
\address{Scientific Computing and Theoretical Physics Groups, Oak Ridge National Laboratory, Oak Ridge, TN 37831 USA}
\address{Department of Physics and Astronomy, University of Tennessee, Knoxville, TN 37996-1200}
\ead{bronson@ornl.gov}

\begin{abstract}
We develop implicit-explicit (IMEX) schemes for neutrino transport in a background material in the context of a two-moment model that evolves the angular moments of a neutrino phase-space distribution function.
Considering the upper and lower bounds that are introduced by Pauli's exclusion principle on the moments, an algebraic moment closure based on Fermi-Dirac statistics and a convex-invariant time integrator both are demanded.
A finite-volume/first-order discontinuous Galerkin(DG) method is used to illustrate how an algebraic moment closure based on Fermi-Dirac statistics is needed to satisfy the bounds.
Several algebraic closures are compared with these bounds in mind, and the Cernohorsky and Bludman closure, which satisfies the bounds, is chosen for our IMEX schemes.
For the convex-invariant time integrator, two IMEX schemes named PD-ARS have been proposed.
PD-ARS denotes a convex-invariant IMEX Runge-Kutta scheme that is high-order accurate in the streaming limit, and works well in the diffusion limit.
Our two PD-ARS schemes use second- and third-order, explicit, strong-stability-preserving Runge-Kutta methods as their explicit part, respectively, and therefore are second- and third-order accurate in the streaming limit, respectively.
The accuracy and convex-invariance of our PD-ARS schemes are demonstrated in the numerical tests with a third-order DG method for spatial discretization and a simple Lax-Friedrichs flux.
The method has been implemented in our high-order neutrino-radiation hydrodynamics (\texttt{thornado}) toolkit.
We show preliminary results employing tabulated neutrino opacities.
\end{abstract}

\section{Introduction}

Core-collapse supernovae (CCSNe) are the explosions of massive stars that end their lives.
They are the dominant source of heavy elements and play important roles in many astrophysical phenomena, such as neutron star and black hole formation.  
Furthermore, these explosions occur at energies and densities relevant to address fundamental questions in nuclear, particle, and gravitational physics. 
A solid theoretical framework for the CCSN explosion mechanism may help answer important questions in fundamental physics.\cite{janka_etal_2007}

One essential part of the explosion mechanism is neutrino transport.
The neutrino energy deposition drives the CCSN explosion.\cite{mezzacappaMesser_1999}
Ideally, neutrino transport would be modeled by the Boltzmann transport equation, which is an integro-partial-differential equation evolving a phase-space distribution function $f$.\cite{Bruenn_1985}
Simulating the neutrino transport implies finding a solution of the Boltzmann equation for a specific domain and period with acceptable accuracy.

However, solving the Boltzmann transport equation full-dimensionally is expensive.
To balance physical fidelity and computational expediency, an approximate method called the two-moment method has been adopted.\cite{mezzacappaMesser_1999}
Using the two-moment method, the evolved variables are the zeroth and first angular moments of the distribution function $f$ -- the spectral particle density $\cJ$ and flux $\bcH$, respectively.
However, the transport term in the Boltzmann transport equation has the dot product of the velocity vector and the gradient of the phase-space distribution function as its form.
Integrating this term with an unit direction vector as the weight function couples the moment equations: it introduces the second angular moment $\bcK$ to the first moment $\bcH$'s equation.
Knowledge of $\bcK$ is needed to close the two-moment equation system.
Therefore, an algebraic closure that gives a predicted $\bcK$ given $\cJ$ and $\bcH$ is needed.
The better the closure predicts $\bcK$, the more accurate the two-moment method. 
Two-moment method has been widely applied with different algebraic closures, such as the Minerbo\cite{minerbo_1978} closure(e.g.~{O'Connor} and {Couch}\cite{oConnorCouch_2018}, Pan and et al\cite{pan_etal_2018}, Glas et al\cite{glas_etal_2018}, and Just et al\cite{just_etal_2018} ) and the Levermore\cite{levermore_1984} closure(e.g.~Vartanyan et al\cite{vartanyan_etal_2018}, Cabezon et al\cite{cabezon_etal_2018}, and Kuroda et al\cite{kuroda_etal_2016}). 

Applying the two-moment method does simplify the problem, but doesn't guarantee an affordable solution.
To be precise, how to discrete the continuous equation system given by the two-moment method and solve the discretized system efficiently is remain a question.
In fact, the time scales of neutrino interactions with the background ($\sim\mathcal{O}(10^{-13})$~second) is short compared to the duration of the CCSN explosion ($\sim\mathcal{O}(1)$~second).  
This means that a lot of time step would be needed for solving the system fully explicitly. 
On the other hand, solving the moment equations fully implicitly requires inverting globally many band-structured matrices whose sizes depend on the spatial discretization.
Such a global inversion is both expensive and unfriendly to parallelization.
To circumvent these challenges, implicit-explicit (IMEX) methods are taken into consideration.
By treating the transport terms in the two-moment equations explicitly and the collision terms implicitly, IMEX methods are subject only to a time step governed by the explicit transport terms, and the matrices to be inverted are block diagonal.
Therefore, IMEX methods require fewer time steps comparing with fully explicit method, and the calculation for each step is easily parallelizable.  
%For a non-relativistic system where the propagating speed of gravitional wave and fluid is much slower than the light speed, fully implicit method is a better choice for neutrino transport in CCSN simulation.\cite{cernohorskyVanWeert_1992}(read ?)
For a relativistic circumstance that we have, where the gravity, fluid, and neutrino have a relativistic propagating speed, IMEX method is an efficient method.

To model neutrino transport using two-moment method, two things need to be chosen carefully: an algebraic closure based on Fermi-Dirac statistic for closing the two-moment equations and a convex-invariant, diffusion-accurate IMEX scheme to ensure a physical result.
Since the neutrino distribution function is bounded ($f\in[0,1]$) by the Pauli exclusion principle, its moments as weight integrals of a bounded function over domain $\omega\in\bbS^{2}$ are also bounded.
We call the moments satisfying the constraints due to Fermi-Dirac statistics \textit{realizable moments}.
The algebraic closure should give a realizable $\bcK$, and the well-posedness of the closure requires realizable $\cJ$ and $\bcH$.
It explains why an algebraic closure based on Fermi-Dirac statistics is needed.
Realizability of $\cJ$ and $\bcH$ after each time step requires a convex-invariant IMEX scheme.
Since the realizable moments form a convex set, it is possible to construct a realizability-preserving method with convex-invariant IMEX scheme for two-moment neutrino transport.
What's more, the physics of neutrino transport in CCSN requires the IMEX scheme to be diffusion-accurate.

The study of moment realizability and realizability-preserving method with diffusion-accurate IMEX schemes motivate this work.
Gottlieb et al\cite{gottlieb_etal_2001} showed standard IMEX scheme with strong-stability-preserving can't have an order higher than first without a restrict time step requirement.
To have second-order (or higher-order) accuracy, some correction steps are needed.
Unfortunately, the correction steps can deteriorate the accuracy of the IMEX scheme in the diffusion limit or restrict the time step.
To keep things simple, we focus on IMEX schemes without correction steps and require them to be high-order (second or higher order) in the streaming limit and diffusion-accurate.
We call these IMEX schemes \textit{PD-ARS}.

\texttt{thornado} is our toolkit for high-order neutrino-radiation hydrodynamics based on high-order Runge-Kutta Discontinuous Galerkin (RKDG) methods.
It is developed at the University of Tennessee, Knoxville and Oak Ridge National Laboratory.
It currently solves the Euler equations for fluid dynamics and the two-moment approximation of the radiative transfer equation, both in the non-relativistic limits.\cite{endeve_etal_2018}

This paper is organized as follows: Section~\ref{se:Two-MomentModel} discusses the mathematical model, algebraic closures, and the constraints on the moments and algebraic closures imposed by Fermi-Dirac statistics;
Section~\ref{se:SpacialDiscretization} gives a first-order finite-volume spatial discretization and shows how the spatial discretization preserves constraints in an IMEX step;
Section~\ref{se:TimeIntegration} discusses how to use convex combination to construct PD-ARS scheme and two PD-ARS schemes, one with second-order accuracy in the streaming limit and the other with third-order accuracy in the same limit;
Section~\ref{se:NumericalTests} presents the results of the numerical tests, which demonstrate the properties of the PD-ARS schemes; Section~\ref{se:Conclusion} summarizes the achievements of this paper and discusses future works.

%\rc{Reference check}
\section{Two-Moment Model}\label{se:Two-MomentModel}

\subsection{Transport Equations}
Considering the neutrino pass through a static dense matter and only emission, absorption and elastic scattering exist, the transport equation of the neutrino after scaling to dimensionless units can be written as
\begin{equation}
  \pd{f}{t}+\vect{\ell}\cdot\nabla f
  =\f{1}{\tau}\,\cC(f),
  \label{eq:boltzmann}
\end{equation}
which is known as Boltzmann equation.
The distribution function $f = f(\omega,\varepsilon,\vect{x},t)$ gives the number of neutrino propagating in the direction $\omega\in\bbS^{2}$, with neutrino energy $\varepsilon\in\bbR^{+}$, at position $\vect{x}\in\bbR^{3}$ and time $t\in\bbR^{+}$.  
$\vect{\ell} = \vect{\ell}(\omega)\in\bbR^{3}$ is the unit vector which parallel to the neutrino three-momentum direction: $\vect{p}=\varepsilon\,\vect{\ell}$.
On the right-hand side, $\tau$ is interaction strength parameter: $\tau\ll1$ for opaque region where neutrino has strong interaction with the background; $\tau\gg1$ for transparent region where neutrino has weak interaction with the background and streams freely.
$\cC(f)$ is the collision term, which models emission, absorption, and isotropic and elastic scattering: 
\begin{equation}
  \cC(f)=\xi\,\big(\,f_{0}-f\,\big)
  +(1-\xi)\,\big(\,\f{1}{4\pi}\int_{\bbS^{2}}f\,d\omega-f\,\big),
  \label{eq:collisionTerm}
\end{equation}
with $\xi = \sigma_{\Ab} / (\sigma_{\Ab}  + \sigma_{\Scatt} )$ be the emission and absorption contribution parameter: $\xi = 1$ when the scattering opacity $\sigma_{\Scatt} = 0$, pure emission and absorption; $\xi = 0$ when the absorption opacity $\sigma_{\Ab} = 0$, pure scattering. 
$f_{0}$ is the neutrino equilibrium distribution function which has the following form:
\begin{equation}
  f_{0}(\vect{z})=\f{1}{e^{(\varepsilon-\mu(\vect{x}))/T(\vect{x})}+1}.
  \label{eq:fermiDirac}
\end{equation}
$T(\vect{x})$ is the temperature in energy unit and $\mu(\vect{x})$ is the neutrino chemical potential.
Both of them depend on the properties of the background as a function of $\vect{x}$.

\subsection{Two-Moment Model}
An approximate solution of Boltzmann equation Eq.~\eqref{eq:boltzmann} can be found by employing two-moment model.
Define the angular moments of the distribution function as following
\begin{equation}
  \big\{\,\cJ,\vect{\cH},\vect{\cK}\,\big\}(\vect{z},t)
  =\f{1}{4\pi}\int_{\bbS^{2}}f(\omega,\vect{z},t)\,\{\,1,\vect{\ell},\vect{\ell}\otimes\vect{\ell}\,\}\,d\omega,
  \label{eq:angularMoments}
\end{equation}
where $\vect{z}:=\{\varepsilon,\vect{x}\}$, the zeroth moment $\cJ$ is referred as the particle density, the first moment $\bcH$ is the particle flux, and the second moment $\bcK$ is the stress tensor.
Then integral Eq.~\eqref{eq:boltzmann} as the zeroth and the first moments both side to have
\begin{equation}
  \pd{\vect{\cM}}{t}+\nabla\cdot\vect{\cF}=\f{1}{\tau}\,\vect{\cC}(\vect{\cM}),
  \label{eq:momentEquations}
\end{equation}
with $\vect{\cM}=(\cJ,\vect{\cH})^{T}$, $\vect{\cF}=(\vect{\cH},\vect{\cK})^{T}$, and
\begin{equation}
  \vect{\cC}(\vect{\cM})=\vect{\eta}-\vect{\cD}\,\vect{\cM},
  \label{eq:collisionTermMoments}
\end{equation}
where $\vect{\eta}=(\xi\,f_{0},\vect{0})^{T}$, $\vect{\cD}=\mbox{diag}(\xi,\vect{I})$, and
$\vect{I}$ is the identity matrix.
Therefore, solving Boltzmann equation Eq.~\eqref{eq:boltzmann} for a neutrino distribution function $f(\omega,\vect{z},t)$ is converted to solving two-moment equations for the neutrino number density $\cJ(\vect{z},t)$ and the neutrino flux $\bcH(\vect{z},t)$.

\subsection{Algebraic Closures }
As it shown in previous section, the $i$-th order moment equation requires information of the $(i+1)$-th order moment and leads the two-moment system be open. 
To close the two-moment system, a strategy named algebraic closure is built.
For a two-moment method, algebraic closures give an approximate $\bcK$ based on the lower moments as following:
\begin{equation}
\bcK = \vect{k} \cJ,
\end{equation}
where $\vect{k}$ is the Eddington tensor.
Assume the neutrino transport is symmetric about a preferred direction $\widehat{\vect{h}}=\vect{\cH}/|\vect{\cH}|$, Levermore\cite{levermore_1984} proposed
\begin{equation}
  \vect{k}=\f{1}{2}\big[\,\big(1-\chi\big)\,\vect{I}+\big(3\,\chi-1\big)\,\widehat{\vect{h}}\otimes\widehat{\vect{h}}\,\big],
  \label{eq:eddingtonTensor}
\end{equation}
where $\chi=\chi(\cJ,|\vect{\cH}|)$ is the Eddington factor. 

\subsection{Constraints on Moments}
One fundamental fact about neutrino is that they are fermion which follow the Pauli exclusion principle, $f \in (0,1)$.
As a result, the angular moments of $f$ are confined as the following: 
\begin{align}
\cJ \in(0,1), \quad &(1-\cJ)\cJ-|\vect{\cH}|  > 0, \label{eq:MomentsBounds}\\
  \chi_{\mbox{\tiny min}}
  =\max\big(1-\f{2}{3\cJ},h^{2}\big)
  < & \chi<\min\big(1,\f{1}{3\cJ}-\f{\cJ}{1-\cJ}h^{2}\big)=\chi_{\mbox{\tiny max}},
  \label{eq:eddingtonFactorBounds}
\end{align}
where $h = |\bcH|/\cJ$ is the flux factor.[citation]

The inequalities in Eq.~\eqref{eq:MomentsBounds} define the realizable $\bcM$ which defines a convex set in $\cJ -\bcH$ space.
As we will see later in Section \ref{se:SpacialDiscretization}, this convexity makes a constraint-preserving spacial discretization possible.

The inequalities in Eq.~\eqref{eq:eddingtonFactorBounds} desire more attention than have be given.
They have the equal importance as the inequalities in Eq.~\eqref{eq:MomentsBounds} in preserving the validity of the simulation result.
However, as Fig.~\eqref{fig:EddingtonFactorsWithDifferentClosure} shows, not all the algebraic closures satisfy the Eddington factor bounds Eq.~\eqref{eq:eddingtonFactorBounds}: Kershaw\cite{kershaw_1976}, Wilson\cite{wilson_1975,leblancWilson_1970}, Levermore\cite{levermore_1984}, Minerbo \cite{minerbo_1978}, Janka 2\cite{janka_1992} may work fine with neutrino low occupied setting, but not for high occupancy situation; Janka 1\cite{janka_1991} can be dangerous for both the cases.
The only algebraic closure that remains in the bounds among those seven plotted closures is Cernohorsky \& Bludman's \cite{cernohorskyBludman_1994}.

Though Levermore closure and Minerbo closure don't conserve the bounds with high occupied distribution, they are widely used: {O'Connor} and {Couch}\cite{oConnorCouch_2018}, Pan and et al\cite{pan_etal_2018}, Glas et al\cite{glas_etal_2018} and Just et al\cite{just_etal_2018} use Minerbo closure and Vartanyan et al\cite{vartanyan_etal_2018}, Cabezon et al\cite{cabezon_etal_2018} and Kuroda et al\cite{kuroda_etal_2016} use Levermore clsoure.

\begin{figure}[h]
  \centering
  \begin{tabular}{cc}
    \includegraphics[width=0.5\textwidth]{figures/Closures0_10}
    \includegraphics[width=0.5\textwidth]{figures/Closures0_90}
  \end{tabular}
   \caption{Plot of Eddington factors $\chi$ versus flux factor $h$ for different values of $\cJ$ for various algebraic closures: $\cJ=0.1$ (left panel, low occupied) and $\cJ=0.9$ (right panel, high occupied).  In each panel we plot the Eddington factors of Kershaw (red), Wilson (yellow), Levermore (green), Minerbo (light blue), Cernohorsky \& Bludman (blue) and Janka (purple and pink) closures.  We also plot $\chi_{\mbox{\tiny min}}$ and $\chi_{\mbox{\tiny max}}$ defined in Eq.~\eqref{eq:eddingtonFactorBounds} (lower and upper dash black lines, respectively).}
  \label{fig:EddingtonFactorsWithDifferentClosure}
\end{figure}


\section{Spatial Discretization}\label{se:SpacialDiscretization}
Here we use the finite-volume/first-order discontinuous Galerkin method to illustrate how the closure affects the realizability-preserving property of the scheme.
To simply the illustration, let's focus on one-dimension uniform mesh on Cartesian coordinates.
We divide the spatial domain $D$ evenly into $N$ cells and denote the $i$th cell as $\bK_{i}$ with $i = 1,\ldots,N$, so that
\begin{equation*}
D = \cup_{i = 1}^{N} \bK_{i} \quad \text{with} \quad
  \bK_{i}=\{\,x : x\in(x_{i-1/2}, x_{i+1/2})\}.
\end{equation*}
Each of the cells has a length, $|\dx|$ and $|\dx| = D/N = \int_{\bK}d\vect{x}$.
The cell-average moments:
\begin{equation}
\bcM_{i} = \dfrac{1}{|\dx|} \int_{\bK_i}\bcM dx.
\end{equation}
Integrating Eq.~\eqref{eq:momentEquations} for each cell $\bK_{i}$ given
\begin{equation}
\dfrac{d \bcM_{i}}{d t} = - \dfrac{1}{|\dx|} \left( \widehat{\bcF}(\bcM_{i},\bcM_{i+1}) -  \widehat{\bcF}(\bcM_{i-1},\bcM_{i})\right) + \f{1}{\tau}\,\cC(\bcM_{i}),
\label{eq:SemiDiscretizatedMomentEquation}
\end{equation}
where $\widehat{\bcF}(\vect{\cM}_{a},\vect{\cM}_{b})$ is the numerical flux and $\f{1}{\tau}\,\cC(\bcM_{i})$ is the integrated collision term.
In this paper we use the simple Lax-Friedrichs flux, given by
\begin{equation}
  \widehat{\bcF}_{LF}(\vect{\cM}_{a},\vect{\cM}_{b})
  =\f{1}{2}\,\big(\,\bcF(\vect{\cM}_{a})+\bcF(\vect{\cM}_{b})-(\,\vect{\cM}_{b}-\vect{\cM}_{a}\,)\,\big).
  \label{eq:Lax-Friedrichs flux}
\end{equation}
Therefore, by treating the flux term explicitly and the collision term implicitly, we have
\begin{align}
\bcM_{i}^{n+1} = \widetilde{\bcM}^{n}_{i} + \f{\dt}{\tau}\,\cC(\bcM^{n+1}_{i}),
\label{eq:MomentIMEX}
\end{align}
where we define
\begin{align}
\widetilde{\bcM}^{n}_{i} 
& = \bcM_{i}^{n} - \dfrac{\dt}{|\dx|} \left( \widehat{\bcF}(\bcM^{n}_{i},\bcM^{n}_{i+1}) -  \widehat{\bcF}(\bcM^{n}_{i-1},\bcM^{n}_{i})\right)\nonumber \\
& = (1-\beta)\bcM_{i}^{n} + \beta\left[ \f{1}{2}\left( \bcM^{n}_{i+1}-\bcF(\bcM^{n}_{i+1})\right)  + \f{1}{2}\left( \bcM^{n}_{i-1}+\bcF(\bcM^{n}_{i-1})\right)\right],
\label{eq:widetildeM}
\end{align}
and $\beta := \frac{\dt}{|\dx|}$.

Assume $\bcM_{i}^{n}$ is realizable for all $i$. 
For Eq.~\eqref{eq:MomentIMEX}, Lemma 3 in \cite{chu_etal_2018} says that $\bcM^{n+1}_{i}$ is realizable given $\f{\dt}{\tau} > 0$ and provided that $\widetilde{\bcM}^{n}_{i}$ is realizable.
Substituting the Lax-Friedrichs flux, Eq.~\eqref{eq:Lax-Friedrichs flux}, in Eq.~\eqref{eq:SemiDiscretizatedMomentEquation}, $\widetilde{\bcM}^{n}_{i}$ becomes a convex combination of $\bcM_{i}^{n}$ and the square bracket in Eq.~\eqref{eq:widetildeM} if $\beta \in [0,1]$.
$\widetilde{\bcM}^{n}_{i}$ is realizable if $\beta \in [0,1]$ and the term in square brackets is realizable.
Given Lemma 2 in \cite{chu_etal_2018}, the term in square brackets is realizable if $\cJ$, $\bcH$ and $\bcK$ are realizable.
In other words, realizability of the term in square brackets depends on the algebraic closure.
If the closure satisfies Eq.~\eqref{eq:eddingtonFactorBounds},
$\bcM^{n+1}_{i}$ is realizable provided $\beta \in [0,1]$.
Note that $\beta \leq 1$ is also the CFL condition of the Forward-Euler method applied to a transport-only Boltzmann equation.

\clearpage
\section{Time Integration} \label{se:TimeIntegration}

Suppose that an algebraic closure based on Fermi-Dirac statistics is used (i.e., the Eddington factor satisfies Eq.~\eqref{eq:eddingtonFactorBounds}).
Here we consider the construction of an Implicit-Explicit (IMEX) time integration scheme which maintains the bounds in Eq.~\eqref{eq:MomentsBounds}.  
The semi-discretization of the two-moment model results in a system of ordinary differential equations of the form
\begin{equation}
  \dfrac{d \vect{u}}{d t} = \vect{\cT}(\vect{u}) + \f{1}{\tau}\,\vect{\cQ}(\vect{u}),
\end{equation}
where the solution vector
\begin{equation}
  \vect{u}(t) = \left( \bcM_{1}(t),\ldots,\bcM_{N}(t)\right) ^{T}
\end{equation}
is the collection of all cell-averaged moments, $\vect{\cT}$ is the transport operator, corresponding to the first term on the right-hand side of Eq.~\eqref{eq:SemiDiscretizatedMomentEquation}, and $\f{1}{\tau}\,\vect{\cQ}$ is the collision operator, corresponding to the second term on the right-hand side of Eq.~\eqref{eq:SemiDiscretizatedMomentEquation}.  

Since the set of realizable moments is convex, convex-invariant schemes, which maintain states in a convex set, can be used to design realizability-preserving schemes for the two-moment model. 
Ideally, the scheme should also be high-order accurate and work well in the asymptotic diffusion limit (characterized by frequent collisions and long time scales).  
The following discussion considers the construction of such convex-invariant schemes.  

\subsection{Standard IMEX Schemes}

Treating the transport operator explicitly and the collision operator implicitly, a standard $s$-stage IMEX scheme takes the following form \cite{pareschiRusso_2005}: 
\begin{align}
  \vect{u}^{(i)}
  &=\vect{u}^{n}
  +\dt\sum_{j=1}^{i-1}\tilde{a}_{ij}\,\vect{\cT}(\vect{u}^{(j)})
  +\dt\sum_{j=1}^{i}a_{ij}\,\f{1}{\tau}\,\vect{\cQ}(\vect{u}^{(j)}),
  \quad i=1,\ldots,s, \label{imexStages} \\
  \vect{u}^{n+1}
  &=\vect{u}^{n}
  +\dt\sum_{i=1}^{s}\tilde{w}_{i}\,\vect{\cT}(\vect{u}^{(i)})
  +\dt\sum_{i=1}^{s}w_{i}\,\f{1}{\tau}\,\vect{\cQ}(\vect{u}^{(i)}), \label{imexIntermediate} 
\end{align}
where $(\tilde{a}_{ij})$ and $(a_{ij})$, coefficients of the $i$-th stage, are the elements of matrices $\tilde{A}$ and $A$, respectively.
The vectors $\tilde{\vect{w}}=(\tilde{w}_{1},\ldots,\tilde{w}_{s})^{T}$ and $\vect{w}=(w_{1},\ldots,w_{s})^{T}$ are the weights in the assembly step in Eq.~\eqref{imexIntermediate}.
These coefficients and weights must satisfy certain order conditions for consistency and accuracy.  
For second-order temporal accuracy, the following conditions are required:
\begin{equation}
  \sum_{i=1}^{s}\tilde{w}_{i}=\sum_{i=1}^{s}w_{i}=1,
  \label{orderConditions1}
\end{equation}
and
\begin{equation}
  \sum_{i=1}^{s}\tilde{w}_{i}\,\tilde{c}_{i}
  =\sum_{i=1}^{s}\tilde{w}_{i}\,c_{i}
  =\sum_{i=1}^{s}w_{i}\,\tilde{c}_{i}
  =\sum_{i=1}^{s}w_{i}\,c_{i}=\f{1}{2}, 
  \label{orderConditions2}
\end{equation}
where $\tilde{c}_{i} = \sum_{j=1}^{s}\tilde{a}_{ij}$ and $c_{i}=\sum_{j=1}^{s}a_{ij}$.

\subsection{Constraint-Preserving Implicit-Explicit Schemes}
Beyond the accuracy requirement, a convex-invariant IMEX scheme must satisfy additional constraints.
Our idea is finding the constraints on $a_{ij}$, $\tilde{a}_{ij}$, $\tilde{w}_{i}$, and $w_{i}$ that enable each $\vect{u}^{(i)}$ and $\vect{u}^{n+1}$ be convex combination of realizable moments:

Following Hu et al\cite{hu_etal_2018}, the stage values in Eq.~\eqref{imexStages} can be rewritten as
\begin{equation}
  \vect{u}^{(i)}
  =\sum_{j=0}^{i-1}c_{ij}\Big[\,\vect{u}^{(j)}+\hat{c}_{ij}\,\dt\,\vect{\cT}(\vect{u}^{(j)})\,\Big]
  +a_{ii}\,\dt\,\f{1}{\tau}\,\vect{\cQ}(\vect{u}^{(i)}),\quad i=1,\ldots,s,
  \label{eq:imexStagesRewrite}
\end{equation}
with $c_{ij}$ and $\hat{c}_{ij}$ be some expression of $a_{ij}$ and $\tilde{a}_{ij}$.
By requiring the IMEX scheme be globally stiffly accurate (GSA), i.e. $a_{si}=w_{i}$ and $\tilde{a}_{si}=\tilde{w}_{i}$ for $i=1,\ldots,s$, $\vect{u}^{n+1} = \vect{u}^{(s)}$.
Using those Lemma proved in \cite{chu_etal_2018}, we can find those constraints that $c_{ij}$ and $\hat{c}_{ij}$ (or $a_{ij}$ and $\tilde{a}_{ij}$) need to satisfy to ensure $\vect{u}^{(i)}$ and $\vect{u}^{n+1}$ be realizable.

\subsection{Diffusion Accurate, Constraint-Preserving Implicit-Explicit Schemes}
Diffusion accurate is the other property the time integration should have.
In the diffusion limit, the distribution function is nearly isotropic so that $\vect{\cK}\approx\f{1}{3}\,\cJ\,\vect{I}$ and $\vect{\cH}\approx-\f{1}{3}\,\tau\,\nabla\cJ$.
The moment behavior are governed by (e.g., \cite{jinLevermore_1996})
\begin{equation}
  \pd{\cJ}{t} + \nabla\cdot\vect{\cH} = 0
  \quad\text{and}\quad
  \vect{\cH} = - \tau\,\nabla\cdot\vect{\cK}.  
  \label{eq:diffusionLimit}
\end{equation}
In the context of IMEX schemes, the above relationships imply that the relation $A\,\vec{\vect{\cH}}=-\f{1}{3}\,\tau\,\tilde{A}\,\nabla\vec{\cJ}$ should hold. 
We derived the diffusion accurate condition in \cite{chu_etal_2018} and proved that only type~ARS (investigated by Ascher, Ruuth, and Spiteri\cite{ascher_etal_1997}) IMEX schemes can be our candidates.

For type~ARS, $a_{i1} = 0$ for $i=1,\ldots,s$.
$c_{ij}$ and $\hat{c}_{ij}$ are given by
    \begin{equation}
     \begin{aligned}
      c_{i0} &= 1-\sum_{j=2}^{i-1}\sum_{l=j}^{i-1}a_{il}b_{lj}, \quad &
      c_{ij} &= \sum_{l=j}^{i-1}a_{il}b_{lj}, \\
      \tilde{c}_{i0} &= \tilde{a}_{i1}+\sum_{j=2}^{i-1}a_{ij}\tilde{b}_{j1}, \quad &
      \tilde{c}_{ij} &= \tilde{a}_{ij}+\sum_{l=j+1}^{i-1}a_{il}\tilde{b}_{lj},  
     \end{aligned}
     \label{eq:positivityCoefficientsARS}
    \end{equation}
    \begin{equation}
      b_{ii} = \f{1}{a_{ii}}, \quad
      b_{ij} = -\f{1}{a_{ii}}\sum_{l=j}^{i-1}a_{il}b_{lj}, \quad
      \tilde{b}_{ij} = -\f{1}{a_{ii}}\Big(\tilde{a}_{ij}+\sum_{l=j+1}^{i-1}a_{il}\tilde{b}_{lj}\Big).  
    \end{equation}
    Note that $c_{i1}=\tilde{c}_{i1}=0$ in Eq.~\eqref{eq:positivityCoefficientsARS} so that $\sum_{j=0}^{i-1}c_{ij}=1$.
    
Combining the conditions we have, we aim to find type~ARS IMEX scheme with.
\begin{enumerate}
    \item Consistency of the implicit coefficients:
    \begin{equation}
      \sum_{i=1}^{s}w_{i}=1.
    \end{equation}
    \item Second-order accuracy in the streaming limit:
    \begin{equation}
      \sum_{i=1}^{s}\tilde{w}_{i}=1
      \quad\text{and}\quad
      \sum_{i=1}^{s}\tilde{w}_{i}\,\tilde{c}_{i}=\f{1}{2},
      \label{eq:orderConditionsEx}
    \end{equation}
    \item Convex-invariant:
    \begin{align}
      &a_{ii}>0, \quad c_{i0},\tilde{c}_{i0}\ge0, \quad \text{for} \quad i=2,\ldots,s, \nonumber \\
      &\text{and} \quad c_{ij},\tilde{c}_{ij}\ge0, \quad \text{for} \quad i=3,\ldots,s, \quad\text{and}\quad j=2,\ldots,i-1.  
    \end{align}
    with $\sum_{j=0}^{i-1}c_{ij}=1$, for $i=1,\ldots,s$, and $c_{\Sch}=\min_{\substack{i = 2,\ldots,s \\ 
                  j = 0,2,\ldots,i-1}}\,\f{1}{\hat{c}_{ij}}>0$.
    \item Well-behaved in the diffusion limit:
    \begin{equation}
      \vect{e}_{i}^{T}A^{-1}\tilde{A}\,\vect{e} = 1, \quad i=2,\ldots,s,
      \label{eq:diffusionCondition}
    \end{equation}
    where $\vect{e}_{i}$ is the $i$th column of the $s\times s$ identity matrix.
    \item Less than five stages ($s\le4$).
    \item Globally stiffly accurate: $a_{si}=w_{i}$ and $\tilde{a}_{si}=\tilde{w}_{i},\quad i=1,\ldots,s$. 
\end{enumerate}  
We call the IMEX scheme satisfying the above conditions {PD-ARS}. (Definition 3 in \cite{chu_etal_2018}.)
We give two optimized PD-ARS schemes here: PD-ARS with SSPRK2 and  PD-ARS with SSPRK3. 
They have second- and third-order accuracy in the streaming limit, respectively.
\subsubsection{PD-ARS with SSPRK2}
Here we give the optimized PD-ARS with SSPRK2 in the standard double Butcher tableau form: explicit tableau on the left ($\tilde{A}$), implicit tableau on the right ($A$):
\begin{equation}
  \begin{array}{c | c c c}
  	0 & 0   & 0 & 0 \\
  	1 & 1   & 0 & 0 \\
  	1 & 1/2 & 1/2 & 0 \\ \hline
  	  & 1/2 & 1/2 & 0
  \end{array}
  \qquad
  \begin{array}{c | c c c}
  	0 & 0 & 0            & 0            \\
  	1 & 0 & 1            & 0            \\
  	1 & 0 & 1/2 & 1/2 \\ \hline
  	  & 0 & 1/2 & 1/2
  \end{array}
\end{equation}
For this scheme, $c_{\mbox{\tiny Sch}}= 1$ and only two implicit solver are needed for each time step.
\subsubsection{PD-ARS with SSPRK3}
Here we give the optimized PD-ARS with SSPRK3.
Its standard double Butcher tableau are (explicit tableau on the left, implicit tableau on the right)
\begin{equation}
  \begin{array}{c | c c c c}
  	    &     &     &     &  \\
  	 1  & 1   &     &     &  \\
  	1/2 & 1/4 & 1/4 &  \\
  	 1  & 1/6 & 1/6 & 2/3 &  \\ \hline
  	    & 1/6 & 1/6 & 2/3 &
  \end{array}
  \qquad
  \begin{array}{c | c c c c}
  	0 & 0 & 0            & 0            \\
  	1 & 0 & 1            & 0            \\
  	1/2 & 0 & 1/4 & 1/4 \\ 
  	1 & 0 & 1/6 & 1/6 & 2/3\\\hline
  	  & 0 & 1/6 & 1/6 & 2/3
  \end{array}
\end{equation}
For this schemes, $c_{\mbox{\tiny Sch}}= 1$.
Three implicit solver are taken for each time step.
Since PD-ARS with SSPRK3 is as accurate as PD-ARS with SSPRK2 in collision involved region, see Section~\ref{se:NumericalTests} for the plots, it may not offer any advantage.

\clearpage
\section{Numerical Tests}\label{se:NumericalTests}

In this section we present the numerical results obtained with the PD-ARS schemes provided in this paper.
The tests in Section \ref{se: Accuracy Tests} are designed to compare the accuracy of the schemes in various regimes.
The test in Section \ref{se: Neutrino Stationary State Test} demonstrates the realizability-preserving of PD-ARSs scheme.
All the tests in this subsection were applied with third-order accurate spatial discretization (polynomials of degree $k=2$) and time step $\dt = 0.1 \times \dx $.
\subsection{Accuracy Tests}
\label{se: Accuracy Tests}
To compare the accuracy of the IMEX scheme, we applied our PDARSs and Pareschi \& Russo \cite{pareschiRusso_2005} (SSP2332) scheme to some smooth problems in streaming, absorption, ans scattering dominated regimes in one spatial dimension.
All the tests in this subsection were applied with the maximum entropy closure in the low occupancy limit.
In the streaming test, the second-order and third-order accurate explicit strong stability-preserving Runge-Kutta methods\cite{gottlieb_etal_2001} , (SSPRK2 and SSPRK3, respectively) are also applied to serve as an reference.
More information and the definition of the absolute error and the relative error can be found in \cite{chu_2018}.

\subsubsection{Sine Wave Streaming}
This test involves the streaming part only: the collision term is turned off. 
A periodic domain $D$ was applied and the initial condition is given by...
More information can be found in \cite{chu_2018}.
In Figure ~\ref{fig: SineWaveStreaming}, the absolute error for the number density is plotted versus the number of elements $N$.
Errors obtained with SSPRK3 and PD-ARS with SSPRK3 decrease as $N^{-3}$ as expected.
All the other schemes have errors decreases as $N^{-2}$.
In the streaming limit, PD-ARSs degenerate to their explicit part, SSPRK2 and SSPRK3, respectively.
Hence, errors of PD-ARS and SSPRKs are indistinguishable on the plot.
Among the second-order schemes, SSP2332 has the smallest error.
\begin{figure}[h]
  \centering
    \includegraphics[width=0.6\textwidth]{figures/SineWaveStreaming}
   \caption{Absolute error versus number of elements $N$ for the streaming sine wave test.  Results employing various time stepping schemes are compared: SSPRK2 (cyan triangles pointing up), SSPRK3 (cyan triangles pointing down), SSP2332 (green crossing), PD-ARS with SSPRK2 (light red circles) and PD-ARS with SSPRK2 (light red hexagram). Black dash-dot reference lines are proportional to $N^{-1}$ (top), $N^{-2}$ (middle), and $N^{-3}$ (bottom), respectively.}
   \label{fig: SineWaveStreaming}
\end{figure}

\subsubsection{Sine Wave Damping}
This test is adapted from \cite{skinnerOstriker_2013}.
Figure~\ref{fig:SineWaveDamping} shows the numerical result of sine wave damping test.
See \cite{chu_2018} for more information about the test setting.
SSP2332 as a second-order scheme display second order convergence rate.
PD-ARSs convergent in first order, as a price paid for realizability-preserving.
\begin{figure}[h]
  \centering
    \includegraphics[width=0.7\textwidth]{figures/SineWaveDamping}
   \caption{Relative error versus number of elements $N$ for the damping sine wave test. Results for different values of the absorption opacity $\sigma_{\Ab}$, employing various IMEX time stepping schemes, are compared.  Errors for $\sigma_{\Ab}=0.1$, $1$, and $10$ are plotted with red, green, and blue lines, respectively.  The IMEX schemes employed are SSP2332 ($+$), PD-ARS with SSPRK2 (circles) and PD-ARS with SSPRK3 (hexagram).  Black dash-dot reference lines are proportional to $N^{-1}$ (top) and $N^{-2}$ (bottom), respectively.}
  \label{fig:SineWaveDamping}
\end{figure}

\subsubsection{Sine Wave Diffusion}
The last test with known smooth solution is sine wave diffusion test.
It is adopted from \cite{radice_etal_2013}.
See \cite{chu_2018} for more information about the test setting.
Figure~\ref{fig:SineWaveDiffusionJ} shows the numerical result.
SSP2332 and PD-ARSs display third-order accuracy for the number density $\cJ$ and second-oder accuracy for $\cH_{x}$ and their errors are difficult to distinguish.
PD-ARS with SSP2332 behaviors as good as SSP2332 in the diffusion region but requires 1/3 less implicit solver.
\begin{figure}[h]
  \centering
  \begin{tabular}{cc}
    \includegraphics[width=0.5\textwidth]{figures/SineWaveDiffusionJ}
    \includegraphics[width=0.5\textwidth]{figures/SineWaveDiffusionH}
  \end{tabular}
   \caption{Absolute error for the number density $\cJ$ (left) and the number flux $\cH_{x}$ (right) versus number of elements for the sine wave diffusion test.  Results with different values of the scattering opacity $\sigma_{\Scatt}$, employing different IMEX schemes, are compared.  Errors with $\sigma_{\Scatt}=10^{2}$, $10^{3}$, and $10^{4}$ are plotted with red, green, and blue lines, respectively.  The IMEX schemes employed are:  SSP2332 ($+$), PD-ARS with SSPRK2 (circles) and PD-ARS with SSPRK3 (hexagram). Black dash-dot on the left plot reference lines are proportional to $N^{-1}$ (top) and $N^{-2}$ (bottom), respectively. Black dash-dot on the right plot reference lines are proportional to $N^{-2}$ .}
   \label{fig:SineWaveDiffusionJ}
\end{figure}

\subsection{Neutrino Stationary State Test} \label{se: Neutrino Stationary State Test}
Next we consider a more realistic-like test: two-dimensional neutrino transport with emission, absorption and elastic scattering under a stationary realistic-like background.
This test is designed to test the realizability-preserving properties of the PD-ARSs schemes.
Figure~\ref{fig:NeutrinoStationaryTestEOS} plots the thermal state and the opacities we used.
This test is computed on a two-dimensional domain $D=\{\vect{x}\in\bbR^{2}:x^{1}\in[0,200], x^{2}\in[0,200]\}$ with a grid of 128 elements on each direction.
We take 10 energy groups vary from 0~MeV to 300~MeV and initial the neutrino number density $\cJ$ and flux density $\bcH$ both zero(arbitrary small).
For realizability-preserving test, CB closure with PD-ARSs schemes, SSP2332, scheme proposed by Cavaglieri \& Bewley \cite{cavaglieriBewley2015} and scheme proposed by McClarren et al. \cite{mcclarren_etal_2008} are all applied.
Only PD-ARSs schemes produces realizability-preserving result and evolve the model to a stable state.
The realizability-preserving plot and the evolve process of PD-ARSs are given in Figure~\ref{fig:NeutrinoStationaryTestEvolve}. 

\begin{figure}[h]
  \centering
  \begin{tabular}{cc}
    \includegraphics[width=0.45\textwidth]{figures/NStatinaryS_EOS}
    \includegraphics[width=0.45\textwidth]{figures/NSS_Opacities}
  \end{tabular}
   \caption{Thermal state plot (left) and neutrino absorptivity ($\sigma_{\Ab}$) and neutrino elastic scattering opacity ($ \sigma_{\Scatt}$) in space (right) for the neutrino stationary state test.}
   \label{fig:NeutrinoStationaryTestEOS}
\end{figure}

\begin{figure}[h]
  \centering
    \includegraphics[width=\textwidth]{figures/NSS_1_1}\\
    \includegraphics[width=\textwidth]{figures/NSS_3_1} \\
    \includegraphics[width=\textwidth]{figures/NSS_5_1} \\
   \caption{Readability plots (left column), the number density $\cJ$ versus radius plots (center column) and the flux factor $|\bcH|/\cJ$ versus radius plots (right column) at t = 0.01~ms ,0.35~ms and 5.0~ms for the neutrino stationary state test. For the realizability plots, the light blue area demonstrate the realizability domain and the black lines define its boundary. Each $\bcM=(\cJ,\bcH)^{T}$ state is marked by a red dot. The results of PD-ARS with SSPRK2 and PD-ARS with SSPRK3 are indistinguishable in these plot.}
      \label{fig:NeutrinoStationaryTestEvolve}
\end{figure}
\section{Conclusion}\label{se:Conclusion}

We have developed IMEX schemes for two-moment neutrino transport in \texttt{thornado} that respect Fermi-Dirac statistics.
The schemes employ algebraic closure based on Fermi-Dirac statistics, a first-order discontinuous Galerkin method, the simple Lax-Friedrichs flux, and convex-invariant time integration to maintain point-wise realizability of the moments.
Since the realizability-preserving property is obtained from the convexity of the realizable domain, it's possible to construct a scheme with a high-order DG method.
See \cite{chu_etal_2018} for some examples with third-order DG methods.

In the application motivating this work, the neutrino distribution function can vary from 0 to 1.
Hence, we have considered algebraic closure respecting Fermi-Dirac statistics for both low and high occupancy.
Among the seven algebraic closures we considered -- Kershaw\cite{kershaw_1976}, Wilson\cite{wilson_1975,leblancWilson_1970}, Levermore\cite{levermore_1984}, Minerbo \cite{minerbo_1978}, Janka 1\cite{janka_1991}, Janka 2\cite{janka_1992}, and Cernohorsky \& Bludman\cite{cernohorskyBludman_1994} -- only the Cernohorsky \& Bludman closure respected Fermi-Dirac statistics for all occupancies.
As a result, we employed the Cernohorsky \& Bludman closure in all of the numerical tests in Section~\ref{se:NumericalTests}.

Two PD-ARS schemes are proposed for the desired IMEX scheme.
The one with SSPRK2 is second-order accurate and strong-stability preserving in the streaming limit while the other with SSPRK3 is third-order accurate.
Their properties, both accuracy and convex-invariance, were demonstrated with numerical tests.

In this work, we adopted Cartesian coordinates, linear collision term, and a fixed non-relativistic material background.
More realistic problems of scientific interest, such as with the scattering with energy exchange and relativistic effects, are left for future research.
\section{Acknowledgment} \label{se:Acknowledgment}

This research is sponsored, in part, by the Laboratory Directed Research and Development Program of Oak Ridge National Laboratory (ORNL), managed by UT-Battelle, LLC for the U.S. Department of Energy under Contract No. De-AC05-00OR22725. This material is based, in part, on work supported by the U.S. Department of Energy, Office of Science, Office of Advanced Scientific Computing Research. This research was also supported by the Exascale Computing Project (17-SC-20-SC), a collaborative effort of the U.S. Department of Energy Office of Science and the National Nuclear Security Administration, and by the National Science Foundation Gravitational Physics Program (NSF-GP 1806692).

%\section*{References}
\bibliography{references/references}

\end{document}


