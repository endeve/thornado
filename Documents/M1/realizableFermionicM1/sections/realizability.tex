\section{Moment Realizability and Closures for Fermion}
\label{sec:realizability}

Our goal is to design a numerical method for solving the system of equations given by Eq.~\eqref{eq:momentEquations}, which approximates Eq.~\eqref{eq:boltzmann}.  
To this end, we want to obtain solutions that preserve features of the original problem. 

\subsection{Moment Realizability}

\begin{define}
  The moments $\vect{\cM}=\big(\cJ,\vect{\cH}\big)^{T}$ are realizable if they are consistent with a distribution function satisfying $0\le f \le 1$.
\end{define}
\begin{lemma}
  Let the distribution function $f$ be bounded by $0\le f \le1$.  
  Then $0\le\cJ\le1$ and $\big(1-\cJ\big)\cJ-|\vect{\cH}|\ge0$
\end{lemma}
\begin{proof}
  Since this lemma is independent from $\vect{z}$ and t, we can drop $\vect{z}$ and t from the expressions for this proof.
  Let's prove $0\le\cJ\le1$.
  
  Since $f$ is positive, we have
    \begin{equation}
      \cJ=\f{1}{4\pi}\int_{\bbS^{2}}f(\omegaNu)\,d\omegaNu\ge0.
    \end{equation}
    
  Since $1-f$ is positive, we have
    \begin{equation}
      1-\cJ=\f{1}{4\pi}\int_{\bbS^{2}}(1-f(\omegaNu))\,d\omegaNu\ge0.
    \end{equation}
    Therefore, $0\le\cJ\le1$.

    For $\big(1-\cJ\big)\cJ-|\vect{\cH}|\ge0$, we can prove it's valid with 1-dimension $\vect{\cH}$ first.
    In 1-dimension, $\vect{\ell}(\omegaNu) = \omegaNu$ and
    \begin{equation}
    \vect{\cH}
      =\f{1}{4\pi}\int_{\bbS^{2}}f(\omegaNu) \omegaNu d\omegaNu = \f{1}{2}\int_{-1}^{1}f(\omegaNu) \omegaNu d\omegaNu.
    \end{equation}
    Following Banach \& Larecki \cite{banachLarecki_2017}, we can rewrite it to be
    \begin{align}
    \vect{\cH}
        &=(1-\cJ)\,\cJ \nonumber \\
        &-\Big\{\,
          \f{1}{2}\int_{-1}^{1-2\cJ}(1-2\cJ-\omegaNu)f(\omegaNu)\,d\omegaNu
          +\f{1}{2}\int_{1-2\cJ}^{1}(2\cJ-1+\omegaNu)[1-f(\omegaNu)]\,d\omegaNu
        \,\Big\}.
        \label{eq:momentumRewrite1}
    \end{align}
    Since $(1-2\cJ-\omegaNu)f(\omegaNu)$ is positive when $-1\leq\omegaNu\leq1-2\cJ$ and $(2\cJ-1+\omegaNu)[1-f(\omegaNu)]$ is positive when $1-2\cJ\leq\omegaNu\leq1$,
    \begin{align}
    \vect{\cH} \leq (1-\cJ)\,\cJ.
    \label{eq:HupBry1D}
    \end{align}
    Similarly, we can also rewrite $\vect{\cH}$ as
    \begin{align}
     \vect{\cH}
        &=-(1-\cJ)\,\cJ \nonumber \\
        &-\Big\{\,
          \f{1}{2}\int_{-1}^{2\cJ-1}(1-2\cJ+\omegaNu)[1-f(\omegaNu)]\,d\omegaNu
          +\f{1}{2}\int_{2\cJ-1}^{1}(2\cJ-1-\omegaNu)f(\omegaNu)\,d\omegaNu
        \,\Big\}.
        \label{eq:momentumRewrite2}
    \end{align}
    Since $(1-2\cJ+\omegaNu)[1-f(\omegaNu)]$ is negative when $-1\leq\omegaNu\leq2\cJ-1$ and $(2\cJ-1-\omegaNu)f(\omegaNu)$ is negative when $2\cJ-1\leq\omegaNu\leq1$,
    \begin{align}
    \vect{\cH} \geq -(1-\cJ)\,\cJ.
     \label{eq:HlowBry1D}
    \end{align}
    Combining Eq.~\eqref{eq:HupBry1D} and ~\eqref{eq:HlowBry1D} gives $\big(1-\cJ\big)\cJ-|\vect{\cH}|\ge0$ in 1-dimension.
    
    To prove $\big(1-\cJ\big)\cJ-|\vect{\cH}|\ge0$ for 3-dimensional $\vect{\cH}$, let's introduce a unit vector $\bn$ with a arbitrary direction.
    Then 
    \begin{equation}
    \vect{\cH} \cdot \bn = |\vect{\cH}| \cos\alpha,
    \end{equation}
    with $\alpha$ be the angle between $\vect{\cH}$ and $\bn$. 
    We know that $-|\vect{\cH}|\leq\vect{\cH} \cdot \bn\leq|\vect{\cH}|$ since $-1\leq\cos\alpha\leq1$.
    While given by definition,
    \begin{align}
    \vect{\cH} \cdot \bn & = \f{1}{4\pi}\int_{\bbS^{2}}f(\omegaNu)\ \bl(\omegaNu) d\omegaNu \cdot \bn \nonumber \\
     &= \f{1}{4\pi}\int_{\bbS^{2}}f(\omegaNu)\ \bl(\omegaNu) \cdot \bn d\omegaNu \nonumber \\
     &= \f{1}{4\pi}\int_{0}^{\pi}\int_{0}^{2\pi}f(\theta,\phi)\ \cos \theta \sin\theta d\theta d\phi
    \end{align}
    while $\theta$ is the angle between $\bl$ and $\bn$, and the coordinate are rotated so that the $\theta$ from $\omegaNu(\theta,\phi)$ is the same with the angle between $\bl$ and $\bn$.
    Note the above expression can also be rewritten by changing variables as
    \begin{align}
    \vect{\cH} \cdot \bn & = \f{1}{2}\int_{0}^{\pi} \f{1}{2\pi}\int_{0}^{2\pi}f(\theta,\phi)\ d\phi \cdot \cos \theta \sin\theta d\theta  \\
    & =  \f{1}{2}\int_{-1}^{1} \f{1}{2\pi}\int_{0}^{2\pi}f(\mu,\phi)\ d\phi \mu d\mu.
    \end{align}
    Besides,
    \begin{align}
    \cJ = \f{1}{4\pi}\int_{\bbS^{2}}f(\omegaNu)\,d\omegaNu = \f{1}{2}\int_{-1}^{1} \f{1}{2\pi}\int_{0}^{2\pi}f(\mu,\phi)\ d\phi d\mu.
    \end{align}
    If we define $f(\mu) =  \f{1}{2\pi}\int_{0}^{2\pi}f(\mu,\phi)\ d\phi$, then
    \begin{align}
    \cJ &= \f{1}{2}\int_{-1}^{1} f(\mu) d\mu, \\
    \vect{\cH} \cdot \bn  &= \f{1}{2}\int_{-1}^{1} f(\mu) \mu d\mu.
    \end{align}
    We know $0\leq\,f(\mu)\leq1$.
    Recall the proof we gave for 1-dimensional $\vect{\cH}$, we have
    \begin{equation}
     \big(1-\cJ\big)\cJ-\vect{\cH} \cdot \bn\ge0.
    \end{equation}
    Note this inequality is valid for all the possible $\bn$ which includes the special case with $\cos\alpha=1$ and
    \begin{equation}
    \big(1-\cJ\big)\cJ-|\vect{\cH}|\ge0.
    \end{equation} 
\end{proof}

We now define the realizable set $\cR$
\begin{equation}
  \cR:=\big\{\,\vect{\cM}=\big(\cJ,\vect{\cH}\big)^{T}~|~\cJ\in[0,1]~\text{and}~(1-\cJ)\cJ-|\vect{\cH}|\ge0\,\big\}
  \label{eq:realizableSet}
\end{equation}

\begin{lemma}
  The set $\cR$, defined in Eq.~\eqref{eq:realizableSet}, is convex.  
\end{lemma}
\begin{proof}
  A convex set $Q$ is a set has the following property: 
  \begin{align}
  \theta a + (1-\theta)b \in Q, \text{\space} \text{for } \forall a, b \in Q \text{ and } 0\leq\,\theta\,\leq1. 
  \end{align} 
  Let $\vect{\cM}_{a}=\big(\cJ_{a},\vect{\cH}_{a}\big)^{T}$ and $\vect{\cM}_{b}=\big(\cJ_{b},\vect{\cH}_{b}\big)^{T}$ to be two arbitrary elements of $\cR$.
  If $\cR$ is convex, we will have $\vect{\cM}_{ab} = \big(\cJ_{ab},\vect{\cH}_{ab}\big)^{T} = \theta \vect{\cM}_{a} + (1-\theta)\vect{\cM}_{b}$ in $\cR$ with $0\leq\theta\leq1$.
  Let's prove it is true.
  
  For the first component
  \begin{align}
  \cJ_{ab} = \theta \cJ_{a} + (1-\theta)\cJ_{b},
  \end{align}
  since $\theta$, $\cJ_{a}$, $1-\theta$ and $\cJ_{b}$ are all nonnegative, $\cJ_{ab}$ is also nonnegative ($\cJ_{ab}\geq0$).
  Meanwhile, since $\cJ_{a}\leq1$ and $\cJ_{b}\leq1$, 
  \begin{align}
  \cJ_{ab} \leq \theta \cdot 1 + (1-\theta) \cdot 1 = 1.
  \end{align}
  Therefore, $\cJ_{ab} \in [0,1]$.
  
  For the second component,
  \begin{align}
  \vect{\cH}_{ab} 
  & = \theta \vect{\cH}_{a} + (1-\theta)\vect{\cH}_{b} \nonumber\\
  & = \f{1}{4\pi}\int_{\bbS^{2}} \left[ \theta\,f_{a}(\omegaNu) + (1-\theta) f_{b}(\omegaNu) \right] \bl(\omegaNu) d\omegaNu. 
  \end{align}
  Since $\theta\,f_{a}(\omegaNu) + (1-\theta) f_{b}(\omegaNu) \in [0,1]$, we can define $\tilde{f} = \theta\,f_{a}(\omegaNu) + (1-\theta) f_{b}(\omegaNu)$.
  $\tilde{f} \ in [0,1]$ and 
  \begin{align}
  \cJ_{ab} = \f{1}{4\pi}\int_{\bbS^{2}} \tilde{f} d\omegaNu \\
  \vect{\cH}_{ab} = 
  \end{align}
\end{proof}

Include figure to illustrate the convex set.  

\begin{lemma}
  Let $\big\{\cJ_{a},\vect{\cH}_{a},\vect{\cK}_{a}\big\}$ be defined as in Eq.~\eqref{eq:angularMoments} with distribution function $f_{a}\in[0,1]$.  
  Similarly, $\big\{\cJ_{b},\vect{\cH}_{b},\vect{\cK}_{b}\big\}$ be moments of a distribution function $f_{b}\in[0,1]$.  
  Let $\Phi^{\pm}(\vect{\cM})=\f{1}{2}\big(\vect{\cM}\pm\widehat{\vect{e}}\cdot\vect{\cF}(\vect{\cM})\big)$, where $\widehat{\vect{e}}\in\bbR^{3}$ is an arbitrary unit vector.  
  Then
  \begin{equation}
    \Phi^{+}(\vect{\cM}_{a})+\Phi^{-}(\vect{\cM}_{b})\in\cR.
  \end{equation}
\end{lemma}
\begin{proof}
  Prove this.
\end{proof}

\subsection{Moment Closures}

\subsubsection{Maximum Entropy Closure}
Basic principles
Chernohorsky \& Bludman
Banach \& Larecki

\subsubsection{Kershaw Closure}
Basic principles
Banach \& Larecki