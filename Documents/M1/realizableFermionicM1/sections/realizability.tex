\section{Moment Realizability and Closures for Fermion}
\label{sec:realizability}

Our goal is to design a numerical method for solving the system of equations given by Eq.~\eqref{eq:momentEquations}, which approximates Eq.~\eqref{eq:boltzmann}.  
To this end, we want to obtain solutions that preserve features of the original problem. 

\subsection{Moment Realizability}

\begin{define}
  The moments $\vect{\cM}=\big(\cJ,\vect{\cH}\big)^{T}$ are realizable if they are consistent with a distribution function satisfying $0\le f \le 1$.
\end{define}
\begin{lemma}
  Let the distribution function $f$ be bounded by $0\le f \le1$.  
  Then $0\le\cJ\le1$ and $\big(1-\cJ\big)\cJ-|\vect{\cH}|\ge0$
\end{lemma}
\begin{proof}
  Since this lemma is independent from $\vect{z}$ and t, we can drop them from the expression for convenience.
  Let's proof $0\le\cJ\le1$.
  
  Since $f$ is positive, we have
    \begin{equation}
      \cJ=\f{1}{4\pi}\int_{\bbS^{2}}f(\omegaNu)\,d\omegaNu\ge0.
    \end{equation}
    
  Since $1-f$ is positive, we have
    \begin{equation}
      1-\cJ=\f{1}{4\pi}\int_{\bbS^{2}}(1-f(\omegaNu))\,d\omegaNu\ge0.
    \end{equation}
    Therefore, $0\le\cJ\le1$.

    For $\big(1-\cJ\big)\cJ-|\vect{\cH}|\ge0$, we can proof it's valid with 1-dimension $\vect{\cH}$ first.
    In 1-dimension, $\vect{\ell}(\omegaNu) = \omegaNu$ and
    \begin{equation}
    \vect{\cH}
      =\f{1}{4\pi}\int_{\bbS^{2}}f(\omegaNu) \omegaNu d\omegaNu = \f{1}{2}\int_{-1}^{1}f(\omegaNu) \omegaNu d\omegaNu.
    \end{equation}
    Following Banach \& Larecki \cite{banachLarecki_2017}, we can rewrite it to be
    \begin{align}
    \vect{\cH}
        &=(1-\cJ)\,\cJ \nonumber \\
        &-\Big\{\,
          \f{1}{2}\int_{-1}^{1-2\cJ}(1-2\cJ-\omegaNu)f(\omegaNu)\,d\omegaNu
          +\f{1}{2}\int_{1-2\cJ}^{1}(2\cJ-1+\omegaNu)[1-f(\omegaNu)]\,d\omegaNu
        \,\Big\}.
        \label{eq:momentumRewrite1}
    \end{align}
    Since $(1-2\cJ-\omegaNu)f(\omegaNu)$ is positive when $-1\leq\omegaNu\leq1-2\cJ$ and $(2\cJ-1+\omegaNu)[1-f(\omegaNu)]$ is positive when $1-2\cJ\leq\omegaNu\leq1$,
    \begin{align}
    \vect{\cH} \leq (1-\cJ)\,\cJ.
    \label{eq:HupBry1D}
    \end{align}
    Similarly, we can also rewrite $\vect{\cH}$ as
    \begin{align}
     \vect{\cH}
        &=-(1-\cJ)\,\cJ \nonumber \\
        &-\Big\{\,
          \f{1}{2}\int_{-1}^{2\cJ-1}(1-2\cJ+\omegaNu)[1-f(\omegaNu)]\,d\omegaNu
          +\f{1}{2}\int_{2\cJ-1}^{1}(2\cJ-1-\omegaNu)f(\omegaNu)\,d\omegaNu
        \,\Big\}.
        \label{eq:momentumRewrite2}
    \end{align}
    Since $(1-2\cJ+\omegaNu)[1-f(\omegaNu)]$ is negative when $-1\leq\omegaNu\leq2\cJ-1$ and $(2\cJ-1-\omegaNu)f(\omegaNu)$ is negative when $2\cJ-1\leq\omegaNu\leq1$,
    \begin{align}
    \vect{\cH} \geq -(1-\cJ)\,\cJ.
     \label{eq:HlowBry1D}
    \end{align}
    Combining Eq.~\eqref{eq:HupBry1D} and ~\eqref{eq:HlowBry1D} gives $\big(1-\cJ\big)\cJ-|\vect{\cH}|\ge0$ in 1-dimension.
\end{proof}

We now define the realizable set $\cR$
\begin{equation}
  \cR:=\big\{\,\vect{\cM}=\big(\cJ,\vect{\cH}\big)^{T}~|~\cJ\in[0,1]~\text{and}~(1-\cJ)\cJ-|\vect{\cH}|\ge0\,\big\}
  \label{eq:realizableSet}
\end{equation}

\begin{lemma}
  The set $\cR$, defined in Eq.~\eqref{eq:realizableSet}, is convex.  
\end{lemma}
\begin{proof}
  Prove this
\end{proof}

Include figure to illustrate the convex set.  

\begin{lemma}
  Let $\big\{\cJ_{a},\vect{\cH}_{a},\vect{\cK}_{a}\big\}$ be defined as in Eq.~\eqref{eq:angularMoments} with distribution function $f_{a}\in[0,1]$.  
  Similarly, $\big\{\cJ_{b},\vect{\cH}_{b},\vect{\cK}_{b}\big\}$ be moments of a distribution function $f_{b}\in[0,1]$.  
  Let $\Phi^{\pm}(\vect{\cM})=\f{1}{2}\big(\vect{\cM}\pm\widehat{\vect{e}}\cdot\vect{\cF}(\vect{\cM})\big)$, where $\widehat{\vect{e}}\in\bbR^{3}$ is an arbitrary unit vector.  
  Then
  \begin{equation}
    \Phi^{+}(\vect{\cM}_{a})+\Phi^{-}(\vect{\cM}_{b})\in\cR.
  \end{equation}
\end{lemma}
\begin{proof}
  Prove this.
\end{proof}

\subsection{Moment Closures}

\subsubsection{Maximum Entropy Closure}
Basic principles
Chernohorsky \& Bludman
Banach \& Larecki

\subsubsection{Kershaw Closure}
Basic principles
Banach \& Larecki