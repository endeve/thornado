\section{Moment Realizability and Closures for Fermion}
\label{sec:realizability}

Our goal is to design a numerical method for solving the system of equations given by Eq.~\eqref{eq:momentEquations}, which approximates Eq.~\eqref{eq:boltzmann}.  
To this end, we want to obtain solutions that preserve features of the original problem. 

Since we only consider the angular dependence of the distribution function in this section, we simplify the notation by suppressing the $\vect{z}$ and $t$ dependence, and write $f(\omegaNu,\vect{z},t)=f(\omegaNu)$.  

\subsection{Moment Realizability}

\begin{define}
  The moments $\vect{\cM}=\big(\cJ,\vect{\cH}\big)^{T}$ are realizable if they are consistent with a distribution function satisfying $0\le f(\omegaNu) \le 1$.
\end{define}

We begin by stating the bounds on the moments by restating results from Theorem~7.1 in \cite{banachLarecki_2017} in the following Lemma.  
\begin{lemma}
  Let the distribution function be bounded by $0\le f(\omegaNu) \le1$ for all $\omegaNu\in\bbS^{2}$.  
  Then $0\le\cJ\le1$ and $\big(1-\cJ\big)\,\cJ-|\vect{\cH}|\ge0$.  
\end{lemma}
\begin{proof}
  First, let $\mu=\cos\thetaNu$ and write
  \begin{equation}
    \cJ=\f{1}{2}\int_{\bbS^{1}}\mathfrak{f}(\mu)\,d\mu,
    \quad\text{where}\quad
    \mathfrak{f}(\mu)=\f{1}{2\pi}\int_{0}^{2\pi}f(\mu,\phiNu)\,d\phiNu.  
  \end{equation}  
  Since $f(\omegaNu)\in[0,1]$, we have $\mathfrak{f}(\mu)\in[0,1]$, which implies $\cJ\in[0,1]$.  
  Next we need to show that $|\vect{\cH}|\le(1-\cJ)\,\cJ$.  
  Let $\widehat{\vect{n}}\in\bbR^{3}$ (independent of $\omegaNu$) be an arbitrary unit vector, and write
  \begin{equation}
    \widehat{\vect{n}}\cdot\vect{\cH}
    =\int_{\bbS^{1}}\mathfrak{f}(\mu)\,\mu\,d\mu,
    \label{eq:NdotH}
  \end{equation}
  where we have aligned the polar axis of the spherical momentum space coordinate system with $\widehat{\vect{n}}$; i.e., $\widehat{\vect{n}}\cdot\vect{\ell}(\omega)=\mu$.  
  First, following \cite{banachLarecki_2017}, we can write the integral on the right-hand side of Eq.~\eqref{eq:NdotH} as
  \begin{equation*}
    (1-\cJ)\cJ - \f{1}{2}\int_{-1}^{1-2\cJ}(1-2\cJ-\mu)\,\mathfrak{f}(\mu)\,d\mu
    -\f{1}{2}\int_{1-2\cJ}^{1}(2\cJ-1+\mu)\,[1-\mathfrak{f}(\mu)]\,d\mu.  
  \end{equation*}
  Since $\mathfrak{f}\in[0,1]$, $(1-2\cJ-\mu)$ is nonnegative on the interval $\mu\in[-1,1-2\cJ]$, and $(2\cJ-1+\mu)$ is nonnegative on the interval $\mu\in[1-2\cJ,1]$, we have $\widehat{\vect{n}}\cdot\vect{\cH}\le(1-\cJ)\,\cJ$.  
  In a similar manner, we can also write the integral on the right-hand side of Eq.~\eqref{eq:NdotH} as
  \begin{equation*}
    -(1-\cJ)\cJ-\f{1}{2}\int_{-1}^{2\cJ-1}(1-2\cJ+\mu)\,[1-\mathfrak{f}(\mu)]\,d\mu
    -\f{1}{2}\int_{2\cJ-1}^{1}(2\cJ-1-\mu)\,\mathfrak{f}(\mu)\,d\mu.  
  \end{equation*}
  Then, since $\mathfrak{f}\in[0,1]$, $(1-2\cJ+\mu)$ is nonpositive on the interval $\mu\in[-1,2\cJ-1]$, and $(2\cJ-1-\mu)$ is nonpositive on the interval $\mu\in[2\cJ-1,1]$, we have $\widehat{\vect{n}}\cdot\vect{\cH}\ge-(1-\cJ)\,\cJ$.  
  We therefore have $-(1-\cJ)\,\cJ\le\widehat{\vect{n}}\cdot\vect{\cH}\le(1-\cJ)\,\cJ$.  
  Since this holds for any unit vector $\widehat{\vect{n}}\in\bbR^{3}$, we must have $|\vect{\cH}|\le(1-\cJ)\,\cJ$.  
\end{proof}

We now define the realizable set $\cR$
\begin{equation}
  \cR:=\big\{\,\vect{\cM}=\big(\cJ,\vect{\cH}\big)^{T}~|~\cJ\in[0,1]~\text{and}~\gamma(\vect{\cM})\equiv(1-\cJ)\cJ-|\vect{\cH}|\ge0\,\big\}
  \label{eq:realizableSet}
\end{equation}

\begin{lemma}
  The set $\cR$, defined in Eq.~\eqref{eq:realizableSet}, is convex.  
\end{lemma}
\begin{proof}
  A convex set $Q$ is a set has the following property: 
  \begin{align}
  \theta a + (1-\theta)b \in Q, \text{\space} \text{for } \forall a, b \in Q \text{ and } 0\leq\,\theta\,\leq1. 
  \end{align} 
  Let $\vect{\cM}_{a}=\big(\cJ_{a},\vect{\cH}_{a}\big)^{T}$ and $\vect{\cM}_{b}=\big(\cJ_{b},\vect{\cH}_{b}\big)^{T}$ to be two arbitrary elements of $\cR$.
  If $\cR$ is convex, we will have $\vect{\cM}_{ab} = \big(\cJ_{ab},\vect{\cH}_{ab}\big)^{T} = \theta \vect{\cM}_{a} + (1-\theta)\vect{\cM}_{b}$ in $\cR$ with $0\leq\theta\leq1$.
  Let's prove it is true.
  
  For the first component
  \begin{align}
  \cJ_{ab} = \theta \cJ_{a} + (1-\theta)\cJ_{b},
  \end{align}
  since $\theta$, $\cJ_{a}$, $1-\theta$ and $\cJ_{b}$ are all nonnegative, $\cJ_{ab}$ is also nonnegative ($\cJ_{ab}\geq0$).
  Meanwhile, since $\cJ_{a}\leq1$ and $\cJ_{b}\leq1$, 
  \begin{align}
  \cJ_{ab} \leq \theta \cdot 1 + (1-\theta) \cdot 1 = 1.
  \end{align}
  Therefore, $\cJ_{ab} \in [0,1]$.
  
  For the second component,
  \begin{align}
  \vect{\cH}_{ab} 
  & = \theta \vect{\cH}_{a} + (1-\theta)\vect{\cH}_{b} \nonumber\\
  & = \f{1}{4\pi}\int_{\bbS^{2}} \left[ \theta\,f_{a}(\omegaNu) + (1-\theta) f_{b}(\omegaNu) \right] \bl(\omegaNu) d\omegaNu. 
  \end{align}
  Since $\theta\,f_{a}(\omegaNu) + (1-\theta) f_{b}(\omegaNu) \in [0,1]$, we can define $\tilde{f} = \theta\,f_{a}(\omegaNu) + (1-\theta) f_{b}(\omegaNu)$.
  $\tilde{f} \ in [0,1]$ and 
  \begin{align}
  \cJ_{ab} = \f{1}{4\pi}\int_{\bbS^{2}} \tilde{f} d\omegaNu \\
  \vect{\cH}_{ab} = 
  \end{align}
\end{proof}

Include figure to illustrate the convex set.  

\begin{lemma}
  Let $\big\{\cJ_{a},\vect{\cH}_{a},\vect{\cK}_{a}\big\}$ be defined as in Eq.~\eqref{eq:angularMoments} with distribution function $f_{a}\in[0,1]$.  
  Similarly, $\big\{\cJ_{b},\vect{\cH}_{b},\vect{\cK}_{b}\big\}$ be moments of a distribution function $f_{b}\in[0,1]$.  
  Let $\Phi^{\pm}(\vect{\cM})=\f{1}{2}\big(\vect{\cM}\pm\widehat{\vect{e}}\cdot\vect{\cF}(\vect{\cM})\big)$, where $\widehat{\vect{e}}\in\bbR^{3}$ is an arbitrary unit vector.  
  Then
  \begin{equation}
    \Phi^{+}(\vect{\cM}_{a})+\Phi^{-}(\vect{\cM}_{b})\in\cR.
  \end{equation}
\end{lemma}
\begin{proof}
  Prove this.
\end{proof}

\subsection{Moment Closures}

Flux factor $h=|\vect{\cH}|/\cJ$
\begin{equation}
  \chi(\cJ,h)=\f{1}{3}+\f{2\,(1-\cJ)\,(1-2\cJ)}{3}\,\Theta\Big(\f{h}{1-\cJ}\Big)
\end{equation}

\subsubsection{Maximum Entropy Closure}
Basic principles
Chernohorsky \& Bludman \cite{cernohorskyBludman_1994}
\begin{equation}
  \Theta_{\mbox{\tiny ME}}^{\mbox{\tiny CB}}(x)
  =x^{2}\,\big(\,3-x+3\,x^{2}\,\big)
\end{equation}
Banach \& Larecki
\begin{equation}
  \Theta_{\mbox{\tiny ME}}^{\mbox{\tiny BL}}(x)
  =\f{1}{8}\,\big(\,9\,x^{2}-5+\sqrt{33\,x^{4}-42\,x^{2}+25}\,\big)
\end{equation}

\subsubsection{Kershaw Closure}
Basic principles
Banach \& Larecki \cite{banachLarecki_2017}
\begin{equation}
  \Theta_{\mbox{\tiny K}}^{\mbox{\tiny BL}}(x)=x^{2}
\end{equation}

\subsubsection{Low Occupancy Limit}