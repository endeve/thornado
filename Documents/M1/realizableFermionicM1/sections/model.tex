\section{Mathematical Model}

In this section we give a brief summary of the mathematical model.  

\subsection{Boltzmann Equation}

In this paper, we consider approximate solutions to the Boltzmann equation for radiation transport in slab geometry
\begin{equation}
  \pd{f}{t}+\vect{\ell}\cdot\nabla f
  =\sigma_{\Ab}\,\big(\,f_{0}-f\,\big)
  +\sigma_{\Scatt}\,\big(\,\f{1}{4\pi}\int_{\bbS^{2}}f\,d\omegaNu-f\,\big),
  \label{eq:boltzmann}
\end{equation}
where the distribution function $f(\omegaNu,\epsilonNu,\vect{x},t):\bbS^{2}\times\bbR^{+}\times\bbR^{3}\times\bbR^{+}\to\bbR^{+}$ gives the number of particles propagating in direction $\omegaNu\in\bbS^{2}:=\{\,\omegaNu=(\thetaNu,\phiNu)~|~\thetaNu\in[0,\pi],\phiNu\in[0,2\pi)\,\}$, with energy $\epsilonNu\in\bbR^{+}$, at position $\vect{x}\in\bbR^{3}$ and time $t\in\bbR^{+}$.  
Thus, we use spherical momentum space coordinates $(\omegaNu,\epsilonNu)$.  
We also define the energy-position coordinates $\vect{z}:=\{\epsilonNu,\vect{x}\}\in\bbR^{+}\times\bbR^{3}$.  
The unit vector $\vect{\ell}(\omegaNu)\in\bbR^{3}$ (independent of $\vect{z}$) is parallel to the particle three-momentum.  
On the right-hand side of Eq.~\eqref{eq:boltzmann}, $\sigma_{\Ab}(\vect{z})\ge0$ and $\sigma_{\Scatt}(\vect{z})\ge0$ are the absorption and scattering opacities due to interactions with a background, respectively, and $f_{0}(\vect{z})$ is the equilibrium distribution distribution function.  
Here, we consider transport of Fermions (e.g., neutrinos), so that the equilibrium distribution function takes the form
\begin{equation}
  f_{0}(\vect{z})=\f{1}{e^{(\epsilonNu-\mu(\vect{x}))/T(\vect{x})}+1},  
\end{equation}
where temperature $T$ and the chemical potential $\mu$ are properties of the background.  

\subsection{Angular Moment Equations}

The Boltzmann equation is often too expensive to solve directly.  
Instead, equations for angular moments of the distribution function can be solved for.  
To this end, we define the angular moments of the distribution function
\begin{equation}
  \big\{\,\cJ,\vect{\cH},\vect{\cK}\,\big\}(\vect{z},t)
  =\f{1}{4\pi}\int_{\bbS^{2}}f(\omegaNu,\vect{z},t)\,\{\,1,\vect{\ell},\vect{\ell}\otimes\vect{\ell}\,\}\,d\omegaNu,
  \label{eq:angularMoments}
\end{equation}
where $\cJ$ (zeroth moment) is the density, $\vect{\cH}$ (first moment) the flux, and $\vect{\cK}$ (second moment) is the stress tensor.  
Note that the moments defined in Eq.~\ref{eq:angularMoments} are \emph{spectral moments} (depending on the energy $\epsilonNu$).  
The \emph{grey moments} are obtained by integration over energy
\begin{equation}
  \big\{\,J,\vect{H},\vect{K}\,\big\}(\vect{x},t)
  =\int_{\bbR^{+}}\big\{\,\cJ,\vect{\cH},\vect{\cK}\,\big\}(\epsilonNu,\vect{x},t)\,\epsilonNu^{2}d\epsilonNu.  
\end{equation}

Taking the zeroth and first moments of Eq.~\eqref{eq:boltzmann} gives the two-moment model, comprising of a system of conservation laws with sources
\begin{equation}
  \pd{\vect{\cM}}{t}+\nabla\cdot\vect{\cF}(\vect{\cM})=\vect{\cC}(\vect{\cM}),
  \label{eq:momentEquations}
\end{equation}
where $\vect{\cM}=(\cJ,\vect{\cH})^{T}$, the components of the flux vector $\vect{\cF}$ are $\vect{\cF}^{i}=\vect{e}_{i}\cdot\vect{\cF}=(\vect{e}_{i}\cdot\vect{\cH},\vect{e}_{i}\cdot\vect{\cK})^{T}$, where $\vect{e}_{i}$ is the unit vector in the $i$th coordinate direction, and $\vect{\cC}=\vect{\cQ}-\vect{\cD}\,\vect{\cM}$.  
Here we have also defined $\vect{\cQ}=(\sigma_{\Ab}\cJ_{0},\vect{0})^{T}$, where $\cJ_{0}=f_{0}$, and $\vect{\cD}=\mbox{diag}(\sigma_{\Ab},\vect{I}\sigma_{\Tot})$, where $\sigma_{\Tot}=\sigma_{\Ab}+\sigma_{\Scatt}$, and $\vect{I}$ is the identity matrix.  
In order to close the system of equations, the stress tensor $\vect{\cK}$ must be related to the lower moments by a closure procedure.  
To this end, Levermove \cite{levermore_1984} defines $\vect{k}=\vect{\cK}/\cJ$, assumes that the radiation field is symmetric about a preferred direction $\vect{h}=\vect{\cH}/|\vect{\cH}|$, and writes
\begin{equation}
  \vect{k}=\f{1}{2}\big[\,\big(1-\chi\big)\,\vect{I}+\big(3\,\chi-1\big)\,\vect{h}\otimes\vect{h}\,\big],
\end{equation}
where $\chi=\chi(\cJ,|\vect{\cH}|)$ is the Eddington factor.  
We will return to the issue of determining the Eddington factor in Section~\ref{sec:realizability}.  

This concludes the initial description of our model.  