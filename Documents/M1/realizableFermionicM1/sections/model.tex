\section{Mathematical Model}

In this section we give a summary of the mathematical model we employ.  

\subsection{Boltzmann Equation}

In this paper we consider approximate solutions to the Boltzmann equation for radiation transport in slab geometry, which after scaling to dimensionless units can be written as
\begin{equation}
  \pd{f}{t}+\vect{\ell}\cdot\nabla f
  =\f{1}{\tau}\,\cC(f),
  \label{eq:boltzmann}
\end{equation}
where the distribution function $f(\omegaNu,\epsilonNu,\vect{x},t):\bbS^{2}\times\bbR^{+}\times\bbR^{3}\times\bbR^{+}\to\bbR^{+}$ gives the number of particles propagating in direction $\omegaNu\in\bbS^{2}:=\{\,\omegaNu=(\thetaNu,\phiNu)~|~\thetaNu\in[0,\pi],\phiNu\in[0,2\pi)\,\}$, with energy $\epsilonNu\in\bbR^{+}$, at position $\vect{x}\in\bbR^{3}$ and time $t\in\bbR^{+}$; i.e., we use spherical momentum space coordinates $(\omegaNu,\epsilonNu)$.  
We also define the energy-position coordinates $\vect{z}:=\{\epsilonNu,\vect{x}\}\in\bbR^{+}\times\bbR^{3}$.  
The unit vector $\vect{\ell}(\omegaNu)\in\bbR^{3}$ (independent of $\vect{z}$) is parallel to the particle three-momentum.  
On the right-hand side of Eq.~\eqref{eq:boltzmann}, $\tau$ is the ratio of the particle mean-free path (due to interactions with a background) to some characteristic length scale of the problem; in opaque regions, $\tau\ll1$, while for free streaming radiation, $\tau\gg1$.  
The collision term is given by
\begin{equation}
  \cC(f)=\xi\,\big(\,f_{0}-f\,\big)
  +(1-\xi)\,\big(\,\f{1}{4\pi}\int_{\bbS^{2}}f\,d\omegaNu-f\,\big),
  \label{eq:collisionTerm}
\end{equation}
where $\xi\in[0,1]$ is the ratio of the absorption opacity $\sigma_{\Ab}\,(\ge0)$ to the total opacity $\sigma_{\Tot}=\sigma_{\Ab}+\sigma_{\Scatt}$, and $\sigma_{\Scatt}\,(\ge0)$ is the scattering opacity.  
Then, $\xi=1$ models pure emission and absorption, while $\xi=0$ models pure scattering.  
In general, $\sigma_{\Ab}(\vect{z})$ and $\sigma_{\Scatt}$ (and $\tau$ and $\xi$) depend on $\vect{z}$.  
The equilibrium distribution function is denoted $f_{0}(\vect{z})$.  
Here, we consider transport of Fermions (e.g., neutrinos), so that the equilibrium distribution function takes the form
\begin{equation}
  f_{0}(\vect{z})=\f{1}{e^{(\epsilonNu-\mu(\vect{x}))/T(\vect{x})}+1},  
  \label{eq:fermiDirac}
\end{equation}
where the temperature $T$ and the chemical potential $\mu$ are properties of the background.  

\subsection{Angular Moment Equations: Two-Moment Model}

The Boltzmann equation is often too expensive to solve directly.  
Instead, equations for angular moments of the distribution function can be solved for.  
To this end, we define the angular moments of the distribution function
\begin{equation}
  \big\{\,\cJ,\vect{\cH},\vect{\cK}\,\big\}(\vect{z},t)
  =\f{1}{4\pi}\int_{\bbS^{2}}f(\omegaNu,\vect{z},t)\,\{\,1,\vect{\ell},\vect{\ell}\otimes\vect{\ell}\,\}\,d\omegaNu,
  \label{eq:angularMoments}
\end{equation}
where we will refer to $\cJ$ (zeroth moment) as the particle density, $\vect{\cH}$ (first moment) as the particle flux, and $\vect{\cK}$ (second moment) as the stress tensor.  
Note that the moments defined in Eq.~\eqref{eq:angularMoments} are \emph{spectral moments} (depending on energy, position, and time).  
The \emph{grey moments} (depending only on position and time) are obtained by integration over energy as
\begin{equation}
  \big\{\,J,\vect{H},\vect{K}\,\big\}(\vect{x},t)
  =\int_{\bbR^{+}}\big\{\,\cJ,\vect{\cH},\vect{\cK}\,\big\}(\epsilonNu,\vect{x},t)\,\epsilonNu^{2}d\epsilonNu.  
\end{equation}

Taking the zeroth and first moments of Eq.~\eqref{eq:boltzmann} gives the two-moment model, comprising of a system of conservation laws with sources
\begin{equation}
  \pd{\vect{\cM}}{t}+\nabla\cdot\vect{\cF}(\vect{\cM})=\f{1}{\tau}\,\vect{\cC}(\vect{\cM}),
  \label{eq:momentEquations}
\end{equation}
where $\vect{\cM}=(\cJ,\vect{\cH})^{T}$, the components of the flux vector $\vect{\cF}$ are $\vect{\cF}^{i}=\vect{e}_{i}\cdot\vect{\cF}=(\vect{e}_{i}\cdot\vect{\cH},\vect{e}_{i}\cdot\vect{\cK})^{T}$, where $\vect{e}_{i}$ is the unit vector in the $i$th coordinate direction.  
On the right-hand side of Eq.~\eqref{eq:momentEquations}, the collision term becomes
\begin{equation}
  \vect{\cC}(\vect{\cM})=\vect{\eta}-\vect{\cD}\,\vect{\cM}, 
  \label{eq:collisionTermMoments}
\end{equation}
where we have defined $\vect{\eta}=(\xi\,f_{0},\vect{0})^{T}$ and $\vect{\cD}=\mbox{diag}(\xi,\vect{I})$, where $\vect{I}$ is the identity matrix.  
In order to close the system of equations, the components of the stress tensor $\vect{\cK}$ must be related to the lower moments through a closure procedure.  
To this end, Levermore \cite{levermore_1984} defines the Eddington tensor $\vect{k}=\vect{\cK}/\cJ$, assumes that the radiation field is symmetric about a preferred direction $\widehat{\vect{h}}=\vect{\cH}/|\vect{\cH}|$, and writes
\begin{equation}
  \vect{k}=\f{1}{2}\big[\,\big(1-\chi\big)\,\vect{I}+\big(3\,\chi-1\big)\,\widehat{\vect{h}}\otimes\widehat{\vect{h}}\,\big],
  \label{eq:eddingtonTensor}
\end{equation}
where $\chi=\chi(\cJ,|\vect{\cH}|)$ is the Eddington factor.  
The two-moment model is then closed once the Eddington factor is determined from $\cJ$ and $\vect{\cH}$.  
We will return to the issue of determining the Eddington factor in Section~\ref{sec:algebraicClosure}.  

This concludes the initial description of our model.  