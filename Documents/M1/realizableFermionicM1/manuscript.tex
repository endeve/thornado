%\documentclass[1p]{elsarticle}
\documentclass[review]{elsarticle}

\usepackage{lineno,hyperref}
\usepackage{amsfonts}
\usepackage{amsmath}
\usepackage{amssymb}
\usepackage{amsthm}
\usepackage{mathrsfs}
\usepackage{subcaption}
\usepackage{float}

\modulolinenumbers[5]
\hypersetup{
  bookmarksnumbered = true,
  bookmarksopen=false,
  pdfborder=0 0 0,         % make all links invisible, so the pdf looks good when printed
  pdffitwindow=true,      % window fit to page when opened
  pdfnewwindow=true, % links in new window
  colorlinks=true,           % false: boxed links; true: colored links
  linkcolor=blue,            % color of internal links
  citecolor=magenta,    % color of links to bibliography
  filecolor=magenta,     % color of file links
  urlcolor=cyan              % color of external links
}


\journal{Journal of Computational Physics}

%%%%%%%%%%%%%%%%%%%%%%%
%% Elsevier bibliography styles
%%%%%%%%%%%%%%%%%%%%%%%
%% To change the style, put a % in front of the second line of the current style and
%% remove the % from the second line of the style you would like to use.
%%%%%%%%%%%%%%%%%%%%%%%

%% Numbered
%\bibliographystyle{model1-num-names}

%% Numbered without titles
%\bibliographystyle{model1a-num-names}

%% Harvard
%\bibliographystyle{model2-names.bst}\biboptions{authoryear}

%% Vancouver numbered
%\usepackage{numcompress}\bibliographystyle{model3-num-names}

%% Vancouver name/year
%\usepackage{numcompress}\bibliographystyle{model4-names}\biboptions{authoryear}

%% APA style
%\bibliographystyle{model5-names}\biboptions{authoryear}

%% AMA style
%\usepackage{numcompress}\bibliographystyle{model6-num-names}

%% `Elsevier LaTeX' style
\bibliographystyle{elsarticle-num}
%%%%%%%%%%%%%%%%%%%%%%%

\newtheorem{define}{Definition}
\newtheorem{lemma}{Lemma}
\newtheorem{prop}{Proposition}
\newtheorem{rem}{Remark}
\newtheorem{theorem}{Theorem}

%\newcommand{\ee}[1]{{\color{red} EE:~#1}}
%\newcommand{\rc}[1]{{\color{blue} RC:~#1}}

\newcommand{\modified}[1]{{\color{red}#1}}

\usepackage{./definitions}

\begin{document}

\begin{frontmatter}

\title{Realizability-Preserving DG-IMEX Method for the Two-Moment Model of Fermion Transport \tnoteref{support}\tnoteref{copyright}}
\tnotetext[support]{
This research is sponsored, in part, by the Laboratory Directed Research and Development Program of Oak Ridge National Laboratory (ORNL), managed by UT-Battelle, LLC for the U. S. Department of Energy under Contract No. De-AC05-00OR22725.  
This research was supported by the Exascale Computing Project (17-SC-20-SC), a collaborative effort of the U.S. Department of Energy Office of Science and the National Nuclear Security Administration.  
This material is based, in part, upon work supported by the U.S. Department of Energy, Office of Science, Office of Advanced Scientific Computing Research.  
Eirik Endeve was supported in part by NSF under Grant No. 1535130.}
\tnotetext[copyright]{
This manuscript has been authored by UT-Battelle, LLC under Contract No. DE-AC05-00OR22725 with the U.S. Department of Energy. The United States Government retains and the publisher, by accepting the article for publication, acknowledges that the United States Government retains a non-exclusive, paid-up, irrevocable, world-wide license to publish or reproduce the published form of this manuscript, or allow others to do so, for United States Government purposes. The Department of Energy will provide public access to these results of federally sponsored research in accordance with the DOE Public Access Plan(http://energy.gov/downloads/doe-public-access-plan).}

%% Group authors per affiliation:
\author[utk-phys]{Ran Chu}
\ead{rchu@vols.utk.edu}

\author[ornl,utk-phys,jics]{Eirik Endeve\corref{cor}}
\ead{endevee@ornl.gov}

\author[ornl,utk-math]{Cory D. Hauck}
\ead{hauckc@ornl.gov}

\author[utk-phys,jics]{Anthony Mezzacappa}
\ead{mezz@utk.edu}

\cortext[cor]{Corresponding author. Tel.:+1 865 576 6349; fax:+1 865 241 0381}

\address[ornl]{Computational and Applied Mathematics Group, Oak Ridge National Laboratory, Oak Ridge, TN 37831 USA }

\address[utk-phys]{Department of Physics and Astronomy, University of Tennessee Knoxville, TN 37996-1200}

\address[jics]{Joint Institute for Computational Sciences, Oak Ridge National Laboratory, Oak Ridge, TN 37831-6354}

\address[utk-math]{Department of Mathematics, University of Tennessee Knoxville, TN 37996-1320}

\begin{abstract}
Building on the framework of Zhang \& Shu \cite{zhangShu_2010a,zhangShu_2010b}, we develop a realizability-preserving method to simulate the transport of particles (fermions) through a background material using a two-moment model that evolves the angular moments of a phase space distribution function $f$.  
The two-moment model is closed using algebraic moment closures; e.g., as proposed by Cernohorsky \& Bludman \cite{cernohorskyBludman_1994} and Banach \& Larecki \cite{banachLarecki_2017a}.  
Variations of this model have recently been used to simulate neutrino transport in nuclear astrophysics applications, including core-collapse supernovae and compact binary mergers.  
We employ the discontinuous Galerkin (DG) method for spatial discretization (in part to capture the asymptotic diffusion limit of the model) combined with implicit-explicit (IMEX) time integration to stably bypass short timescales induced by frequent interactions between particles and the background.  
Appropriate care is taken to ensure the method preserves strict algebraic bounds on the evolved moments (particle density and flux) as dictated by Pauli's exclusion principle, which demands a bounded distribution function (i.e., $f\in[0,1]$).  
This realizability-preserving scheme combines a suitable CFL condition, a realizability-enforcing limiter, a closure procedure based on Fermi-Dirac statistics, and an IMEX scheme whose stages can be written as a convex combination of forward Euler steps combined with a backward Euler step.  
The IMEX scheme is formally only first-order accurate, but works well in the diffusion limit, and --- without interactions with the background --- reduces to the optimal second-order strong stability-preserving explicit Runge-Kutta scheme of Shu \& Osher \cite{shuOsher_1988}.  
Numerical results demonstrate the realizability-preserving properties of the scheme.  
We also demonstrate that the use of algebraic moment closures not based on Fermi-Dirac statistics can lead to unphysical moments in the context of fermion transport.  
\end{abstract}

\begin{keyword}
Boltzmann equation, 
Radiation transport, 
Hyperbolic conservation laws, 
Discontinuous Galerkin, 
Implicit-Explicit, 
Moment Realizability
\end{keyword}

\end{frontmatter}

\tableofcontents

\linenumbers

\section{Introduction}

Core-collapse supernovae (CCSNe) are the explosions of massive stars that end their lives.
They are the dominant source of heavy elements and play important roles in many astrophysical phenomena, such as neutron star and black hole formation.  
Furthermore, these explosions occur at energies and densities relevant to address fundamental questions in nuclear, particle, and gravitational physics. 
A solid theoretical framework for the CCSN explosion mechanism may help answer important questions in fundamental physics.\cite{janka_etal_2007}

One essential part of the explosion mechanism is neutrino transport.
The neutrino energy deposition drives the CCSN explosion.\cite{mezzacappaMesser_1999}
Ideally, neutrino transport would be modeled by the Boltzmann transport equation, which is an integro-partial-differential equation evolving a phase-space distribution function $f$.\cite{Bruenn_1985}
Simulating the neutrino transport implies finding a solution of the Boltzmann equation for a specific domain and period with acceptable accuracy.

However, solving the Boltzmann transport equation full-dimensionally is expensive.
To balance physical fidelity and computational expediency, an approximate method called the two-moment method has been adopted.\cite{mezzacappaMesser_1999}
Using the two-moment method, the evolved variables are the zeroth and first angular moments of the distribution function $f$ -- the spectral particle density $\cJ$ and flux $\bcH$, respectively.
However, the transport term in the Boltzmann transport equation has the dot product of the velocity vector and the gradient of the phase-space distribution function as its form.
Integrating this term with an unit direction vector as the weight function couples the moment equations: it introduces the second angular moment $\bcK$ to the first moment $\bcH$'s equation.
Knowledge of $\bcK$ is needed to close the two-moment equation system.
Therefore, an algebraic closure that gives a predicted $\bcK$ given $\cJ$ and $\bcH$ is needed.
The better the closure predicts $\bcK$, the more accurate the two-moment method. 
Two-moment method has been widely applied with different algebraic closures, such as the Minerbo\cite{minerbo_1978} closure(e.g.~{O'Connor} and {Couch}\cite{oConnorCouch_2018}, Pan and et al\cite{pan_etal_2018}, Glas et al\cite{glas_etal_2018}, and Just et al\cite{just_etal_2018} ) and the Levermore\cite{levermore_1984} closure(e.g.~Vartanyan et al\cite{vartanyan_etal_2018}, Cabezon et al\cite{cabezon_etal_2018}, and Kuroda et al\cite{kuroda_etal_2016}). 

Applying the two-moment method does simplify the problem, but doesn't guarantee an affordable solution.
To be precise, how to discrete the continuous equation system given by the two-moment method and solve the discretized system efficiently is remain a question.
In fact, the time scales of neutrino interactions with the background ($\sim\mathcal{O}(10^{-13})$~second) is short compared to the duration of the CCSN explosion ($\sim\mathcal{O}(1)$~second).  
This means that a lot of time step would be needed for solving the system fully explicitly. 
On the other hand, solving the moment equations fully implicitly requires inverting globally many band-structured matrices whose sizes depend on the spatial discretization.
Such a global inversion is both expensive and unfriendly to parallelization.
To circumvent these challenges, implicit-explicit (IMEX) methods are taken into consideration.
By treating the transport terms in the two-moment equations explicitly and the collision terms implicitly, IMEX methods are subject only to a time step governed by the explicit transport terms, and the matrices to be inverted are block diagonal.
Therefore, IMEX methods require fewer time steps comparing with fully explicit method, and the calculation for each step is easily parallelizable.  
%For a non-relativistic system where the propagating speed of gravitional wave and fluid is much slower than the light speed, fully implicit method is a better choice for neutrino transport in CCSN simulation.\cite{cernohorskyVanWeert_1992}(read ?)
For a relativistic circumstance that we have, where the gravity, fluid, and neutrino have a relativistic propagating speed, IMEX method is an efficient method.

To model neutrino transport using two-moment method, two things need to be chosen carefully: an algebraic closure based on Fermi-Dirac statistic for closing the two-moment equations and a convex-invariant, diffusion-accurate IMEX scheme to ensure a physical result.
Since the neutrino distribution function is bounded ($f\in[0,1]$) by the Pauli exclusion principle, its moments as weight integrals of a bounded function over domain $\omega\in\bbS^{2}$ are also bounded.
We call the moments satisfying the constraints due to Fermi-Dirac statistics \textit{realizable moments}.
The algebraic closure should give a realizable $\bcK$, and the well-posedness of the closure requires realizable $\cJ$ and $\bcH$.
It explains why an algebraic closure based on Fermi-Dirac statistics is needed.
Realizability of $\cJ$ and $\bcH$ after each time step requires a convex-invariant IMEX scheme.
Since the realizable moments form a convex set, it is possible to construct a realizability-preserving method with convex-invariant IMEX scheme for two-moment neutrino transport.
What's more, the physics of neutrino transport in CCSN requires the IMEX scheme to be diffusion-accurate.

The study of moment realizability and realizability-preserving method with diffusion-accurate IMEX schemes motivate this work.
Gottlieb et al\cite{gottlieb_etal_2001} showed standard IMEX scheme with strong-stability-preserving can't have an order higher than first without a restrict time step requirement.
To have second-order (or higher-order) accuracy, some correction steps are needed.
Unfortunately, the correction steps can deteriorate the accuracy of the IMEX scheme in the diffusion limit or restrict the time step.
To keep things simple, we focus on IMEX schemes without correction steps and require them to be high-order (second or higher order) in the streaming limit and diffusion-accurate.
We call these IMEX schemes \textit{PD-ARS}.

\texttt{thornado} is our toolkit for high-order neutrino-radiation hydrodynamics based on high-order Runge-Kutta Discontinuous Galerkin (RKDG) methods.
It is developed at the University of Tennessee, Knoxville and Oak Ridge National Laboratory.
It currently solves the Euler equations for fluid dynamics and the two-moment approximation of the radiative transfer equation, both in the non-relativistic limits.\cite{endeve_etal_2018}

This paper is organized as follows: Section~\ref{se:Two-MomentModel} discusses the mathematical model, algebraic closures, and the constraints on the moments and algebraic closures imposed by Fermi-Dirac statistics;
Section~\ref{se:SpacialDiscretization} gives a first-order finite-volume spatial discretization and shows how the spatial discretization preserves constraints in an IMEX step;
Section~\ref{se:TimeIntegration} discusses how to use convex combination to construct PD-ARS scheme and two PD-ARS schemes, one with second-order accuracy in the streaming limit and the other with third-order accuracy in the same limit;
Section~\ref{se:NumericalTests} presents the results of the numerical tests, which demonstrate the properties of the PD-ARS schemes; Section~\ref{se:Conclusion} summarizes the achievements of this paper and discusses future works.

%\rc{Reference check}
\section{Mathematical Model}
\label{sec:model}

In this section we give a summary of the mathematical model.  

\subsection{Boltzmann Equation}

We consider approximate solutions to the Boltzmann equation for the transport of massless particles through a static material in Cartesian geometry, which, after scaling to dimensionless units, can be written as
\begin{equation}
  \pd{f}{t}+\vect{\ell}\cdot\nabla f
  =\f{1}{\tau}\,\cC(f),
  \label{eq:boltzmann}
\end{equation}
where the distribution function $f\colon(\omega,\varepsilon,\vect{x},t)\in\bbS^{2}\times\bbR^{+}\times\bbR^{3}\times\bbR^{+}\to\bbR^{+}$ gives the number of particles propagating in the direction $\omega\in\bbS^{2}:=\{\,\omega=(\thetaNu,\phiNu)~|~\thetaNu\in[0,\pi],\phiNu\in[0,2\pi)\,\}$, with energy $\varepsilon\in\bbR^{+}$, at position $\vect{x}\in\bbR^{3}$ and time $t\in\bbR^{+}$.  
Here we use spherical momentum space coordinates $(\varepsilon,\omega)$, and the unit vector $\vect{\ell}(\omega)\in\bbR^{3}$ (independent of $\varepsilon$ and $\vect{x}$) is parallel to the particle three-momentum $\vect{p}=\varepsilon\,\vect{\ell}$.  
We also define the energy-position coordinates $\vect{z}:=\{\varepsilon,\vect{x}\}\in\bbR^{+}\times\bbR^{3}$.  
On the right-hand side of Eq.~\eqref{eq:boltzmann}, $\tau$ is the ratio of the particle mean-free path (due to interactions with a background) to some characteristic length scale of the problem.  
In opaque regions, $\tau\ll1$, while for free streaming particles, $\tau\gg1$.  
The collision operator, which models emission, absorption, and isotropic and elastic scattering, is given by
\begin{equation}
  \cC(f)=\xi\,\big(\,f_{0}-f\,\big)
  +(1-\xi)\,\big(\,\f{1}{4\pi}\int_{\bbS^{2}}f\,d\omega-f\,\big),
  \label{eq:collisionTerm}
\end{equation}
where $\xi=\sigma_{\Ab}/\sigma_{\Tot}\in[0,1]$ is the ratio of the absorption opacity $\sigma_{\Ab}\,(\ge0)$ to the total opacity $\sigma_{\Tot}=\sigma_{\Ab}+\sigma_{\Scatt}$, and $\sigma_{\Scatt}\,(\ge0)$ is the scattering opacity.  
In particular, $\xi=1$ models pure emission and absorption, while $\xi=0$ models pure scattering.  
In general, $\sigma_{\Ab}$ and $\sigma_{\Scatt}$ (and $\tau$ and $\xi$) depend on $\vect{z}$.  
The equilibrium distribution function is denoted by $f_{0}(\vect{z})$.  
Here, we consider transport of Fermions (e.g., neutrinos), so the equilibrium distribution function takes the form
\begin{equation}
  f_{0}(\vect{z})=\f{1}{e^{(\varepsilon-\mu(\vect{x}))/T(\vect{x})}+1},  
  \label{eq:fermiDirac}
\end{equation}
where the temperature $T$ and the chemical potential $\mu$ depend on properties of the background.  

\subsection{Angular Moment Equations: Two-Moment Model}

The Boltzmann equation is often too expensive to solve directly.  
Instead, approximate equations for angular moments of the distribution function are solved.  
To this end, we define the angular moments of the distribution function
\begin{equation}
  \big\{\,\cJ,\vect{\cH},\vect{\cK}\,\big\}(\vect{z},t)
  =\f{1}{4\pi}\int_{\bbS^{2}}f(\omega,\vect{z},t)\,\{\,1,\vect{\ell},\vect{\ell}\otimes\vect{\ell}\,\}\,d\omega.  
  \label{eq:angularMoments}
\end{equation}
We refer to $\cJ$ (zeroth moment) as the particle density, $\vect{\cH}$ (first moment) as the particle flux, and $\vect{\cK}$ (second moment) as the stress tensor.  
Note that the moments defined in Eq.~\eqref{eq:angularMoments} are \emph{spectral moments} (depending on energy as well as position and time).  
The \emph{grey moments} (depending only on position and time) are obtained by integration over energy:
\begin{equation}
  \big\{\,J,\vect{H},\vect{K}\,\big\}(\vect{x},t)
  =\int_{\bbR^{+}}\big\{\,\cJ,\vect{\cH},\vect{\cK}\,\big\}(\varepsilon,\vect{x},t)\,\varepsilon^{2}d\varepsilon.  
\end{equation}

Taking the zeroth and first moments of Eq.~\eqref{eq:boltzmann} gives the two-moment model, comprising a system of conservation laws with sources
\begin{equation}
  \pd{\vect{\cM}}{t}+\nabla\cdot\vect{\cF}=\f{1}{\tau}\,\vect{\cC}(\vect{\cM}),
  \label{eq:momentEquations}
\end{equation}
where $\vect{\cM}=(\cJ,\vect{\cH})^{T}$ and $\vect{\cF}=(\vect{\cH},\vect{\cK})^{T}$.  
Components of the fluxes in each coordinate direction are $\vect{\cF}^{i}=\vect{e}_{i}\cdot\vect{\cF}=(\vect{e}_{i}\cdot\vect{\cH},\vect{e}_{i}\cdot\vect{\cK})^{T}$, where $\vect{e}_{i}$ is the unit vector parallel to the $i$th coordinate direction.  
On the right-hand side of Eq.~\eqref{eq:momentEquations}, the source term is
\begin{equation}
  \vect{\cC}(\vect{\cM})=\vect{\eta}-\vect{\cD}\,\vect{\cM}, 
  \label{eq:collisionTermMoments}
\end{equation}
where $\vect{\eta}=(\xi\,f_{0},\vect{0})^{T}$ and $\vect{\cD}=\mbox{diag}(\xi,\vect{I})$, with $\vect{I}$ the identity matrix.  

In order to close the system given by Eq.~\eqref{eq:momentEquations}, the components of the stress tensor $\vect{\cK}$ must be related to the lower moments through a closure procedure.  
To this end, Levermore \cite{levermore_1984} defined the Eddington tensor $\vect{k}=\vect{\cK}/\cJ$ and assumed that the radiation field is symmetric about a preferred direction $\widehat{\vect{h}}=\vect{\cH}/|\vect{\cH}|$ so that
\begin{equation}
  \vect{k}=\f{1}{2}\big[\,\big(1-\chi\big)\,\vect{I}+\big(3\,\chi-1\big)\,\widehat{\vect{h}}\otimes\widehat{\vect{h}}\,\big],
  \label{eq:eddingtonTensor}
\end{equation}
where $\chi=\chi(\cJ,|\vect{\cH}|)$ is the Eddington factor.  
The two-moment model is then closed once the Eddington factor is determined from $\cJ$ and $\vect{\cH}$.  
We will return to the issue of determining the Eddington factor in Section~\ref{sec:algebraicClosure}.  
\section{Moment Realizability and Closures for Fermion}
\label{sec:realizability}

Our goal is to design a numerical method for solving the system of equations given by Eq.~\eqref{eq:momentEquations}, which approximates Eq.~\eqref{eq:boltzmann}.  
To this end, we want to obtain solutions that preserve features of the original problem. 

Since we only consider the angular dependence of the distribution function in this section, we simplify the notation by suppressing the $\vect{z}$ and $t$ dependence, and write $f(\omegaNu,\vect{z},t)=f(\omegaNu)$.  

\subsection{Moment Realizability}

\begin{define}
  The moments $\vect{\cM}=\big(\cJ,\vect{\cH}\big)^{T}$ are realizable if they are consistent with a distribution function satisfying $0\le f(\omegaNu) \le 1$.
\end{define}

We begin by stating the bounds on the moments by restating results from Theorem~7.1 in \cite{banachLarecki_2017} in the following Lemma.  
\begin{lemma}
  Let the distribution function be bounded by $0\le f(\omegaNu) \le1$ for all $\omegaNu\in\bbS^{2}$.  
  Then $0\le\cJ\le1$ and $\big(1-\cJ\big)\,\cJ-|\vect{\cH}|\ge0$. 
  \label{lem: MomentRealizable} 
\end{lemma}
\begin{proof}
  First, let $\mu=\cos\thetaNu$ and write
  \begin{equation}
    \cJ=\f{1}{2}\int_{\bbS^{1}}\mathfrak{f}(\mu)\,d\mu,
    \quad\text{where}\quad
    \mathfrak{f}(\mu)=\f{1}{2\pi}\int_{0}^{2\pi}f(\mu,\phiNu)\,d\phiNu.  
  \end{equation}  
  Since $f(\omegaNu)\in[0,1]$, we have $\mathfrak{f}(\mu)\in[0,1]$, which implies $\cJ\in[0,1]$.  
  Next we need to show that $|\vect{\cH}|\le(1-\cJ)\,\cJ$.  
  Let $\widehat{\vect{n}}\in\bbR^{3}$ (independent of $\omegaNu$) be an arbitrary unit vector, and write
  \begin{equation}
    \widehat{\vect{n}}\cdot\vect{\cH}
    =\int_{\bbS^{1}}\mathfrak{f}(\mu)\,\mu\,d\mu,
    \label{eq:NdotH}
  \end{equation}
  where we have aligned the polar axis of the spherical momentum space coordinate system with $\widehat{\vect{n}}$; i.e., $\widehat{\vect{n}}\cdot\vect{\ell}(\omega)=\mu$.  
  First, following \cite{banachLarecki_2017}, we can write the integral on the right-hand side of Eq.~\eqref{eq:NdotH} as
  \begin{equation*}
    (1-\cJ)\cJ - \f{1}{2}\int_{-1}^{1-2\cJ}(1-2\cJ-\mu)\,\mathfrak{f}(\mu)\,d\mu
    -\f{1}{2}\int_{1-2\cJ}^{1}(2\cJ-1+\mu)\,[1-\mathfrak{f}(\mu)]\,d\mu.  
  \end{equation*}
  Since $\mathfrak{f}\in[0,1]$, $(1-2\cJ-\mu)$ is nonnegative on the interval $\mu\in[-1,1-2\cJ]$, and $(2\cJ-1+\mu)$ is nonnegative on the interval $\mu\in[1-2\cJ,1]$, we have $\widehat{\vect{n}}\cdot\vect{\cH}\le(1-\cJ)\,\cJ$.  
  In a similar manner, we can also write the integral on the right-hand side of Eq.~\eqref{eq:NdotH} as
  \begin{equation*}
    -(1-\cJ)\cJ-\f{1}{2}\int_{-1}^{2\cJ-1}(1-2\cJ+\mu)\,[1-\mathfrak{f}(\mu)]\,d\mu
    -\f{1}{2}\int_{2\cJ-1}^{1}(2\cJ-1-\mu)\,\mathfrak{f}(\mu)\,d\mu.  
  \end{equation*}
  Then, since $\mathfrak{f}\in[0,1]$, $(1-2\cJ+\mu)$ is nonpositive on the interval $\mu\in[-1,2\cJ-1]$, and $(2\cJ-1-\mu)$ is nonpositive on the interval $\mu\in[2\cJ-1,1]$, we have $\widehat{\vect{n}}\cdot\vect{\cH}\ge-(1-\cJ)\,\cJ$.  
  We therefore have $-(1-\cJ)\,\cJ\le\widehat{\vect{n}}\cdot\vect{\cH}\le(1-\cJ)\,\cJ$.  
  Since this holds for any unit vector $\widehat{\vect{n}}\in\bbR^{3}$, we must have $|\vect{\cH}|\le(1-\cJ)\,\cJ$.  
\end{proof}

We now define the realizable set $\cR$
\begin{equation}
  \cR:=\big\{\,\vect{\cM}=\big(\cJ,\vect{\cH}\big)^{T}~|~\cJ\in[0,1]~\text{and}~\gamma(\vect{\cM})\equiv(1-\cJ)\cJ-|\vect{\cH}|\ge0\,\big\}
  \label{eq:realizableSet}
\end{equation}

\begin{lemma}
  The set $\cR$, defined in Eq.~\eqref{eq:realizableSet}, is convex.  
\end{lemma}
\begin{proof}
  A convex set $Q$ is a set has the following property: 
  \begin{equation*}
  \theta a + (1-\theta)b \in Q,~\text{for}~\forall a, b \in Q~\text{and}~0\leq\,\theta\,\leq1. 
  \end{equation*} 
  Let $\vect{\cM}_{a}=\big(\cJ_{a},\vect{\cH}_{a}\big)^{T}$ and $\vect{\cM}_{b}=\big(\cJ_{b},\vect{\cH}_{b}\big)^{T}$ to be two arbitrary elements of $\cR$.
  If $\cR$ is convex, we will have $\vect{\cM}_{ab} = \big(\cJ_{ab},\vect{\cH}_{ab}\big)^{T} = \theta \vect{\cM}_{a} + (1-\theta)\vect{\cM}_{b}$ in $\cR$ with $0\leq\theta\leq1$.
  The first component of $\vect{\cM}_{ab}$ is
  \begin{equation*}
  \cJ_{ab} = \theta \cJ_{a} + (1-\theta)\cJ_{b}.
  \end{equation*}
  Since $\theta$, $\cJ_{a}$, $1-\theta$ and $\cJ_{b}$ are all nonnegative, $\cJ_{ab}$ is also nonnegative.
  Meanwhile, since $\cJ_{a}\leq1$ and $\cJ_{b}\leq1$, $\cJ_{ab}\leq1$.
  That's $\cJ_{ab} \in [0,1]$.
  Since the second derivative of $\gamma(\vect{\cM})$ is $-2$ which is always negative, $\gamma(\vect{\cM})$ is a concave function.
  Using Jensen's inequality, we have
  \begin{equation*}
  \gamma(\vect{\cM_{ab}}) \geq \theta\gamma(\vect{\cM_{a}}) + (1-\theta)\gamma(\vect{\cM_{b}}) \ge0.
  \end{equation*}
\end{proof}

The Fig.~\ref{fig:RealizableSetFermionic} illustrates how the convex realizable set $\cR$ looks like in $|\vect{\cH}| - \cJ$ coordinates.
Total 1000,000 distribution functions having form $\mathfrak{f}(\mu;a,b) = \dfrac{1}{e^{a+b\mu}+1}$ were evaluated.
The parameters, $a$ and $b$, were chosen from $[-100,100]$ independently.
The variable $\mu$ ranges from $-1$ to $1$ with $d\mu=\f{1}{128}$.
And the integrals in angular moment expressions are calculated using Trapezoid rule.
\begin{figure}
\centering
\includegraphics[width=1.0\linewidth]{figures/RealizableSetFermionic}
\caption[Numerical plot of the convex realizable set $\cR$.]{
A numerical plot of the convex realizable set $\cR$.
The black line are the boundary given by $\gamma(\vect{\cM}) = 0$. 
The red line are the boundary given by $|\vect{\cH}| = \cJ$.
Every light-blue dot represents a sample whose distribution function $\mathfrak{f}(\mu)$ is an arbitrary Fermi-Dirac function and its angular moments were given by integrals using Trapezoid rule.
Total 1000,000 $\mathfrak{f}(\mu)$ were sampled for making this plot.}
\label{fig:RealizableSetFermionic}
\end{figure}

\begin{lemma}
  Let $\big\{\cJ_{a},\vect{\cH}_{a},\vect{\cK}_{a}\big\}$ be defined as in Eq.~\eqref{eq:angularMoments} with distribution function $f_{a}\in[0,1]$.  
  Similarly, $\big\{\cJ_{b},\vect{\cH}_{b},\vect{\cK}_{b}\big\}$ be moments of a distribution function $f_{b}\in[0,1]$.  
  Let $\Phi^{\pm}(\vect{\cM})=\f{1}{2}\big(\vect{\cM}\pm\widehat{\vect{e}}\cdot\vect{\cF}(\vect{\cM})\big)$, where $\widehat{\vect{e}}\in\bbR^{3}$ is an arbitrary unit vector and $\vect{\cF}(\vect{\cM})=\big(\vect{\cH},\vect{\cK}\big)^{T}$.  
  Then
  \begin{equation}
    \vect{\cM}_{ab} \equiv \Phi^{+}(\vect{\cM}_{a})+\Phi^{-}(\vect{\cM}_{b})\in\cR.
  \end{equation}
\end{lemma}
\begin{proof}
  Using the same treatment we made in proof of Lemma~\ref{lem: MomentRealizable}, we can rewrite the first component of $\vect{\cM}_{ab}$ as
  \begin{align*}
  \f{1}{2}\big(\vect{\cJ_{a}} + \widehat{\vect{e}}\cdot\vect{\cH_{a}}+\vect{\cJ_{b}} - \widehat{\vect{e}}\cdot\vect{\cH_{b}}\big) = &\f{1}{2} \left\lbrace \f{1}{2}\int_{\bbS^{1}}\left[ \mathfrak{f}_{a}(\mu)\,(1+\mu) + \mathfrak{f}_{b}(\mu)\,(1-\mu)\right] \,d\mu \right\rbrace \\
  \equiv  &\f{1}{2}\int_{\bbS^{1}} \mathfrak{f}_{ab}(\mu)\,d\mu
  \end{align*}
  where $\mathfrak{f}_{ab}(\mu) \equiv \f{1}{2}\left[ \mathfrak{f}_{a}(\mu)\,(1+\mu) + \mathfrak{f}_{b}(\mu)\,(1-\mu)\right]$ and $\mathfrak{f}_{ab}(\mu)\in[0,1]$.
  And the second component is
  \begin{align*} 
   \f{1}{2}\big(\vect{\cH_{a}} + \widehat{\vect{e}}\cdot\vect{\cK_{a}}+\vect{\cH_{b}} - \widehat{\vect{e}}\cdot\vect{\cK_{b}}\big) =
   &\f{1}{2} \left\lbrace \f{1}{2}\int_{\bbS^{1}}\left[ \mathfrak{f}_{a}(\mu)\,(1+\mu) + \mathfrak{f}_{b}(\mu)\,(1-\mu)\right]\mu \,d\mu \right\rbrace \\
   \equiv  &\f{1}{2}\int_{\bbS^{1}} \mathfrak{f}_{ab}(\mu)\,\mu d\mu. 
  \end{align*}
  Using Lemma~\ref{lem: MomentRealizable}, we know that $\vect{\cM}_{ab}\in\cR$ and can be written as $\vect{\cM}_{ab}(\cJ_{ab},\vect{\cH}_{ab})$.
\end{proof}

\subsection{Moment Closures}

Flux factor $h=|\vect{\cH}|/\cJ$
\begin{equation}
  \chi(\cJ,h)=\f{1}{3}+\f{2\,(1-\cJ)\,(1-2\cJ)}{3}\,\Theta\Big(\f{h}{1-\cJ}\Big)
\end{equation}

\subsubsection{Maximum Entropy Closure}
Basic principles
Chernohorsky \& Bludman \cite{cernohorskyBludman_1994}
\begin{equation}
  \Theta_{\mbox{\tiny ME}}^{\mbox{\tiny CB}}(x)
  =x^{2}\,\big(\,3-x+3\,x^{2}\,\big)
\end{equation}
Banach \& Larecki
\begin{equation}
  \Theta_{\mbox{\tiny ME}}^{\mbox{\tiny BL}}(x)
  =\f{1}{8}\,\big(\,9\,x^{2}-5+\sqrt{33\,x^{4}-42\,x^{2}+25}\,\big)
\end{equation}

\subsubsection{Kershaw Closure}
Basic principles
Banach \& Larecki \cite{banachLarecki_2017}
\begin{equation}
  \Theta_{\mbox{\tiny K}}^{\mbox{\tiny BL}}(x)=x^{2}
\end{equation}

\subsubsection{Low Occupancy Limit}


\begin{figure}[h!]
  \begin{subfigure}[b]{0.5\textwidth}
    \includegraphics[width=\textwidth]{figures/MabWithMI}
    \caption{$\vect{\cM}_{ab}$ with Minerbo closure.}
  \end{subfigure}
  %
  \begin{subfigure}[b]{0.5\textwidth}
    \includegraphics[width=\textwidth]{figures/MabWithBLKS}
    \caption{$\vect{\cM}_{ab}$ with Kershaw closure.}
  \end{subfigure}
  \begin{subfigure}[b]{0.5\textwidth}
    \includegraphics[width=\textwidth]{figures/MabWithCBME}
    \caption{$\vect{\cM}_{ab}$ with Chernohorsky \& Bludman maximum entropy closure.}
  \end{subfigure}
  \begin{subfigure}[b]{0.5\textwidth}
    \includegraphics[width=\textwidth]{figures/MabWithBLME}
    \caption{$\vect{\cM}_{ab}$ with Banach \& Larecki maximum entropy closure.}
  \end{subfigure}
  \label{fig:MabWithDifferentClosure}
\end{figure}
\section{Algebraic Moment Closures}
\label{sec:algebraicClosure}

Algebraic moment closures for the two-moment model are computationally efficient as they provide the Eddington factor in Eq.~\eqref{eq:eddingtonTensor} in closed form as a function of the density $\cJ$ and the flux factor $h=|\vect{\cH}|/\cJ$.  
The family of algebraic closures we consider in this paper can be written in the form \cite{cernohorskyBludman_1994}
\begin{equation}
  \chi(\cJ,h)=\f{1}{3}+\f{2\,(1-\cJ)\,(1-2\cJ)}{3}\,\Theta\Big(\f{h}{1-\cJ}\Big),
  \label{eq:eddingtonFactor}
\end{equation}
where the closure function $\Theta(x)$ varies with the specifics of the closure procedure.  
We will consider two basic closure procedures in more detail below: the maximum entropy (ME) closure and the Kershaw (K) closure.  

\paragraph{Low Occupancy Limit}
We note in passing that in the low occupancy limit ($\cJ\ll1$), the Eddington factor in Eq.~\eqref{eq:eddingtonFactor} becomes independent of $\cJ$; i.e.,
\begin{equation}
  \chi(\cJ,h)\to\chi_{0}(h)=\f{1}{3}+\f{2}{3}\,\Theta\big(h\big).  
  \label{eq:eddingtonFactorLow}
\end{equation}

\subsection{Maximum Entropy Closure}

The basic principle behind the maximum entropy closure is that one can approximate the real fermion distribution with a distribution which have the Fermi-Dirac distribution form 
\begin{equation}
f(\omega)=\f{1}{e^{a + b\omega}+1}, 
\end{equation} 
meanwhile maximize the entropy function
\begin{equation}
S[f(\omega)] = (1-f)\log(1-f) + f\log f.
\end{equation} 
Given moments $\cJ$ and $\vect{\cH}$, a distribution satisfying above requirements can be found. 
Therefore, the stress tensor $\vect{\cK}$.

One ME closure given by Chernohorsky \& Bludman \cite{cernohorskyBludman_1994} is
\begin{equation}
  \Theta_{\mbox{\tiny ME}}^{\mbox{\tiny CB}}(x)
  =x^{2}\,\big(\,3-x+3\,x^{2}\,\big).
\end{equation}

Another ME closure given by Banach \& Larecki \cite{banachLarecki_2017b} is
\begin{equation}
  \Theta_{\mbox{\tiny ME}}^{\mbox{\tiny BL}}(x)
  =\f{1}{8}\,\big(\,9\,x^{2}-5+\sqrt{33\,x^{4}-42\,x^{2}+25}\,\big).
\end{equation}

\subsection{Kershaw Closure}
The basic principle behind Kershaw closure is derived from the fact that $\cR$ is convex.
Due to the convexity, one can write any element in $\cR$ as a convex combination of the elements on the realization boundary. 
Therefore, Kershaw closure satisfies realizability inherently.
For fermionic radiation, the boundary is based on the Heaviside step functions.
Here we introduce a closure function $\Theta(x)$ given by Banach \& Larecki \cite{banachLarecki_2017a}:
\begin{equation}
  \Theta_{\mbox{\tiny K}}^{\mbox{\tiny BL}}(x)=x^{2}
\end{equation}

Figure~\ref{fig:MabWithDifferentClosure} illustrates the behavior of these four closure.
As it shows, Minerbo closure (right bottom) is the only one among those four closures that does not preserve realizability.
\begin{figure}[h]
  \centering
  \begin{tabular}{cc}
    \includegraphics[width=0.5\textwidth]{figures/MabWithBLME}
    \includegraphics[width=0.5\textwidth]{figures/MabWithCBME} \\
    \includegraphics[width=0.5\textwidth]{figures/MabWithBLKS}
    \includegraphics[width=0.5\textwidth]{figures/MabWithMI}
  \end{tabular}
   \caption{Illustration of $\vect{\cM}_{ab}$ with four different closures: Banach \& Larecki maximum entropy closure (left top), Chernohorsky \& Bludman maximum entropy closure (right top), Kershaw closure (left bottom), and Minerbo closure (right bottom).
   Total $10^{6}$ pairs of random Fermi-Dirac distribution were generated.
   With these pairs, $\vect{\cM}_{ab}$ with different closures were calculated separately and marked with light-blue points.
   Black lines define the boundary of $\cR$: $\gamma(\vect{\cM}) = 0$.}
  \label{fig:MabWithDifferentClosure}
\end{figure}
\section{Discontinuous Galerkin Method}
\label{sec:dg}

Here we briefly outline the DG method for the moment equations.  
(See, e.g., \cite{cockburnShu_2001}, for a comprehensive review on the application of DG methods to solve hyperbolic conservation laws.)  
Since we do not include any physics that couples the energy dimension, the particle energy $\epsilonNu$ is simply treated as a parameter.  
For notational convenience, we will suppress explicit energy dependence of the moments.  
Employing Cartesian coordinates, we write the moment equations in $d$ spatial dimensions as
\begin{equation}
  \pd{\vect{\cM}}{t}+\sum_{i=1}^{d}\pderiv{}{x^{i}}\big(\,\vect{\cF}^{i}(\vect{\cM})\,\big)
  =\frac{1}{\tau}\,\vect{\cC}(\vect{\cM}),
  \label{eq:angularMomentsCartesian}
\end{equation}
where $x^{i}$ is the coordinate along the $i$th coordinate dimension.  
We divide the spatial domain $D$ into a disjoint union $\mathscr{T}$ of open elements $\bK$, so that $D = \cup_{\bK \in \mathscr{T}}\bK$.  
We require that each element is a $d$-dimensional box in the logical coordinates; i.e.,
\begin{equation}
  \bK=\{\,\vect{x} : x^{i} \in K^{i} := (\xL^{i},\xH^{i}),~|~i=1,\ldots,d\,\}, 
\end{equation}
with surface elements denoted $\tilde{\bK}^{i}=\times_{j\ne i}K^{j}$.  
We let $|\bK|$ denote the volume of an element
\begin{equation}
  |\bK| = \int_{\bK}d\vect{x}, \quad\text{where}\quad d\vect{x} = \prod_{i=1}^{d}dx^{i}.  
\end{equation}
We also define $\tilde{\vect{x}}^{i}$ as the coordinates orthogonal to the $i$th dimension, so that as a set $\vect{x}=\{\tilde{\vect{x}}^{i},x^{i}\}$.  
The width of an element in the $i$th dimension is $|K^{i}|=\xH^{i}-\xL^{i}$.  

We let the approximation space for the DG method, $\mathbb{V}^{k}$, be constructed from the tensor product of one-dimensional polynomials of maximal degree $k$.  
Note that functions in $\mathbb{V}^{k}$ can be discontinuous across element interfaces.  
The semi-discrete DG problem is to find $\vect{\cM}_{h}\in\mathbb{V}^{k}$ (which approximates $\vect{\cM}$ in Eq.~\eqref{eq:angularMomentsCartesian}) such that
\begin{align}
  &\pd{}{t}\int_{\bK}\vect{\cM}_{h}\,v\,d\vect{x}
  +\sum_{i=1}^{d}\int_{\tilde{\bK}^{i}}
  \big(\,
    \widehat{\bcF}^{i}(\vect{\cM}_{h})\,v\big|_{\xH^{i}}
    -\widehat{\bcF}^{i}(\vect{\cM}_{h})\,v\big|_{\xL^{i}}
  \,\big)\,d\tilde{\bx}^{i} \nonumber \\
  &\hspace{24pt}
  -\sum_{i=1}^{d}\int_{\bK}\bcF^{i}(\vect{\cM}_{h})\,\pderiv{v}{x^{i}}\,d\vect{x}
  =\f{1}{\tau}\int_{\bK}\bcC(\vect{\cM}_{h})\,v\,d\vect{x},
  \label{eq:semidiscreteDG}
\end{align}
for all $v\in\mathbb{V}^{k}$ and all $\bK\in\mathscr{T}$.  

In Eq.~\eqref{eq:semidiscreteDG}, $\widehat{\bcF}^{i}(\vect{\cM}_{h})$ is a numerical flux, approximating the flux on the surface of $\bK$ with unit normal along the $i$th coordinate direction.  
It is evaluated with a flux function $\vect{\mathscr{F}}^{i}$ using the DG approximation from both sides of the element interface; i.e.,
\begin{equation}
  \widehat{\bcF}^{i}(\vect{\cM}_{h})\big|_{x^{i}}=\vect{\mathscr{F}}^{i}(\vect{\cM}_{h}(x^{i,-},\tilde{\bx}^{i}),\vect{\cM}_{h}(x^{i,+},\tilde{\bx}^{i})),
\end{equation}
where superscripts $-/+$ in the arguments of $\vect{\cM}_{h}$ indicate that the function is evaluated to the immediate left/right of $x^{i}$.  
In this paper we use the simple Lax-Friedrichs (LF) flux given by
\begin{equation}
  \vect{\mathscr{F}}_{\mbox{\tiny LF}}^{i}(\vect{\cM}_{a},\vect{\cM}_{b})
  =\f{1}{2}\,\big(\,\bcF^{i}(\vect{\cM}_{a})+\bcF^{i}(\vect{\cM}_{b})-\alpha^{i}\,(\,\vect{\cM}_{b}-\vect{\cM}_{a}\,)\,\big),
  \label{eq:fluxFunctionLF}
\end{equation}
where $\alpha^{i}$ is the largest eigenvalue (in absolute value) of the flux Jacobian $\partial\bcF^{i}/\partial\vect{\cM}$.  
For particles propagating at the speed of light, we can simply take $\alpha^{i}=1$ (i.e., the global LF flux).  

\begin{rem}
For simplicity, in Eq.~\eqref{eq:semidiscreteDG}, we have approximated the opacities $\sigma_{\Ab}$ and $\sigma_{\Scatt}$ (and thus $\xi$ and $\tau$) on the right-hand side of Eq.~\eqref{eq:angularMomentsCartesian} with constants in each element; i.e., $\sigma_{\Ab},\sigma_{\Scatt}\in\bbV^{0}$.  
\end{rem}
\section{Positivity-Preserving IMEX Schemes}
\label{sec:imex}

The semi-discretization of the moment equations with the DG method given in Eq.~\eqref{eq:semidiscreteDG} results in a system of ordinary differential equations (ODEs) in each element of the form
\begin{equation}
  \dot{\vect{u}}
  =\vect{\cT}(\vect{u})+\f{1}{\tau}\,\vect{\cQ}(\vect{u}),
\end{equation}
where $\vect{u}$ are the degrees of freedom evolved with the DG method (e.g., expansion coefficients resulting from a modal expansion).  
The transport term $\vect{\cT}$ is due to the second and third term in the left-hand side of Eq.~\eqref{eq:semidiscreteDG}, while the collision term $\vect{\cQ}$ is due to the right-hand side of Eq.~\eqref{eq:semidiscreteDG}.  

In the application of interest to us, the collision term is stiff ($\tau\ll1$) and must be treated with implicit methods, while we can resolve the time scales induced by the transport term, which we will treat with explicit methods; i.e., we will use IMEX methods \cite{pareschiRusso_2005}.  
Until recently, high-order (second or higher order temporal accuracy) positivity-preserving IMEX methods with time step restrictions solely due to the transport operator were not known.  
Chertock et al. \cite{chertock_etal_2015} presented second order accurate IMEX schemes with a correction step.  
The correction step includes the transport operator, and positivity is subject to a time step restriction that scales with $1/\sqrt{\tau}$.  
More recently, Hu et al. \cite{hu_etal_2017}, presented new IMEX schemes for a class of BGK-type collision operators with a correction step that does not include the transport operator.  
In this case, positivity is only subject to time step restrictions stemming from the transport operator.  
These IMEX schemes take the following form \cite{hu_etal_2017}
\begin{align}
  \vect{u}^{(i)}
  &=\vect{u}^{n}
  +\dt\sum_{i=1}^{j-1}\tilde{\alpha}_{ij}\,\vect{\cT}(\vect{u}^{(j)})
  +\dt\sum_{i=1}^{j}\alpha_{ij}\,\f{1}{\tau}\,\vect{\cQ}(\vect{u}^{(j)}),
  \quad i=1,\ldots,s, \\
  \tilde{\vect{u}}^{n+1}
  &=\vect{u}^{n}
  +\dt\sum_{i=1}^{s}\tilde{w}_{i}\,\vect{\cT}(\vect{u}^{(i)})
  +\dt\sum_{i=1}^{s}w_{i}\,\f{1}{\tau}\,\vect{\cQ}(\vect{u}^{(i)}), \\
  \vect{u}^{n+1}
  &=\tilde{\vect{u}}^{n+1}-\alpha\,\dt^{2}\,\f{1}{\tau^{2}}\,\vect{\cQ}'(\vect{u}^{*})\,\vect{\cQ}(\vect{u}^{n+1}), 
\end{align}
\section{Realizability-Preserving DG-IMEX Scheme}
\label{sec:realizableDGIMEX}

The realizability preserving DG scheme is designed to ensure realizability of the cell averages in each element $\bK$, defined as
\begin{equation}
  \vect{\cM}_{\bK}
  =\f{1}{|\bK|}\int_{\bK}\vect{\cM}_{h}\,d\bx.  
\end{equation}
Then, with $v=1$ in Eq.~\eqref{eq:semidiscreteDG}, the stage values for the cell average in the IMEX scheme in Eq.~\eqref{eq:imexStagesRewrite} are given by
\begin{equation}
  \vect{\cM}_{\bK}^{(i)}
  =c_{i0}\,\vect{\cM}_{\bK}^{n}
  +\sum_{j=1}^{i-1}c_{ij}\,\vect{\cM}_{\bK}^{(ij)}
  +a_{ii}\,\dt\,\f{1}{\tau}\,\big(\,\vect{\eta}-\vect{\cD}\,\vect{\cM}_{\bK}^{(i)}\,\big),
  \label{eq:imexStagesCellAverage}
\end{equation}
where we have defined
\begin{equation}
  \vect{\cM}_{\bK}^{(ij)}
  =\vect{\cM}_{\bK}^{(j)}-\hat{c}_{ij}\,\dt\,\big\langle\,\nabla\cdot\vect{\cF}(\vect{\cM}_{h}^{(j)})\,\big\rangle_{\bK},
\end{equation}
and the cell average of the divergence operator is (cf. Section~\ref{sec:dg})
\begin{equation}
  \big\langle\,\nabla\cdot\vect{\cF}(\vect{\cM}_{h})\,\big\rangle_{\bK}
  =\f{1}{|\bK|}\sum_{i=1}^{d}\int_{\tilde{\bK}^{i}}
  \big(\,\widehat{\bcF}^{i}(\vect{\cM}_{h})\big|_{\xH^{i}}-\widehat{\bcF}^{i}(\vect{\cM}_{h})\big|_{\xL^{i}}\,\big)\,d\tilde{\vect{x}}^{i}.  
\end{equation}

\begin{lemma}
  Let $\vect{\cM}_{\bK}^{(i)}$ satisfy Eq.~\eqref{eq:imexStagesCellAverage}.
  Assume that $\vect{\cM}_{\bK}^{n}\in\cR$ and $\vect{\cM}_{\bK}^{(ij)}\in\cR\,\forall\,j\le i-1$.  
  Then, $\vect{\cM}_{\bK}^{(i)}\in\cR$.  
\end{lemma}
\begin{proof}
  The first two terms on the right-hand side of Eq.~\eqref{eq:imexStagesCellAverage} constitute a convex combination of elements in $\cR$; i.e.,
  \begin{equation*}
    c_{i0}\,\vect{\cM}_{\bK}^{n}+\sum_{j=1}^{i-1}c_{ij}\,\vect{\cM}_{\bK}^{(ij)}\in\cR.
  \end{equation*}
  Since $a_{ii}\,\dt/\tau>0$, it follows from Lemma~\ref{lem:implicitStep} that $\vect{\cM}_{\bK}^{(i)}\in\cR$.  
\end{proof}

We proceed to establish conditions for which $\vect{\cM}_{\bK}^{(ij)}\in\cR$.  
To this end, to simplify notation, we consider the moments
\begin{equation}
  \vect{\cM}_{\bK}^{*}
  =\vect{\cM}_{\bK}-\hat{c}\,\dt\,\big\langle\,\nabla\cdot\vect{\cF}(\vect{\cM}_{h})\,\big\rangle_{\bK},
\end{equation}
where $\vect{\cM}_{\bK}$ is the cell average of $\vect{\cM}_{h}\in\bbV^{k}$ and $\hat{c}\ge0$.  
\begin{lemma}
  Let $\{s_{i}\}_{i=1}^{d}$ be a set of positive constants satisfying $\sum_{i=1}^{d}=1$.  
  If for each $i\in\{1,\ldots,d\}$, 
  \begin{equation}
    \Gamma^{i}\big[\vect{\cM}_{h}\big](\tilde{\vect{x}}^{i})
    :=\f{1}{|K^{i}|}
    \Big[\,\int_{K^{i}}\vect{\cM}_{h}\,dx^{i}-\f{\hat{c}\,\dt}{s_{i}}\big(\,\widehat{\bcF}^{i}(\vect{\cM}_{h})\big|_{\xH^{i}}-\widehat{\bcF}^{i}(\vect{\cM}_{h})\big|_{\xL^{i}}\,\big)\,\Big]\in\cR,
    \label{eq:realizableGamma}
  \end{equation}
  then $\vect{\cM}_{\bK}^{*}\in\cR$.  
\end{lemma}
\begin{proof}
  It is easy to show that $\vect{\cM}_{\bK}^{*}$ can be expressed as the convex combination
  \begin{equation}
    \sum_{i=1}^{d}s_{i}\,\f{1}{|\tilde{\vect{\bK}}^{i}|}\int_{\tilde{\bK}^{i}}\Gamma^{i}\big[\vect{\cM}_{h}\big]\,d\tilde{\vect{x}}^{i}.  
    \label{eq:cellAverageInTermsOfGamma}
  \end{equation}
  The result follows immediately.  
\end{proof}
\begin{rem}
  If a quadrature rule $\tilde{\vect{Q}}^{i}:C^{0}\to\bbR$, with positive weights, and points defined by the set $\tilde{\vect{S}}^{i}$, is used to approximate the integral over $\tilde{\bK}^{i}$ in \eqref{eq:cellAverageInTermsOfGamma}, it is sufficient for \eqref{eq:realizableGamma} to hold in the quadrature points $\tilde{\vect{S}}^{i}\subset\tilde{\bK}^{i}$.  
\end{rem}
\section{Realizability-Enforcing Limiter}
\label{sec:limiter}

Condition~2 of Theorem~\ref{the:realizableDGIMEX} requires that the polynomial approximation $\vect{\cM}_{h}=\vect{\cM}_{h}^{(j)}$ ($j\in\{0,\ldots,i-1\}$) is realizable in every point in the quadrature set $S=\cup_{k=1}^{d}\hat{\vect{S}}^{k}$.  
Following Zhang \& Shu \cite{zhangShu_2010a} we use the limiter in \cite{liuOsher_1996} to enforce the bounds on the zeroth moment $\cJ$.  
%The bound-preserving DG-IMEX method developed in previous sections is designed to preserve realizability of the cell averaged moments, i.e., $\vect{\cM}_{\bK}\in\cR$, provided sufficiently accurate quadratures are used to integrate integrals in the DG method, a CFL condition is satisfied, and that the polynomial approximation $\vect{\cM}_{h}$, at time $t^{n}$, is realizable in a set of quadrature points in each element $\bK$.  
%We denote this quadrature set by $S=\cup_{k=1}^{d}\hat{\vect{S}}^{k}\subset\bK$.  
%In the DG method, we use the limiter proposed by Zhang \& Shu \cite{zhangShu_2010a} for scalar conservation laws to enforce the bounds on the zeroth moment $\cJ$ (see also \cite{liuOsher_1996}).  
We replace the polynomial $\cJ_{h}(\vect{x})$ with the limited polynomial
\begin{equation}
  \tilde{\cJ}_{h}(\vect{x})
  =\vartheta_{1}\,\cJ_{h}(\vect{x})+(1-\vartheta_{1})\,\cJ_{\bK},
  \label{eq:limitDensity}
\end{equation}
where the limiter parameter $\vartheta_{1}$ is given by
\begin{equation}
  \vartheta_{1}
  =\min\Big\{\,\Big|\f{M-\cJ_{\bK}}{M_{S}-\cJ_{\bK}}\Big|,\Big|\f{m-\cJ_{\bK}}{m_{S}-\cJ_{\bK}}\Big|,1\,\Big\},
\end{equation}
with $m=0$ and $M=1$, and
\begin{equation}
  M_{S}=\max_{\vect{x}\in S}\cJ_{h}(\vect{x})
  \quad\text{and}\quad
  m_{S}=\min_{\vect{x}\in S}\cJ_{h}(\vect{x}).  
\end{equation}

In the next step, we ensure realizability of the moments by following the framework of \cite{zhangShu_2010b}, developed to ensure positivity of the pressure when solving the Euler equations of gas dynamics.  
We let $\widetilde{\vect{\cM}}_{h}=\big(\tilde{\cJ}_{h},\vect{\cH}_{h}\big)^{T}$.  
Then, if $\widetilde{\bcM}_{h}$ lies outside $\cR$ for any quadrature point $\vect{x}_{q}\in S$, i.e., $\gamma(\widetilde{\bcM}_{h})<0$, there exists an intersection point of the straight line, $\vect{s}_{q}(\psi)$, connecting $\vect{\cM}_{\bK}\in\cR$ and $\widetilde{\vect{\cM}}_{h}$ evaluated in the troubled quadrature point $\vect{x}_{q}$, denoted $\widetilde{\vect{\cM}}_{q}$, and the boundary of $\cR$.  
This line is given by the convex combination 
\begin{equation}
  \vect{s}_{q}(\psi)=\psi\,\widetilde{\vect{\cM}}_{q}+(1-\psi)\,\bcM_{\bK},
\end{equation}
where $\psi\in[0,1]$, and the intersection point $\psi_{q}$ is obtained by solving $\gamma(\bs_{q}(\psi))=0$ for $\psi$, using the bisection algorithm\footnote{In practice, $\psi$ needs not be accurate to many significant digits, and the bisection algorithm can be terminated after a few iterations.}.  
We then replace the polynomial representation $\widetilde{\vect{\cM}}_{h}\to\widehat{\vect{\cM}}_{h}$, where
\begin{equation}
  \widehat{\vect{\cM}}_{h}(\vect{x})=\vartheta_{2}\,\widetilde{\vect{\cM}}_{h}(\vect{x})+(1-\vartheta_{2})\,\vect{\cM}_{\bK},
  \label{eq:limitMoments}
\end{equation}
and $\vartheta_{2}=\min_{q}\psi_{q}$ is the smallest $\psi$ obtained in the element by considering all the troubled quadrature points.  
This limiter is conservative in the sense that it preserves the cell-average $\widehat{\vect{\cM}}_{\bK}=\widetilde{\vect{\cM}}_{\bK}=\vect{\cM}_{\bK}$.  

The realizability-preserving property of the DG-IMEX scheme results from the following theorem.
\begin{theorem}
  Consider the IMEX scheme in Eqs.~\eqref{imexStages}-\eqref{eq:imexCorrection} applied to the DG discretization of the two-moment model in Eq.~\eqref{eq:semidiscreteDG}.  
  Suppose that
  \begin{itemize}
    \item[1.] The conditions of Theorem~\ref{the:realizableDGIMEX} hold.  
    \item[2.] With $\vect{\cM}_{\bK}^{(i)}\in\cR$, the limiter described above is invoked to enforce 
    \begin{equation*}
      \vect{\cM}_{h}^{(i)}(\vect{x})\in\cR ~ \text{for all} ~ \vect{x} \in S.  
    \end{equation*}
    \item[3.] The IMEX scheme is GSA.  
  \end{itemize}
  Then $\vect{\bcM}_{\bK}^{n+1}\in\cR$.  
  \label{the:realizableDGIMEX2}
\end{theorem}
\begin{proof}
  By Theorem~\ref{the:realizableDGIMEX} (with $i=1$), we have $\vect{\cM}_{\bK}^{(1)}\in\cR$.  
  Application of the realizability-enforcing limiter gives $\vect{\cM}_{h}^{(1)}(\vect{x})\in\cR$ for all $\vect{x} \in S$.  
  Repeated application of these steps give $\vect{\cM}_{h}^{(i)}(\vect{x})\in\cR$ for all $\vect{x} \in S$ and $i\in\{1,\ldots,s\}$.  
  Since the IMEX scheme is GSA, $\tilde{\vect{\cM}}_{\bK}^{n+1}\in\cR$.  
  Finally, $\vect{\bcM}_{\bK}^{n+1}\in\cR$ follows from Lemma~\ref{lem:imexCorrectionCellAverage}.  
\end{proof}
\section{Numerical Tests}
\label{sec:numerical}

In this section we present numerical results obtained with the DG-IMEX scheme developed in this paper.  
The first set of tests (Section~\ref{sec:smoothProblems}) are included to compare the time integration schemes in various regimes.  
We are not concerned with moment realizability in Section~\ref{sec:smoothProblems}, and we do not apply the realizability-enforcing limiter in these tests.  
The tests in Sections~\ref{sec:packedBeam} and \ref{sec:fermionImplosion} are designed specifically to demonstrate the robustness of the scheme to dynamics near the boundary of the realizable set $\cR$.  
The test in Section~\ref{sec:homogeneousSphere} (Homogeneous Sphere) is of astrophysical interest.  
Here we consider moment realizability and compare results obtained with various moment closures.  

\subsection{Problems with Known Smooth Solutions}
\label{sec:smoothProblems}

To compare the accuracy of the IMEX schemes, we present results from smooth problems in streaming, absorption, and scattering dominated regimes in one spatial dimension.  
For all tests in this subsection, we use third order accurate spatial discretization (polynomials of degree $k=2$) and we employ the maximum entropy closure in the low occupancy limit (i.e., the Minerbo closure).  
We compare results obtained using IMEX schemes proposed here (PA2+ and PD-ARS) with IMEX schemes from Hu et al. \cite{hu_etal_2018} (PA2), McClarren et al. \cite{mcclarren_etal_2008} (PC2), Pareschi \& Russo \cite{pareschiRusso_2005} (SSP2332), and Cavaglieri \& Bewley \cite{cavaglieriBewley2015} (RKCB2).  
In the streaming test, we also include results obtained with second-order and third-order accurate explicit strong stability-preserving Runge-Kutta methods \cite{gottlieb_etal_2001} (SSPRK2 and SSPRK3, respectively).  
See \ref{app:butcherTables} for further details.  
The time step is set to $\dt=0.1\times\dx$.  

When comparing the numerical results to analytic solutions, errors are computed in the $L^{1}$-error norm.  
We compare results either in the absolute error ($E_{\mbox{\tiny Abs}}^{1}$) or the relative error ($E_{\mbox{\tiny Rel}}^{1}$), defined for a scalar quantity $u_{h}$ (approximating $u$) as
\begin{equation}
  E_{\mbox{\tiny Abs}}^{1}[u_{h}](t)
  =\f{1}{|D|}\sum_{\bK\in\mathscr{T}}\int_{\bK}|u_{h}(\vect{x},t)-u(\vect{x},t)|\,d\vect{x}
  \label{eq:errorNormAbsolute}
\end{equation}
and
\begin{equation}
  E_{\mbox{\tiny Rel}}^{1}[u_{h}](t)
  =\f{1}{|D|}\sum_{\bK\in\mathscr{T}}\int_{\bK}|u_{h}(\vect{x},t)-u(\vect{x},t)|/|u(\vect{x},t)|\,d\vect{x},
  \label{eq:errorNormRelative}
\end{equation}
respectively.  
The integrals in Eqs.~\eqref{eq:errorNormAbsolute} and \eqref{eq:errorNormRelative} are computed with a simple $3$-point equal weight quadrature.  

\subsubsection{Sine Wave: Streaming}

The first test involves the streaming part only, and does not include any collisions ($\sigma_{\Ab}=\sigma_{\Scatt}=0$).  
We consider a periodic domain $D=\{x:x\in[0,1]\}$, and let the initial condition be given by
\begin{equation}
  \cJ(x,t=0)=\cH_{x}(x,t=0)=0.5+0.49\times\sin\big(2\pi\,x\big).  
  \label{eq:initialConditionStreaming}
\end{equation}
We evolve until $t=10$, when the sine wave has completed 10 crossings of the computational domain.  
We vary the number of elements ($N$) from $8$ to $128$ and compute errors for various time stepping schemes.  

In Figure~\ref{fig:SineWaveStreaming}, the absolute error for the number density $E_{\mbox{\tiny Abs}}^{1}[\cJ_{h}](t=10)$ is plotted versus $N$ (see figure caption for details).  
Errors obtained with SSPRK3 are smallest and decrease as $N^{-3}$ (cf. bottom black dash-dot reference line), as expected for a scheme combining third-order accurate time stepping with third-order accurate spatial discretization.  
For all the other schemes, using second-order accurate explicit time stepping, the error decreases as $N^{-2}$.  
Among the second-order accurate methods, SSP2332 has the smallest error, followed by RKCB2.  
Errors for the remaining schemes (including SSPRK2) are indistinguishable on the plot.  
\begin{figure}[H]
  \centering
    \includegraphics[width=\textwidth]{figures/SineWaveStreaming}
   \caption{Absolute error (cf. Eq.~\eqref{eq:errorNormAbsolute}) versus number of elements $N$ for the streaming sine wave test.  Results employing various time stepping schemes are compared: SSPRK2 (cyan triangles pointing up), SSPRK3 (cyan triangles pointing down), PA2 (red), PA2+ (purple), PC2 (blue), RKCB2 (dark green), SSP2332 (green), and PD-ARS (light red circles).  Black dash-dot reference lines are proportional to $N^{-1}$ (top), $N^{-2}$ (middle), and $N^{-3}$ (bottom), respectively.}
  \label{fig:SineWaveStreaming}
\end{figure}

\subsubsection{Sine Wave: Damping}

The next test we consider, adapted from \cite{skinnerOstriker_2013}, consists of a sine wave propagating with unit speed in a purely absorbing medium ($f_{0}=0$, $\sigma_{\Scatt}=0$), which results in exponential damping of the wave amplitude.  
We consider a periodic domain $D=\{x:x\in[0,1]\}$, and let the initial condition ($t=0$) be given as in Eq.~\eqref{eq:initialConditionStreaming}.  
For a constant absorption opacity $\sigma_{\Ab}$, the analytical solution at $t>0$ is given by
\begin{equation}
  \cJ(x,t)=\cJ_{0}(x-t)\times\exp(-\sigma_{\Ab} t)
  \quad\text{and}\quad
  \cH_{x}(x,t)=\cJ(x,t),
\end{equation}
where $\cJ_{0}(x)=\cJ(x,0)$.  

We compute numerical solutions for three values of the absorption opacity ($\sigma_{\Ab}=0.1$, $1$, and $10$), and adjust the end time $t_{\mbox{\tiny end}}$ so that $\sigma_{\Ab}t_{\mbox{\tiny end}}=10$, and the initial condition has been damped by factor $e^{-10}$.  
Thus, for $\sigma_{\Ab}=0.1$ the sine wave crosses the domain 100 times, while for $\sigma_{\Ab}=10$, it crosses the grid once.  

Figure~\ref{fig:SineWaveDamping} shows convergence results, obtained using different values of $\sigma_{\Ab}$, for various IMEX schemes at $t=t_{\mbox{\tiny end}}$.  
Results for $\sigma_{\Ab}=0.1$, $1$, and $10$ are plotted with red, green, and blue lines, respectively (see figure caption for further details).  
All the second-order accurate schemes (PA2, PA2+, RKCB2, and SSP2332) display second-order convergence rates (cf. bottom, black dash-dot reference line).  
For $\sigma_{\Ab}=0.1$, SSP2332 is the most accurate among these schemes, while PA2+ is the most accurate for $\sigma_{\Ab}=10$.  
On the other hand, PC2 and PD-ARS are indistinguishable and display at most first-order accurate convergence, as expected.  
(For $\sigma_{\Ab}=0.1$, PC2 and PD-ARS are the most accurate schemes for $N=8$ and $N=16$.)

\begin{figure}[H]
  \centering
    \includegraphics[width=\textwidth]{figures/SineWaveDamping}
   \caption{Relative error (cf. Eq.~\eqref{eq:errorNormRelative}) versus number of elements for the damping sine wave test.  Results for different values of the absorption opacity $\sigma_{\Ab}$, employing various IMEX time stepping schemes, are compared.  Errors for $\sigma_{\Ab}=0.1$, $1$, and $10$ are plotted with red, green, and blue lines, respectively.  The IMEX schemes employed are: PA2 (triangles pointing left), PA2+ (triangles pointing right), PC2 (asterisk), RKCB2 ($\times$), SSP2332 ($+$), and PD-ARS (circles).  Black dash-dot reference lines are proportional to $N^{-1}$ (top) and $N^{-2}$ (bottom), respectively.}
  \label{fig:SineWaveDamping}
\end{figure}

\subsubsection{Sine Wave: Diffusion}

The final test with known smooth solutions, adopted from \cite{radice_etal_2013}, is diffusion of a sine wave in a purely scattering medium ($f_{0}=0$, $\sigma_{\Ab}=0$).  
The computational domain $D=\{x:x\in[-3,3]\}$ is periodic, and the initial condition is given by
\begin{equation}
  \cJ_{0}(x)=0.5+0.49\times\sin\big(\f{\pi\,x}{3}\big)
  \quad\text{and}\quad
  \cH_{x,0}
  =-\f{1}{3\sigma_{\Scatt}}\pderiv{\cJ_{0}}{x}.  
  \label{eq:initialConditionDiffusion}
\end{equation}
For a sufficiently high scattering opacity, the moment equations limit to a diffusion equation for the number density (deviations appear at the $1/\sigma_{\Scatt}^{2}$-level).  
With the initial conditions in Eq.~\eqref{eq:initialConditionDiffusion}, the analytical solution to the limiting diffusion equation is given by
\begin{equation}
  \cJ(x,t)=\cJ_{0}(x)\times\exp\big(-\f{\pi^{2}\,t}{27\,\sigma_{\Scatt}}\big),
\end{equation}
and $\cH_{x}=(3\,\sigma_{\Scatt})^{-1}\pd{\cJ}{x}$.  
When computing errors for this test, we compare the numerical results obtained with the two-moment model to the analytical solution to the limiting diffusion equation.  
We compute numerical solutions using three values of the scattering opacity ($\sigma_{\Scatt}=10^{2}$, $10^{3}$, and $10^{4}$), and adjust the end time so that $t_{\mbox{\tiny end}}/\sigma_{\Scatt}=1$.  
The initial amplitude of the sine wave has then been reduced by a factor $e^{-\pi^{2}/27}\approx0.694$ for all values of $\sigma_{\Scatt}$.  

\begin{figure}[H]
  \centering
  \includegraphics[width=1.0\textwidth]{figures/SineWaveDiffusionN}
   \caption{Absolute error (cf. Eq.~\eqref{eq:errorNormAbsolute}) for the number density $\cJ$ versus number of elements for the sine wave diffusion test.  Results with different values of the scattering opacity $\sigma_{\Scatt}$, employing different IMEX schemes, are compared.  Errors with $\sigma_{\Scatt}=10^{2}$, $10^{3}$, and $10^{4}$ are plotted with red, green, and blue lines, respectively.  The IMEX schemes employed are: PA2 (triangle pointing left), PA2+ (triangle pointing right), PC2 (asterisk), RKCB2 (cross), SSP2332 (plus), and PD-ARS (circle).  Black dash-dot reference lines are proportional to $N^{-1}$ (top) and $N^{-2}$ (bottom), respectively.}
  \label{fig:SineWaveDiffusionJ}
\end{figure}

\begin{figure}[H]
  \centering
  \includegraphics[width=1.0\textwidth]{figures/SineWaveDiffusionG}
   \caption{Same as in Figure~\ref{fig:SineWaveDiffusionJ}, but for the number flux $\cH_{x}$.}
  \label{fig:SineWaveDiffusionH}
\end{figure}

In Figures~\ref{fig:SineWaveDiffusionJ} and \ref{fig:SineWaveDiffusionH} we plot the absolute error, obtained using different values of $\sigma_{\Scatt}$, for various IMEX schemes at $t=t_{\mbox{\tiny end}}$.  
Results for $\sigma_{\Scatt}=10^{2}$, $10^{3}$, and $10^{4}$ are plotted with red, green, and blue lines, respectively (see figure caption for further details).  
(Scheme PC2 has been shown to work well for this test \cite{radice_etal_2013}, but is included here for comparison with the other IMEX schemes.)
Schemes PD-ARS, RKCB2, and SSP2332 are accurate for this test, and display third-order accuracy for the number density $\cJ$ and second-oder accuracy for $\cH_{x}$.  
For $\sigma=10^{2}$, the errors do not drop below $10^{-6}$ because of differences between the two-moment model and the diffusion equation used to obtain the analytic solution.  
For larger values of the scattering opacity, the two-moment model agrees better with the diffusion model, and we observe convergence over the entire range of $N$.  
Schemes PA2 and PA2+ do not perform well on this test (for reasons discussed in Section~\ref{sec:imex}).  
For $\sigma_{\Scatt}=10^{2}$, errors in $\cJ$ and $\cH_{x}$ decrease with increasing $N$, but for $\sigma_{\Scatt}=10^{4}$, errors remain constant with increasing $N$ over the entire range.  

\subsection{Packed Beam}
\label{sec:packedBeam}

Next we consider a one-dimensional test with discontinuous initial conditions.  
The purpose of this test is to further gauge the accuracy of the two-moment model and demonstrate the robustness of the DG scheme for dynamics close to the boundary of the realizable set $\cR$.  
The computational domain is $D=\{x:x\in[-1,1]\}$, and the initial condition is obtained from a distribution function given by
\begin{equation}
  f(x,\mu)
  =\left\{
  \begin{array}{cl}
    1        & \text{if} ~ x\le x_{\mbox{\tiny D}}, ~ \mu\ge\mu_{\mbox{\tiny D}} \\
    \delta & \text{if} ~ x\le x_{\mbox{\tiny D}}, ~ \mu<   \mu_{\mbox{\tiny D}} \\
    \delta & \text{otherwise},
  \end{array}
  \right.
\end{equation}
so that, with $\mu_{\mbox{\tiny D}}=0$, $\vect{\cM}\equiv\vect{\cM}_{\mbox{\tiny L}}=\big(0.5\,(1+\delta),0.25\,(1-\delta)\big)^{T}$ for $x\le x_{\mbox{\tiny D}}$, and $\vect{\cM}\equiv\vect{\cM}_{\mbox{\tiny R}}=\big(\delta,0\big)^{T}$ for $x> x_{\mbox{\tiny D}}$, where $\delta>0$ is a small parameter ($\delta\ll1$).  
We let $\delta=10^{-8}$, so that the initial conditions are very close to the boundary of the realizable domain (cf. Figure~\ref{fig:RealizableSetFermionic}).  
The analytical solution can be easily obtained by solving the transport equation for all angles $\mu$ (independent linear advection equations), and taking the angular moments.  
The numerical results shown in this section were obtained with the third-order scheme (polynomials of degree $k=2$ and the SSPRK3 time stepper) using $400$ elements.  
The time step is set to $\dt=0.1\times\dx$

Figure~\ref{fig:PackedBeam} shows results for various times obtained with the two-moment model.  
In the upper panels we plot the number density, while the number flux density is plotted in the lower panels.  
Numerical solutions are plotted with solid lines, while the analytical solution is plotted with dashed lines.  
In the left panels, the algebraic maximum entropy closure of Cernohorsky \& Bludman (CB) \cite{cernohorskyBludman_1994} (cf. Eqs.~\eqref{eq:eddingtonFactor} and \eqref{eq:closureMECB}) was used, while in the right panels the Minerbo closure (cf. Eqs.~\eqref{eq:eddingtonFactorLow} and \eqref{eq:closureMECB}) was used.  
For this test, the use of the realizability-preserving limiter described in Section~\ref{sec:limiter} was essential in order to avoid numerical problems.  
For the results obtained with the CB closure, the limiter was enacted whenever moments ventured outside the realizable set given by Eq.~\eqref{eq:realizableSet}.  
For the results obtained with the Minerbo closure, which is not based on Fermi-Dirac statistics, we used a modified limiter, which was enacted when the moments ventured outside the realizable domain of positive distributions; i.e., not bounded by $f < 1$, so that $\cJ > 0$ and $\cJ > \vect{\cH}|$ (e.g., \cite{levermore_1984}; see red line in Figure~\ref{fig:RealizableSetFermionic}).  

\begin{figure}[H]
  \centering
  \begin{tabular}{cc}
    \includegraphics[width=0.5\textwidth]{figures/PackedBeam_ME_CB} &
    \includegraphics[width=0.5\textwidth]{figures/PackedBeam_ME_MI}
  \end{tabular}
   \caption{Numerical results from the packed beam problem at various times: $t=0$ (cyan), $t=0.2$ (magenta), $t=0.4$ (blue), and $t=0.8$ (black).  Results obtained with the Cernohorsky \& Bludman closure are displayed in the left panels, while results obtained with the Minerbo closure are displayed in the right panels.  The analytical solution (dashed lines) is also plotted.}
  \label{fig:PackedBeam}
\end{figure}

As can be seen in Figure~\ref{fig:PackedBeam}, with the CB closure the numerical solution obtained with the two-moment model tracks the analytic solution well, while with the Minerbo closure the numerical solution deviates substantially from the analytic solution.  
With the Minerbo closure, the solution also evolves outside the realizable domain for Fermi-Dirac statistics.  

In the left panel in Figure~\ref{fig:PackedBeam_Realizability} we plot $\gamma(\vect{\cM})=\big(1-\cJ\big)\,\cJ-|\vect{\cH}|$ versus position for various times.  
With the Minerbo closure, $\gamma(\vect{\cM})$ becomes negative in regions of the computational domain (dashed lines), while $\gamma(\vect{\cM})$ remains positive for all $x$ and $t$ the CB closure.  
In the right panel of Figure~\ref{fig:PackedBeam_Realizability} we plot the numerical solutions in the $(\cH,\cJ)$-plane.  
Initially, the moments are located in two points: $\vect{\cM}_{\mbox{\tiny L}}$ and $\vect{\cM}_{\mbox{\tiny R}}$, for $x\le0$ and $x>0$, respectively (marked by circles in Figure~\ref{fig:PackedBeam_Realizability}).  
For $t>0$, the solutions trace out curves in the $(\cH,\cJ)$-plane, connecting $\vect{\cM}_{\mbox{\tiny L}}$ and $\vect{\cM}_{\mbox{\tiny R}}$.  
With the CB closure, the solution curve (blue points) follows the boundary of the realizable set $\cR$ defined in Eq.~\eqref{eq:realizableSet} (cf. black line in Figure~\ref{fig:PackedBeam_Realizability}).  
With the Minerbo closure (magenta points), the solution follows a different curve --- outside the realizable domain for distribution functions bounded by $f\in(0,1)$, but inside the realizable domain of positive distributions (cf. red line in Figure~\ref{fig:PackedBeam_Realizability}).  
We have also run this test using the algebraic maximum entropy closure of Larecki \& Banach \cite{lareckiBanach_2011} and the simpler Kershaw-type closure in \cite{banachLarecki_2017a}.  
The numerical solutions obtained with both of these closures follow the analytic solution well, and remain within the realizable set $\cR$.  
We point out that simply using the realizability-preserving limiter described in Section~\ref{sec:limiter} with the Minerbo closure does not result in a realizability-preserving scheme for Fermi-Dirac statistics because of the properties of this closure discussed in Section~\ref{sec:algebraicClosure}, and plotted in the right panel of Figure~\ref{fig:MabWithDifferentClosure}.  

\begin{figure}[H]
  \centering
  \begin{tabular}{cc}
    \includegraphics[width=0.485\textwidth]{figures/PackedBeam_Realizability} &
    \includegraphics[width=0.485\textwidth]{figures/PackedBeam_RealizableDomain}
  \end{tabular}
   \caption{In the left panel, $\gamma(\vect{\cM})=(1-\cJ)\,\cJ-|\vect{\cH}|$ is plotted versus $x$ for various times in the packed beam problem: $t=0$ (cyan), $t=0.2$ (magenta), $t=0.4$ (blue), and $t=0.8$ (black).  Results obtained with the CB closure, which remain positive throughout the evolution, are plotted with solid lines, while results obtained with the Minerbo closure are plotted with dashed lines.  In the right panel, the moments are plotted in the $(\cH,\cJ)$-plane for the same times as in the left panel.  Results obtained with the CB and Minerbo closures are plotted in blue and magenta, respectively.  The solid black and red lines are contours where $(1-\cJ)\,\cJ=\cH$ and $\cJ=\cH$, respectively.  The initial states are marked with black circles.}
  \label{fig:PackedBeam_Realizability}
\end{figure}

\subsection{Fermion Implosion}
\label{sec:fermionImplosion}

The next test is inspired by line source benchmark (cf. \cite{brunner_2002,garrettHauck_2013}), which is a challenging test for approximate transport algorithms.  
The original line source test consists of an initial delta function particle distribution in radius $R=|\vect{x}|$; i.e., $f_{0}=\delta(R)$.  
For $t>0$, a radiation front propagates in the radial direction, away from $R=0$.  
Apart from capturing details of the exact transport solution, maintaining realizability of the two-moment solution is challenging.  

Here, a modified version of the line source --- dubbed \emph{Fermion Implosion},  designed to test the realizability-preserving properties of the two-moment model for fermion transport --- is computed on a two-dimensional domain $D=\{\vect{x}\in\bbR^{2}:x^{1}\in[-1.28,1.28], x^{2}\in[-1.28,1.28]\}$.  
Instead of initializing with a delta function, we follow the initialization procedure in \cite{garrettHauck_2013}, and approximate the initial condition using an isotropic Gaussian distribution function.  
However, different from \cite{garrettHauck_2013}, the initial distribution function is bounded $f_{0}\in(0,1)$, and reaches a minimum in the center of the computational domain (hence implosion)
\begin{equation}
  f_{0}
  =1-\max\Big[\,e^{-R^{2}/(2\,\sigma_{0}^{2})},10^{-8}\,\Big].  
\end{equation}
We set $\sigma_{0}=0.03$, and evolve to a final time of $t=1.0$.  
We run this test using a grid of $512^{2}$ elements, polynomials of degree $k=1$, and the SSPRK2 time stepping scheme with $\dt=0.1\times\dx^{1}$.  
(There are no collisions included in this test; i.e., $\sigma_{\Ab}=\sigma_{\Scatt}=0$.)  
For comparison, we present results using the algebraic closures of Cernohorsky \& Bludman (CB) and Minerbo.  

\begin{figure}[H]
  \centering
  \begin{tabular}{cc}
    \includegraphics[width=0.495\textwidth]{figures/Implosion_Image} &
    \includegraphics[width=0.495\textwidth]{figures/Implosion_Lineout} \\
    \includegraphics[width=0.495\textwidth]{figures/Implosion_RealizableDomain} &
    \includegraphics[width=0.495\textwidth]{figures/Implosion_LimiterParameters}
  \end{tabular}
   \caption{Numerical results for the Fermion Implosion problem, computed both the CB and Minerbo closures.  Spatial distribution of the number density $\cJ$ (CB closure only) at $t=1$ (upper left panel).  In the upper right panel we plot the number density $\cJ$ versus radius $R=|\vect{x}|$ for various times ($t=0$, $0.1$, $0.2$, and $0.4$) for the CB (blue), the Minerbo (magenta) closures, and the reference transport solution (dashed black lines).  (The initial condition, which is the same for all models, is plotted with cyan.)  Numerical solutions in the $(\cH_{x},\cJ)$-plane (lower left panel), for the same times as plotted in the upper right panel are plotted for CB and Minerbo.  Limiter parameters $\vartheta_{1}$ (solid) and $\vartheta_{2}$ (dashed) in Section~\ref{sec:limiter} (minimum over the whole computational domain) versus time (lower right panel).}
  \label{fig:Implosion}
\end{figure}

Numerical results for the Fermion Implosion problem are plotted in Figure~\ref{fig:Implosion}.  
For $t>0$, the low-density region in the center of the computational domain is quickly filled in, and a cylindrical perturbation propagates radially away from the center.  
For the model with the CB closure, this perturbation, seen as a depression in the density relative to the ambient medium, has reached $R\approx1$ for $t=1$ (upper left panel in Figure~\ref{fig:Implosion}).  
The right panel in Figure~\ref{fig:Implosion} illustrates the difference in dynamics resulting from the two closures.  
(We also plot a reference transport solution obtained using the filtered spherical harmonics scheme described in \cite{garrettHauck_2013}; dashed black lines.\footnote{Kindly provided by Dr. Ming Tse Paul Laiu (private communications).})  
With the CB closure (blue lines), the central density increases towards the maximum value of unity, and an low-density pulse propagates radially.  
The amplitude of the pulse decreases with time due to the geometry of the problem.  
For $t=0.4$, the peak depression in located around $R=0.34$ (with $\cJ\approx0.96$).  
With the Minerbo closure, the central density continues to increase beyond unity, and reaches a maximum of about $\cJ\approx1.37$ at $t=0.1$.  
The central density starts to decrease beyond this point in time, and a steepening pulse propagates radially away from the center.  
(This pulse is trailing the pulse in the model computed with the CB closure.)  
At $t=0.4$, a discontinuity appears to have formed around $R=0.2$, resulting in numerical oscillations.  
Except for the realizability-enforcing limiter (which is not triggered for this model), no other limiters are used to prevent numerical oscillations.  
Although the solutions obtained with the two-moment model differ from the reference transport solution, the results obtained with the CB closure are in closer agreement with the transport solution.  
This is likely because the CB closure is consistent with the bound $f<1$ satisfied by the transport solution in this test.  
(For tests involving lower occupancies, the CB and Minerbo closures are expected to perform similarly.)  
In the lower left panel in Figure~\ref{fig:Implosion}, the moments are plotted in the $(\cH_{x},\cJ)$-plane for the same times as plotted in the upper left panel.  
(Each dot represents the moments at a specific spatial point and time.)  
Initially, $\vect{\cH}=0$, and all the moments lie on the line connecting $(0,0)$ and $(0,1)$; cyan points.  
With the CB closure (blue points), the moments are confined to evolve inside the realizable domain $\cR$ (black), while with the Minerbo closure, the moments are not confined to $\cR$, but to the region above the red lines (the realizable domain for moments of positive distribution functions), and this is the reason for the difference in dynamics in the two models.  
For the model with the CB closure, some moments evolve very close to the boundary of the realizable domain, and the positivity limiter is continuously triggered to damp these moments towards the cell average, which is realizable by the design of the numerical scheme.  
In the lower right panel in Figure~\ref{fig:Implosion} we plot the limiter parameters $\vartheta_{1}$ (solid) and $\vartheta_{2}$ (dashed) (cf. \eqref{eq:limitDensity} and \eqref{eq:limitMoments}) versus time for the CB closure model (blue) and the Minerbo closure model (magenta); the minimum over the whole computational domain is plotted.  
For the CB closure model, the limiter is triggered to prevent both density overshoots and $\gamma(\cM)<0$.  
Late in the simulation ($t\gtrsim0.7$), the minimum value of $\vartheta_{2}$ is around $0.1$.  
For the model using the Minerbo closure, the limiter is not triggered ($\vartheta_{1}=\vartheta_{2}=1$).  

\subsection{Homogeneous Sphere}
\label{sec:homogeneousSphere}

The homogeneous sphere test (e.g., \cite{smit_etal_1997}) considers a sphere with radius $R$.  
Inside the sphere (radius $<R$), the absorption opacity $\sigma_{\Ab}$ and the equilibrium distribution function $f_{0}$ are set to constant values.  
The scattering opacity $\sigma_{\Scatt}$ is set to zero in this test (i.e., $\xi=1$).  
Outside the sphere, the absorption opacity is zero.  
The steady state solution, obtained by solving the transport equation in spherical symmetry, is given by
\begin{equation}
  f_{\mbox{\tiny A}}(r,\mu)=f_{0}\,\big(1-e^{-\chi_{0}\,s(r,\mu)}\big),
  \label{eq:distributionHomogeneousSphere}
\end{equation}
where $r=|\vect{x}|$, 
\begin{equation}
  s(r,\mu)
  =\left\{
  \begin{array}{lll}
    r\,\mu+R\,g(r,\mu) & \mbox{if}\quad r<R, & \mu\in[-1,+1], \\
    2\,R\,g(r,\mu) & \mbox{if}\quad r \ge R, & \mu\in[(1-(R/r)^{2})^{1/2},+1], \\
    0 & \mbox{otherwise},
  \end{array}
  \right.
\end{equation}
and $g(r,\mu)=[1-(r/R)^{2}(1-\mu^{2})]^{1/2}$.  
Thus, $f_{\mbox{\tiny A}}(r,\mu)\in(0,f_{0})~\forall~r,\mu$.  

Here, this test is computed using a three-dimensional Cartesian domain $D=\{\vect{x}\in\bbR^{3}:x^{1}\in[0,2], x^{2}\in[0,2], x^{3}\in[0,2]\}$.  
Because of the symmetry of the problem, and to save computational resources, we only compute the solution in one octant.  
On the inner boundaries, we impose reflecting boundary conditions, while we impose 'homogeneous' boundary conditions on the outer boundary in all three coordinate dimensions; i.e., values for all moments in a boundary element are set equal to the corresponding values in the nearest element just inside $D$.  
Since this test is computed with Cartesian coordinates using a relatively low spatial resolution ($64^{3}$), we have found it necessary to smooth out the opacity over a finite radial extent to avoid numerical artifacts due to a discontinuous absorption opacity.  
Specifically, we use an absorption opacity of the following form
\begin{equation}
  \sigma_{\Ab}(r)=\f{\sigma_{\Ab,0}}{(r/R_{0})^{p}+1}.  
\end{equation}
We set $f_{0}=1$, and compute three versions of this test: one with $\sigma_{\Ab,0}=1$, $R_{0}=1$, and $p=80$ (Test~A), one with $\sigma_{\Ab,0}=10$, $R_{0}=1$, and $p=80$ (Test~B), and one with $\sigma_{\Ab}=10^{3}$, $R_{0}=0.85$, and $p=40$ (Test~C).  
(These values for $R_{0}$ and $p$ result in similar radius for where the optical depth equals $2/3$ in Test~B and Test~C.)
We compute until $t=5$, when the system has reached an approximate steady state.  
In all the tests, we use the IMEX scheme PD-ARS with $\dt=0.1\times\dx^{1}$ --- the least compute-intensive of the convex-invariant IMEX schemes presented here.  
The main purpose of this test is to compare the results obtained using the different algebraic closures discussed in Section~\ref{sec:algebraicClosure}.  

In Figure~\ref{fig:HomogeneousSphere}, we plot results obtained for all tests at $t=5$: Test~A (top panels), Test~B (middle panels), and Test~C (bottom panels).  
The particle density $\cJ$ and the flux factor $h=|\vect{\cH}|/\cJ$ (left and right panels, respectively) are plotted versus radius $r=|\vect{x}|$.  
In each panel, results obtained with the various algebraic closures discussed in Section~\ref{sec:algebraicClosure} are plotted: Minerbo (magenta), CB (blue), BL (green), and Kershaw (cyan).  
The analytical solution is also plotted (dashed black lines).  
\begin{figure}[H]
  \centering
  \begin{tabular}{cc}
    \includegraphics[width=0.5\textwidth]{figures/HomogeneousSphere_ClosureComparison_Chi_1e0_Density}
    \includegraphics[width=0.5\textwidth]{figures/HomogeneousSphere_ClosureComparison_Chi_1e0_FluxFactor} \\
    \includegraphics[width=0.5\textwidth]{figures/HomogeneousSphere_ClosureComparison_Chi_1e1_Density}
    \includegraphics[width=0.5\textwidth]{figures/HomogeneousSphere_ClosureComparison_Chi_1e1_FluxFactor} \\
    \includegraphics[width=0.5\textwidth]{figures/HomogeneousSphere_ClosureComparison_Chi_1e3_Density}
    \includegraphics[width=0.5\textwidth]{figures/HomogeneousSphere_ClosureComparison_Chi_1e3_FluxFactor}
  \end{tabular}
   \caption{Results obtained for the homogeneous sphere problem with the two-moment model for different values of the absorption opacity $\sigma_{\Ab,0}$: $1$ (top panels), $10$ (middle panels), and $1000$ (bottom panels).  The particle density (left panels) and the flux factor (right panels) are plotted versus radius.  Numerical results obtained with the algebraic closures of Minerbo (magenta), CB (blue), BL (green), and Kershaw (cyan) are compared with the analytic solution (dashed black lines).}
  \label{fig:HomogeneousSphere}
\end{figure}

We find good overall agreement between the results obtained with the two-moment model and the analytical solution.  
Partly due to the smoothing of the absorption opacity around the surface, the numerical and analytical solutions naturally differ around $r=1$.  
Aside from some differences discussed in more detail below, the numerical and analytical solutions --- for all values of the absorption opacity $\sigma_{\Ab,0}$ and all closures --- agree well as $r$ tends to zero, as well as when $r\gg1$.  
The results obtained with the maximum entropy closures CB and BL are practically indistinguishable on the plots.  
This is consistent with the similarity of the Eddington factors for these two closures, as shown in Figure~\ref{fig:EddingtonFactorsWithDifferentClosure}.  
We also find that the results obtained with the fermionic Kershaw closure agree well with the maximum entropy closures based on Fermi-Dirac statistics (CB and BL).  
From the plots of the particle density (left panels in Figure~\ref{fig:HomogeneousSphere}), the results obtained with all the closures, including Minerbo, appear very similar.  
(For Test~A, the particle density obtained with the Minerbo closure deviates the most from the analytic solution inside $r\approx0.75$; upper left panel).  
From the plots of the flux factor (right panels in Figure~\ref{fig:HomogeneousSphere}), it is evident that the results obtained with the Minerbo closure --- the only closure not based on Fermo-Dirac statistics --- deviates the most from the analytic solution outside $r=1$, where the flux factor is consistently higher than the analytical solution for all values of $\sigma_{\Ab,0}$.  
The fermionic closures (CB, BL, and Kershaw) track the analytic solution better.  
Similar agreement between the numerical and analytical solutions was reported by Smit et al. \cite{smit_etal_1997}, when using the CB maximum entropy closure with $f_{0}=0.8$ and an unsmoothed absorption opacity $\sigma_{\Ab}=4$.  
We also note that our results appear to be somewhat at odds with the results recently reported by Murchikova et al. \cite{murchikova_etal_2017}, who compared results obtained with the two-moment model using a large number of algebraic closures for this same problem (albeit using an unsmoothed and slightly different value for the absorption opacity).  
Murchikova et al. do not plot the particle density, but find essentially no difference in the flux factor and the Eddington factor when comparing results obtained with the maximum entropy closures of Minerbo and Cernohorsky \& Bludman (CB).  

In Figure~\ref{fig:HomogeneousSphereRealizability}, we further compare the results obtained when using the Minerbo and CB closures by plotting the solutions to the homogeneous sphere problem for Test~C at $t=5$ in the $(|\vect{\cH}|,\cJ)$-plane (cf. the realizable domain in Figure~\ref{fig:RealizableSetFermionic}).  
The numerical solution at each spatial point is represented by a blue (CB) or magenta (Minerbo) dot in the panels.  
In the lower two panel we zoom in on the results obtained with the two closures around the top and lower right regions of the realizable domain (lower left and lower right panel, respectively; cf. green boxes in the upper right panel).  
\begin{figure}[H]
  \centering
  \includegraphics[width=1.0\textwidth]{figures/HomogeneousSphere_Realizability_Chi_1e3_CBandMinerbo}
  \begin{tabular}{cc}
    \includegraphics[width=0.5\textwidth]{figures/HomogeneousSphere_Realizability_Chi_1e3_CBandMinerbo_Box1}
    \includegraphics[width=0.5\textwidth]{figures/HomogeneousSphere_Realizability_Chi_1e3_CBandMinerbo_Box2}
  \end{tabular}
   \caption{Scatter plots of the numerical solution to the homogeneous sphere problem for Test~C in the $(|\vect{\cH}|,\cJ)$-plane.  Results obtained with the CB and Minerbo closures are plotted in the upper panel, blue and magenta points, respectively.  Zoom-ins on the solutions obtained with the two closures are plotted in the lower two panels (cf. green boxes in the upper panel).  The boundaries of the realizable domains $\cR$ and $\cR^{+}$ are indicated with solid black and solid red curves, respectively.  See text for further details.  }
  \label{fig:HomogeneousSphereRealizability}
\end{figure}
As can be seen in the upper panel in Figure~\ref{fig:HomogeneousSphereRealizability}, the solutions to the homogeneous sphere problem obtained with the two closures trace out distinct curves relative to the realizable domain $\cR$, whose boundary is indicated by solid black curves in each panel.  
When using the CB closure, the realizability-preserving DG-IMEX scheme developed here maintains solutions within $\cR$.  
When using the Minerbo closure, the appropriate realizable domain is given by $\cR^{+}$ (cf. Eq.~\eqref{eq:realizableSetPositive}), whose boundary is indicated by solid red lines in Figure~\ref{fig:HomogeneousSphereRealizability}, and we find that the numerical solution ventures outside $\cR$.  
Near the surface around $r=1$, the number density slightly exceeds unity (lower left panel), while for larger radii, the computed flux may exceed the value allowed by Fermi-Dirac statistics (lower right panel).  
\section{Summary and Conclusions}
\label{sec:conclusions}

We have developed a realizability-preserving DG-IMEX scheme for a two-moment model of fermion transport.  
The scheme employs algebraic closures based on Fermi-Dirac statistics and combines a time step restriction (CFL condition), a realizability-enforcing limiter, and a convex-invariant time integrator to maintain point-wise realizability of the moments.  
Since the realizable domain is a convex set, the realizability-preserving property is obtained from convexity arguments, building on the framework in \cite{zhangShu_2010a}.  

In the applications motivating this work, the collision term is stiff in regions of the computational domain, and we have considered IMEX schemes to avoid treating the transport operator implicitly.  
We have considered two recently proposed second-order accurate, convex-invariant IMEX schemes \cite{chertock_etal_2015,hu_etal_2018}, that restore second-order accuracy with an implicit correction step.  
However, we are unable to prove realizability (without invoking a very small time step) with the approach in \cite{chertock_etal_2015}, and we have demonstrated that the approach in \cite{hu_etal_2018} does not perform well in the diffusion limit.  
For these reasons, we have resorted to first-order, convex-invariant IMEX schemes.  
While the proposed scheme (dubbed PD-ARS) is formally only first-order accurate, it works well in the diffusion limit, is convex-invariant with a reasonable time step, and reduces to the optimal second-order accurate explicit SSP-RK scheme in the streaming limit.  

For each stage of the IMEX scheme, the update of the cell-averaged moments can be written as a convex combination of forward Euler steps (implying the Shu-Osher form for the explicit part), followed by a backward Euler step.  
Realizability of the cell-averaged moments due to the explicit part requires the DG solution to be realizable in a finite number of quadrature points in each element and the time step to satisfy a CFL condition.  
For the backward Euler step, realizability of the cell-averages follows easily from the simple form of the collision operator (which includes emission, absorption, and isotropic scattering without energy exchange), and is independent of the time step.  
The CFL condition is then solely due to the transport operator, and the time step can be as large as that of the forward Euler scheme applied to the explicit part of the cell-average update.  
After each stage update, the limiter enforces moment realizability point-wise by damping towards the realizable cell average.  
Numerical experiments are presented to demonstrate the accuracy and realizability-preserving property of the DG-IMEX scheme.  
The applicability of the PD-ARS scheme is not restricted to the fermionic two-moment model.  
It may therefore be a useful option in other applications of kinetic theory where physical constraints confine solutions to a convex set and capturing the diffusion limit is important.  

Realizability of the fermionic two-moment model depends sensitively on the closure procedure.  
For the algebraic closures adapted in this work, realizability of the scheme demands that lower and upper bounds on the Eddington factor are satisfied \cite{levermore_1984,lareckiBanach_2011}.  
The Eddington factors deriving from the maximum entropy closures of Cernohorsky \& Bludman \cite{cernohorskyBludman_1994} and Larecki \& Banach \cite{lareckiBanach_2011}, and the Kershaw-type closure of Larecki \& Banach \cite{banachLarecki_2017a} all satisfy these bounds and are suitable for the fermionic two-moment model.  
Further approximations of the closure procedure (e.g., employing the low occupancy limit, which results in the Minerbo closure \cite{minerbo_1978} when starting with the maximum entropy closure of \cite{cernohorskyBludman_1994}) is not compatible with realizability of the fermionic two-moment model, and we caution against this approach to modeling particle systems governed by Fermi-Dirac statistics; particularly if the low occupancy approximation is unlikely to hold (e.g., when modeling neutrino transport in core-collapse supernovae).  

In this work, we started with a relatively simple kinetic model.  
In particular, we adopted Cartesian coordinates, and assumed a linear collision operator and a fixed material background.  
Scattering with energy exchange and relativistic effects (e.g., due to a moving material and the presence of a strong gravitational field) were not included.  
To solve more realistic problems of scientific interest, some or all of these physical effects will have to be included.  
In the context of developing realizability-preserving schemes, these extensions will provide significant challenges suitable for future investigations, for which the scheme presented here may serve as a foundation.  
\appendix

\section{Butcher Tableau for IMEX Schemes}

For easy reference, we include the Butcher tableau for the IMEX schemes considered in this paper, which can be written in the standard double Butcher tableau
\begin{equation}
  \begin{array}{c | c}
    \tilde{\vect{c}} & \tilde{A} \\
    \hline
    & \tilde{\vect{w}}^{T}
  \end{array}
  \qquad
  \begin{array}{c | c}
    \vect{c} & A \\
    \hline
    \alpha & \vect{w}^{T}
  \end{array},
\end{equation}
where the \emph{explicit tableau} (left; components adorned with a tilde) represent the explicit part of the IMEX scheme, and the \emph{implicit tableau} (right; unadorned components) represent the implicit part of the IMEX scheme.  
For $s$ stages, $\tilde{A}=(\tilde{a}_{ij})$, $\tilde{a}_{ij}=0$ for $j\ge i$, and $A=(a_{ij})$, $a_{ij}=0$ for $j>i$, are $s\times s$ matrices, and $\tilde{\vect{w}}=(\tilde{w}_{1},\ldots,\tilde{w}_{s})^{T}$ and $\vect{w}=(w_{1},\ldots,w_{s})^{T}$.  
The vectors $\tilde{\vect{c}}=(\tilde{c}_{1},\ldots,\tilde{c}_{s})^{T}$ and $\vect{c}=(c_{1},\ldots,c_{s})^{T}$, used for non autonomous systems, satisfy $\tilde{c}_{i}=\sum_{j=1}^{i-1}\tilde{a}_{ij}$ and $c_{i}=\sum_{j=1}^{i}a_{ij}$.  
For the implicit tableau, we have included the scalar $\alpha$, used for the correction step in Eq.~\eqref{eq:imexCorrection}.  

\paragraph{IMEX PA2}

A second-order accurate, positivity-preserving IMEX scheme of type $A$ (the matrix $A$ is invertible) was given in \cite{hu_etal_2017}.  
We refer to this scheme as IMEX PA2.  
For this scheme, the non-zero components of $\tilde{A}$ and $A$ are given by
\begin{align}
  \tilde{a}_{21} &= 0.7369502715, \nonumber \\
  \tilde{a}_{31} &= 0.3215281691, \quad \tilde{a}_{32} = 0.6784718309, \nonumber \\
  a_{11} &= 0.6286351712, \nonumber \\
  a_{21} &= 0.2431004655, \quad a_{22} = 0.1959392570, \nonumber \\
  a_{31} &= 0.4803651051, \quad a_{32} = 0.0746432814, \quad a_{33} = 0.4449916135. \nonumber
\end{align}
The coefficient in the correction step is $\alpha = 0.2797373792$ and the CFL constant is $c_{sch} = 0.5247457524$.
This scheme is globally stiffly accurate (GSA), so that $\tilde{w}_{i}=\tilde{a}_{3i}$ and $w_{i}=a_{3i}$ for $i\le3$.

\paragraph{IMEX PA2+}

We have found another second-order accurate, positivity-preserving IMEX scheme of type $A$, which we refer to as IMEX PA2+.  
This scheme allows for has a larger $c_{sch}$ than IMEX PA2 (i.e., a larger time step while still maintaining positivity).  
The scheme chosen from random sampling of the IMEX parameter space, and was the scheme with the largest $c_{sch}$ that satisfied both the order conditions for second order accuracy and possessed the positivity-preserving property.  
For IMEX PA2+, ($c_{sch} = 0.895041066934$). 
The non-zero components of $\tilde{A}$ and $A$ are given by
\begin{align}
  \tilde{a}_{21} &= 0.909090909090909, \nonumber \\
  \tilde{a}_{31} &= 0.450000000000000, \quad \tilde{a}_{32} = 0.550000000000000, \nonumber \\
  a_{11} &= 0.521932391842510, \nonumber \\
  a_{21} &= 0.479820781424967, \quad a_{22} = 0.002234534340252, \nonumber \\
  a_{31} &= 0.499900000000000, \quad a_{32} = 0.001100000000000, \quad a_{33} = 0.499000000000000. \nonumber
\end{align}
The coefficient in the correction step is $\alpha = 0.260444263529413$.  
This scheme is also GSA; $\tilde{w}_{i}=\tilde{a}_{3i}$ and $w_{i}=a_{3i}$ for $i\le3$.

\paragraph{IMEX PC2}

Another positivity-preserving IMEX scheme was given in \cite{mcclareen_2008} (referred to as a semi-implicit predictor-corrector method in \cite{mcclareen_2008}).  
This scheme can be written in the double Butcher tableau form, and we refer to this scheme as IMEX PC2.  
The non-zero components of $\tilde{A}$ and $A$ are given by
\begin{align}
  \tilde{a}_{21} &= 0.5, \nonumber \\
  \tilde{a}_{32} &= 1, \nonumber \\
  a_{22} &= 0.5, \nonumber \\
  a_{33} &= 1.0. \nonumber
\end{align}
And $\tilde{w}_{i}=\tilde{a}_{3i} = w_{i}=a_{3i}$ for $i\le3$.
A common choice of this scheme is $c_{sch} = 0.3$.

\paragraph{IMEX RKCB2}

We compare the performance of the positivity-preserving IMEX schemes with two other, non-positive, schemes.  
The first one is the second-order accurate IMEX scheme given in \cite{cavaglieriBewley2015}.  
We refer to this scheme as IMEX RKCB2.  
The non-zero components of $\tilde{A}$ and $A$ are given by
\begin{align}
  \tilde{a}_{21} &= 2/5, \nonumber \\
  \tilde{a}_{32} &= 1, \nonumber \\
  a_{22} &= 2/5, \nonumber \\
  a_{32} &= 5/6, \quad a_{33} = 1/6. \nonumber
\end{align}
And $w_{i} = a_{3i} = \tilde{w}_{i}$ for stiff accuracy.
CFL ...

\paragraph{IMEX SSP2332}

The second scheme we use for comparison is the second-order accurate IMEX scheme given in \cite{pareschiRusso_2005}.  
We refer to this scheme as IMEX SSP2332.  
The non-zero components of $\tilde{A}$ and $A$ are given by
\begin{align}
  \tilde{a}_{21} &= 1/2, \nonumber \\
  \tilde{a}_{31} &= 1/2, \quad \tilde{a}_{32} = 1/2, \nonumber \\
  a_{11} &= 1/4, \nonumber \\
  a_{22} &= 1/4, \nonumber \\
  a_{31} &= 1/3, \quad a_{32} = 1/3, \quad a_{33} = 1/3. \nonumber
\end{align}
And $w_{i} = a_{3i} = \tilde{w}_{i}$ for stiff accuracy.
CFL ...

\paragraph{SSPRK2 and SSPRK3}

A second-order accurate, strong-stability-preserving explicit Runge-Kutta scheme, referred as SSPRK2, has the following non-zero components:
\begin{align}
  \tilde{a}_{21} &= 1, \nonumber \\
  \tilde{w}_{1} &= 1/2, \quad \tilde{w}_{2} = 1/2. \nonumber 
\end{align}
A third-order accurate, strong-stability-preserving explicit Runge-Kutta scheme, was referred as SSPRK3.
It has the following non-zero components:
\begin{align}
  \tilde{a}_{21} &= 1, \nonumber \\
  \tilde{a}_{31} &= 1/4, \quad \tilde{a}_{32} = 1/4, \nonumber \\
  \tilde{w}_{1} &= 1/6, \quad \tilde{w}_{2} = 1/6, \quad \tilde{w}_{3} =2/3 \nonumber
\end{align}

\section*{References}

\bibliography{./references/references.bib}
\end{document}