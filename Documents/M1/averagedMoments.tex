\documentclass[10pt]{article}

\usepackage{amsfonts}
\usepackage{amsmath}
\usepackage{amssymb}
\usepackage{amsthm}
\usepackage{booktabs}
\usepackage{mathrsfs}
\usepackage{graphicx}
\usepackage{cite}
\usepackage{times}
\usepackage{url}
\usepackage{hyperref}
\usepackage{lineno}
\usepackage{yhmath}
\usepackage{natbib}
\usepackage{../definitions}
\hypersetup{
  bookmarksnumbered = true,
  bookmarksopen=false,
  pdfborder=0 0 0,         % make all links invisible, so the pdf looks good when printed
  pdffitwindow=true,      % window fit to page when opened
  pdfnewwindow=true, % links in new window
  colorlinks=true,           % false: boxed links; true: colored links
  linkcolor=blue,            % color of internal links
  citecolor=magenta,    % color of links to bibliography
  filecolor=magenta,     % color of file links
  urlcolor=cyan              % color of external links
}

\newcommand{\ee}[1]{{\color{blue} EE:~#1}}
\newcommand{\Lo}{\textsc{L}}
\newcommand{\Hi}{\textsc{H}}
\newcommand{\trans}{\textsc{T}}
\newcommand{\dx}{\Delta x}

\newtheorem{define}{Definition}
\newtheorem{lemma}{Lemma}
\newtheorem{prop}{Proposition}
\newtheorem{rem}{Remark}
\newtheorem{theorem}{Theorem}

\begin{document}

\title{Cell Averaged and Energy Integrated Moments}
\author{Eirik Endeve}

\maketitle

Moments are defined as weighted integrals of the distribution function; i.e.,
\begin{equation}
  \vect{\mathcal{M}}(\varepsilon,\vect{x},t) = \f{1}{4\pi}\int_{\bbS^{2}}f(\omega,\varepsilon,\vect{x},t)\,\vect{w}(\omega)\,d\omega.  
\end{equation}

We let the computational domain be subdivided into elements
\begin{equation}
  \bK^{(p\bq)} = K_{\varepsilon}^{(p)} \times \bK_{\vect{x}}^{(\bq)},
\end{equation}
where the $p$th energy space element (energy bin) is
\begin{equation}
  K_{\varepsilon}^{(p)} = (\varepsilon_{p-\f{1}{2}},\varepsilon_{p+\f{1}{2}}),
\end{equation}
and the $d$-dimensional position space element (equivalent to a finite volume cell) is
\begin{equation}
  \bK_{\vect{x}}^{(\bq)} = \big\{\,\vect{x}~:~x^{i}\in(x_{q^{i}-\f{1}{2}}^{i},x_{q^{i}+\f{1}{2}}^{i}),~i=1,\ldots,d\,\big\}.  
\end{equation}
We use multi index notation (bold indices) to refer to spatial elements in a multidimensional spatial domain; e.g., $\bq=\{q^{1},\ldots,q^{d}\}$.  
We define the volume of the element as
\begin{equation}
  \Delta V^{(p\bq)} 
  = \int_{\bK^{(p\bq)}}\varepsilon^{2}\,d\varepsilon\,d\vect{x}
  = \int_{K_{\varepsilon}^{(p)}}\varepsilon^{2}\,d\varepsilon\,\int_{\bK_{\vect{x}}^{(\bq)}}d\vect{x}
  = \Delta V_{\varepsilon}^{(p)}\,\Delta V_{\vect{x}}^{(\bq)}.
\end{equation}
The factor $\varepsilon^{2}$ comes from using spherical momentum space coordinates.  
Cartesian position space coordinates are assumed.  
We also define $\Delta\varepsilon_{p} = \varepsilon_{p+\f{1}{2}}-\varepsilon_{p-\f{1}{2}}$ and $\Delta \vect{x}_{\bq}=\prod_{i=1}^{d}(x_{q^{i}+\f{1}{2}}^{i}-x_{q^{i}-\f{1}{2}}^{i})$.  

The moments are approximated by
\begin{equation}
  \vect{\mathcal{M}}_{h}(\varepsilon,\vect{x})=\sum_{p\vect{q}}\chi(\bK^{(p\bq)})\,\vect{\mathcal{M}}_{h}^{(p\bq)}(\varepsilon,\vect{x}),
\end{equation}
where $\chi$ is the indicator function, and the approximation in each element $\bK^{(p\bq)}$ is given by
\begin{equation}
  \vect{\mathcal{M}}_{h}^{(p\bq)}(\varepsilon,\vect{x}) = \sum_{i\bj}\vect{\mathcal{M}}_{i\bj}^{(p\bq)}\,\ell_{i}(\varepsilon)\,\ell_{\vect{j}}(\vect{x}).  
  \label{eq:nodalExpansion}
\end{equation}
(Here we have suppressed dependence on time $t$ for brevity.)
In Eq.~\eqref{eq:nodalExpansion}, $\ell_{i}(\varepsilon)$ and $\ell_{\bj}(\vect{x})$ are local basis polynomials, which we take to be Lagrange polynomials constructed from Gaussian quadrature points in each element.  

The element average (average over energy and position nodes in each element) is computed as
\begin{align}
  \langle\vect{\mathcal{M}}\rangle^{(p\bq)}
  &=\f{1}{\Delta V^{(p\bq)}}\int_{\bK^{(p\bq)}}\vect{\mathcal{M}}_{h}^{(p\bq)}(\varepsilon,\vect{x})\,\varepsilon^{2}\,d\varepsilon\,d\vect{x} \nonumber \\
  &=\f{\Delta\varepsilon_{p}\,\Delta\vect{x}_{\bq}}{\Delta V^{(p\bq)}}\sum_{i\bj}w_{i}\,w_{\bj}\,\varepsilon_{i}^{2}\,\vect{\mathcal{M}}_{i\bj}^{(p\bq)}.  
\end{align}
Energy integrated (grey) moments are computed as
\begin{equation}
  \vect{M}^{(\bq)}
  =\f{4\pi}{\Delta V_{\vect{x}}^{(\bq)}}\int_{\bbR^{+}}\int_{\bK_{\vect{x}}^{(\bq)}}\vect{\mathcal{M}}_{h}(\varepsilon,\vect{x})\,\varepsilon^{2}\,d\varepsilon\,d\vect{x}
  =4\pi\sum_{p}\langle\vect{\mathcal{M}}\rangle^{(p\bq)}\,\Delta V_{\varepsilon}^{(p)}.  
\end{equation}

\end{document}